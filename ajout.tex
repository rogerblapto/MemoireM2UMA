\documentclass[12pt]{article}
\usepackage{amsmath, amssymb, amsthm}
\usepackage[french]{babel}
\usepackage[utf8]{inputenc}
\usepackage[T1]{fontenc}

\title{Isomorphisme de représentations et caractères}
\author{}
\date{}


\newtheorem{propriete}{Propriété}
\newtheorem{theoreme}{Théorème}
\newtheorem{lemme}{Lemme}

\begin{document}
	
	\maketitle
	
	\begin{theoreme}
		Soient \( \rho, \rho' : G \to \mathrm{GL}(V), \mathrm{GL}(V') \) deux représentations complexes de dimension finie d’un groupe fini \( G \). Alors :
		\[
		\rho \cong \rho' \quad \Leftrightarrow \quad \chi_\rho = \chi_{\rho'},
		\]
		où \( \chi_\rho \) et \( \chi_{\rho'} \) sont les caractères des représentations \( \rho \) et \( \rho' \).
	\end{theoreme}
	
	\begin{proof}
		\textbf{Sens direct} : Supposons que \( \rho \cong \rho' \). Il existe alors un isomorphisme d’espaces vectoriels \( T : V \to V' \) tel que :
		\[
		T \circ \rho(g) = \rho'(g) \circ T \quad \text{pour tout } g \in G.
		\]
		Cela implique que \( \rho(g) \) et \( \rho'(g) \) sont semblables, donc ont la même trace :
		\[
		\chi_\rho(g) = \mathrm{Tr}(\rho(g)) = \mathrm{Tr}(\rho'(g)) = \chi_{\rho'}(g),
		\]
		pour tout \( g \in G \). Ainsi, \( \chi_\rho = \chi_{\rho'} \).
		
		\vspace{1em}
		\textbf{Sens réciproque} : Supposons que \( \chi_\rho = \chi_{\rho'} \). On sait que toute représentation complexe d’un groupe fini est complètement réductible (théorème de Maschke). On peut donc écrire :
		\[
		\rho \cong \bigoplus_{i} n_i \rho_i, \quad \rho' \cong \bigoplus_{i} n_i' \rho_i,
		\]
		où les \( \rho_i \) sont des représentations irréductibles non isomorphes deux à deux, et \( n_i, n_i' \in \mathbb{N} \) sont les multiplicités.
		
		Le caractère d’une somme directe est la somme des caractères :
		\[
		\chi_\rho = \sum_i n_i \chi_i, \quad \chi_{\rho'} = \sum_i n_i' \chi_i,
		\]
		où \( \chi_i \) est le caractère de \( \rho_i \).
		
		Comme \( \chi_\rho = \chi_{\rho'} \) et que les caractères irréductibles \( \chi_i \) sont linéairement indépendants dans l’espace des fonctions de classe, on en déduit que :
		\[
		n_i = n_i' \quad \text{pour tout } i.
		\]
		
		Ainsi, \( \rho \cong \rho' \).
		
	\end{proof}
	
	
	\newpage
	
	
	\begin{theoreme}
		Le degré d’un caractère irréductible est égal à la dimension de la représentation irréductible dont il est le caractère.
	\end{theoreme}
	
	\begin{proof}
		Soit \( G \) un groupe fini, et soit \( \rho : G \to \mathrm{GL}(V) \) une représentation irréductible complexe de \( G \), où \( V \) est un espace vectoriel complexe de dimension finie.
		
		On note \( \chi_\rho : G \to \mathbb{C} \) le caractère associé à la représentation \( \rho \). Le \emph{degré} du caractère \( \chi_\rho \) est défini par sa valeur en l’élément neutre du groupe :
		\[
		\chi_\rho(1) = \chi_\rho(e),
		\]
		où \( e \) est l’élément neutre de \( G \).
		
		Par définition du caractère :
		\[
		\chi_\rho(g) = \mathrm{Tr}(\rho(g)) \quad \text{pour tout } g \in G.
		\]
		
		En particulier, pour \( g = e \), on a :
		\[
		\chi_\rho(e) = \mathrm{Tr}(\rho(e)).
		\]
		
		Or, comme \( \rho \) est une représentation, on a \( \rho(e) = \mathrm{Id}_V \), donc :
		\[
		\chi_\rho(e) = \mathrm{Tr}(\mathrm{Id}_V) = \dim V.
		\]
		
		Ainsi,
		\[
		\chi_\rho(1) = \dim V.
		\]
		
		Cela montre que le degré du caractère irréductible est égal à la dimension de la représentation irréductible.
		
	\end{proof}
	
	
	\newpage
	
	\begin{propriete}
		Soit \( G \) un groupe. On note
		\[
		\widehat{G} := \mathrm{Hom}(G, \mathbb{C}^\times)
		\]
		l’ensemble des caractères de \( G \), c’est-à-dire l’ensemble des morphismes de groupes de \( G \) dans le groupe multiplicatif des complexes non nuls \( \mathbb{C}^\times \). Alors \( \widehat{G} \) muni de la multiplication point par point :
		\[
		(\chi_1 \cdot \chi_2)(g) := \chi_1(g) \cdot \chi_2(g), \quad \forall g \in G,
		\]
		forme un groupe abélien.
	\end{propriete}
	
	\begin{proof}
		\textbf{1. Fermeture.} Soient \( \chi_1, \chi_2 \in \widehat{G} \), c’est-à-dire deux morphismes de groupes \( G \to \mathbb{C}^\times \). On définit :
		\[
		(\chi_1 \cdot \chi_2)(g) := \chi_1(g) \cdot \chi_2(g), \quad \forall g \in G.
		\]
		Pour montrer que \( \chi_1 \cdot \chi_2 \in \widehat{G} \), vérifions que c’est un morphisme :
		\[
		(\chi_1 \cdot \chi_2)(gh) = \chi_1(gh) \chi_2(gh) = \chi_1(g)\chi_1(h) \cdot \chi_2(g)\chi_2(h)
		= (\chi_1(g)\chi_2(g))(\chi_1(h)\chi_2(h)) = (\chi_1 \cdot \chi_2)(g)(\chi_1 \cdot \chi_2)(h).
		\]
		
		\textbf{2. Associativité.} La multiplication dans \( \mathbb{C}^\times \) étant associative, on a pour \( \chi_1, \chi_2, \chi_3 \in \widehat{G} \) :
		\[
		((\chi_1 \cdot \chi_2) \cdot \chi_3)(g)
		= (\chi_1(g) \cdot \chi_2(g)) \cdot \chi_3(g)
		= \chi_1(g) \cdot (\chi_2(g) \cdot \chi_3(g))
		= (\chi_1 \cdot (\chi_2 \cdot \chi_3))(g).
		\]
		
		\textbf{3. Élément neutre.} L’application constante \( \mathbf{1}_G : G \to \mathbb{C}^\times \) définie par \( \mathbf{1}_G(g) = 1 \) pour tout \( g \in G \) est un morphisme de groupes (trivialement). Elle vérifie :
		\[
		(\chi \cdot \mathbf{1}_G)(g) = \chi(g) \cdot 1 = \chi(g), \quad \forall \chi \in \widehat{G}, \ \forall g \in G.
		\]
		
		\textbf{4. Inverses.} Pour \( \chi \in \widehat{G} \), définissons \( \chi^{-1} : G \to \mathbb{C}^\times \) par \( \chi^{-1}(g) = \chi(g)^{-1} \). On a :
		\[
		\chi^{-1}(gh) = (\chi(gh))^{-1} = (\chi(g)\chi(h))^{-1} = \chi(h)^{-1} \chi(g)^{-1} = \chi^{-1}(g)\chi^{-1}(h),
		\]
		(car \( \mathbb{C}^\times \) est abélien). Donc \( \chi^{-1} \in \widehat{G} \) et :
		\[
		(\chi \cdot \chi^{-1})(g) = \chi(g) \cdot \chi(g)^{-1} = 1 = \mathbf{1}_G(g).
		\]
		
		\textbf{5. Commutativité.} Pour \( \chi_1, \chi_2 \in \widehat{G} \), et \( g \in G \), on a :
		\[
		(\chi_1 \cdot \chi_2)(g) = \chi_1(g)\chi_2(g) = \chi_2(g)\chi_1(g) = (\chi_2 \cdot \chi_1)(g).
		\]
		
		\noindent
		Ainsi, \( \widehat{G} \) est un groupe abélien pour la multiplication point par point.
		
	\end{proof}
	
	\newpage
	
	
	\begin{theoreme}
		Soient \( E \) et \( F \) deux espaces vectoriels sur un corps \( \mathbb{K} \).  
		Alors il existe un isomorphisme canonique :
		\[
		E \otimes F \cong F \otimes E.
		\]
	\end{theoreme}
	
	\begin{proof}
		Nous construisons un isomorphisme linéaire entre \( E \otimes F \) et \( F \otimes E \), sans choisir de bases.
		
		\textbf{Étape 1 : Définition d'une application bilinéaire.}
		
		Considérons l'application :
		\[
		\beta : E \times F \to F \otimes E, \quad (e, f) \mapsto f \otimes e.
		\]
		Elle est bilinéaire car la multiplication tensorielle est linéaire en chaque variable. Par la propriété universelle du produit tensoriel, cette application induit un unique morphisme linéaire :
		\[
		T : E \otimes F \to F \otimes E, \quad \text{défini par } T(e \otimes f) = f \otimes e.
		\]
		
		\textbf{Étape 2 : Construction de l'inverse.}
		
		De façon symétrique, on définit :
		\[
		\gamma : F \times E \to E \otimes F, \quad (f, e) \mapsto e \otimes f,
		\]
		qui est également bilinéaire, et induit un morphisme linéaire :
		\[
		S : F \otimes E \to E \otimes F, \quad \text{défini par } S(f \otimes e) = e \otimes f.
		\]
		
		\textbf{Étape 3 : Vérification que \( T \) et \( S \) sont inverses.}
		
		Pour tout \( e \in E \), \( f \in F \) :
		\[
		(S \circ T)(e \otimes f) = S(f \otimes e) = e \otimes f,
		\]
		\[
		(T \circ S)(f \otimes e) = T(e \otimes f) = f \otimes e.
		\]
		Donc \( S \circ T = \mathrm{id}_{E \otimes F} \) et \( T \circ S = \mathrm{id}_{F \otimes E} \), ce qui prouve que \( T \) est un isomorphisme, avec inverse \( S \).
		
		\textbf{Conclusion :} L'application \( T \) est un isomorphisme canonique :
		\[
		\boxed{E \otimes F \cong F \otimes E}.
		\]
	\end{proof}
	
	
\end{document}
