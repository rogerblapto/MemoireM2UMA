\documentclass[french]{beamer}
\usepackage[T1]{fontenc}
\usepackage[utf8]{inputenc}
\usepackage[french]{babel}
\usepackage{amsmath, amssymb, amsthm}
\usepackage{graphicx}

\title{Représentations Linéaires des Groupes Infinis}
\author{Votre Nom}
\institute{Votre Université}
\date{\today}

\begin{document}
	
	\begin{frame}
		\titlepage
	\end{frame}
	
	\begin{frame}{Plan}
		\tableofcontents
	\end{frame}
	
	%------------------------------------------------------
	\section{Introduction}
	\begin{frame}{Une analogie musicale}
		Imagine que tu es dans une chorale où chaque chanteur représente une note musicale.\newline
		Le chef de chœur donne des instructions, et chaque chanteur modifie légèrement sa voix en fonction de ces instructions.
	\end{frame}
	
	\begin{frame}{Les concepts clés}
		\begin{itemize}
			\item \textbf{Le groupe} : L'ensemble des règles que le chef peut donner (exemple : monter ou descendre d'un ton, chanter plus fort, etc.).
			\item \textbf{La représentation linéaire} : Une façon de transformer ces instructions en nombres et en calculs pour comprendre leur effet sur les chanteurs.
			\item \textbf{Le caractère} : Une mesure globale de l'effet des instructions sur toute la chorale (exemple : est-ce que le volume global augmente ou diminue ?).
			\item \textbf{Les représentations irréductibles} : Les « groupes de chanteurs » les plus simples qui ne peuvent pas être décomposés davantage.
			\item \textbf{Les groupes infinis} : Imagine une chorale infinie où il y a toujours un chanteur en plus. On veut comprendre comment les règles du chef affectent cette chorale.
		\end{itemize}
	\end{frame}
	
	\begin{frame}{Objectif du projet}
			\begin{itemize}
			\item  L’espace vectoriel est l’ensemble des "sons" ou "signaux" que le groupe peut produire.
			\item Le groupe agit sur cet espace en transformant ces sons d’une certaine manière.
			\item  On étudie comment ces transformations se comportent et comment on peut les décomposer en structures plus simples.
			\item La façon dont ils chantent (volume, hauteur de la voix) est représentée par des nombres.
			\item On peut mélanger plusieurs voix (addition) ou amplifier une voix (multiplication par un nombre).
		\end{itemize}
		Ce projet consiste à traduire les règles d’un groupe en transformations mathématiques et à comprendre les structures fondamentales qui restent les mêmes, quelle que soit la taille du groupe.
	\end{frame}
	

	
	%------------------------------------------------------
	\section{Pourquoi Étendre aux Groupes Infinis ?}
	\begin{frame}{Modélisation des Systèmes Continus}
		\begin{itemize}
			\item Les groupes infinis modélisent des transformations continues (ex: rotations, translations).
			\item Applications en vision par ordinateur et en robotique.
			\item Utilisation en cryptographie post-quantique pour des protocoles plus robustes.
		\end{itemize}
	\end{frame}
	
	\begin{frame}{Lien avec l’Analyse de Fourier}
		\begin{itemize}
			\item Les groupes de Lie jouent un rôle clé en analyse harmonique.
			\item Permet la décomposition spectrale des signaux.
			\item Applications en traitement du signal et compression d’images.
		\end{itemize}
	\end{frame}
	
	%------------------------------------------------------
	\section{Le Caractère d’un Groupe Infini}
	\begin{frame}{Définition et Utilisation}
		\begin{itemize}
			\item Le caractère est une trace des matrices associées à une représentation.
			\item Il permet de classifier et de distinguer les représentations.
			\item Application en reconnaissance de formes et intelligence artificielle.
		\end{itemize}
	\end{frame}
	
	%------------------------------------------------------
	\section{Représentations Linéaires Irréductibles}
	\begin{frame}{Pourquoi sont-elles Importantes ?}
		\begin{itemize}
			\item Toute représentation peut être décomposée en représentations irréductibles.
			\item Elles simplifient la compréhension des structures complexes.
			\item Applications en physique quantique et en cryptographie.
		\end{itemize}
	\end{frame}
	
	%------------------------------------------------------
	\section{Applications Informatiques et Physiques}
	\begin{frame}{Applications en Informatique}
		\begin{itemize}
			\item Traitement d’image : reconnaissance de motifs via les caractères de groupes.
			\item Sécurité informatique : cryptographie et signatures numériques.
			\item Optimisation d’algorithmes en intelligence artificielle.
		\end{itemize}
	\end{frame}
	
	\begin{frame}{Applications en Physique}
		\begin{itemize}
			\item Mécanique quantique : classification des particules élémentaires.
			\item Théorie spectrale : lien entre mathématiques et physique.
			\item Modélisation en théorie des cordes et en cosmologie.
		\end{itemize}
	\end{frame}
	
	%------------------------------------------------------
	\section{Conclusion}
	\begin{frame}{Conclusion}
		\begin{itemize}
			\item L’étude des groupes infinis et de leurs représentations linéaires est cruciale.
			\item Applications variées en informatique, physique et mathématiques appliquées.
			\item Perspectives : avancées en intelligence artificielle et en sécurité quantique.
		\end{itemize}
	\end{frame}
	
\end{document}
