\documentclass[9pt]{beamer}
\usepackage[french]{babel} % Pour la langue française
\usepackage[utf8]{inputenc}
\usepackage[T1]{fontenc}
\usepackage{lmodern}
\usepackage{amsmath, amssymb,amsthm}
\usepackage{babel}
\usepackage{enumitem} % Pour gérer la numérotation personnalisée des listes
\usepackage[all,cmtip]{xy}  % Chargement du paquet xy pour les diagrammes commutatifs
\usepackage{tikz}
\usetikzlibrary{cd} % Pour les diagrammes commutatifs






\newtheorem{proposition}{Proposition}
\newtheorem{mynotation}{Notation}
\newtheorem{propriety}{Propriété}
\newtheorem{remark}{Remarque}

% Définir l'environnement "Théorème" en français
\newtheorem{theoreme}{Théorème}






% Choix du thème
\usetheme{Madrid}

% Titre et informations
\title{Représentation linéaire, caractère et \\ 
	représentations linéaires irréductibles \\ 
	d’un groupe infini.}

\author[SOUNKOUA Roger]{DEPARTEMENT DE MATHÉMATIQUES - INFORMATIQUE \\
	 Mémoire présenté par : SOUNKOUA Roger \\
Sous la direction du \\
\textbf{Dr. Gilbert MANTIKA}\\
(Chargée de Cours, UMa, FS)\\
Pr. Dieukam \\
(Maître de Conférences, UMa, FS)
}
\date{\today}

% Personnalisation du pied de page sur 2 lignes
\setbeamertemplate{footline}{
	\begin{beamercolorbox}[wd=\paperwidth,ht=2.5ex,dp=1ex,left]{footline}
		\hspace{0.5cm} \insertshorttitle  \hfill \insertshortauthor \hspace{0.5cm}
	\end{beamercolorbox}
	\begin{beamercolorbox}[wd=\paperwidth,ht=2.5ex,dp=1ex,left]{footline}
		\hspace{0.5cm} \insertframenumber{} / \inserttotalframenumber 
		\hfill \insertshortinstitute \hspace{0.5cm} \insertdate
	\end{beamercolorbox}
}

% Personnalisation de la couleur
\definecolor{myblue}{rgb}{0.1, 0.2, 0.8}
\setbeamercolor{footline}{bg=myblue, fg=white}


\begin{document}
	
	
	
	
	% Première diapo : titre
	\begin{frame}
		\titlepage
	\end{frame}
	

	
	\section*{Introduction}
	\begin{frame}{Introduction}
Introduction.
	\end{frame}
	

	% Sommaire
\begin{frame}{Sommaire}
	\tableofcontents
\end{frame}
	
	
	\section{Préliminaires}
	
		\begin{frame}{Préliminaires}
		  \frametitle{Définition d'un Groupe}
		
		\begin{definition} 
Un groupe est un couple  $(G, *)$ où $G$ est un ensemble non vide et  
			\[
			* : G \times G \longrightarrow G
			\]
			\[
			(x, y) \longmapsto x * y
			\]
est une loi telle que :
			\begin{enumerate} [label=\roman*)]
				\item $*$ est associative, c'est-à-dire, $\forall x, y, z \in G, \ (x * y) * z = x * (y * z)$ ;
				\item $G$ possède un élément neutre pour la loi $*$, c'est-à-dire, $\exists e \in G$ tel que $\forall x \in G, \ x * e = e * x = x$ ;
				\item Tout élément de $G$ est inversible (ou possède un élément symétrique) dans $G$, c'est-à-dire, $\forall x \in G, \ \exists y \in G$ tel que $x * y = y * x = e$.
			\end{enumerate}
		\end{definition}
		
		\begin{definition} 
			Si \((G, \ast)\) est un groupe tel que la loi \(\ast\) satisfasse à la propriété
			\[
			\forall x, y \in G, \ x \ast y = y \ast x,
			\]
			le groupe \((G, \ast)\) est dit \textbf{commutatif} ou encore \textbf{abélien}.
		\end{definition}
		
			
	\end{frame}
	
	
	
	\begin{frame}
\begin{remark}
Soit \( (G_i, \circ_i) \) une famille de groupes finis indexée par \( i = 1, \dots, n \).  
	Le cardinal du produit direct de ces groupes vérifie la relation suivante :
	\[
	\left| \prod_{i=1}^{n} G_i \right| = \prod_{i=1}^{n} |G_i|,
	\]
	où \( |G_i| \) désigne le cardinal du groupe \( G_i \).
\end{remark}
\end{frame}
	

	
	
	
	
	\begin{frame}{Préliminaires}
	\begin{definition}
		Une catégorie \( \mathcal{C} \) consiste en les données suivantes :
		\begin{enumerate}[label=\roman*)]
			\item Une classe \( |\mathcal{C}| \), dont les éléments sont appelés objets de \( \mathcal{C} \);
			\item À chaque couple d'objets \( (X, Y) \) de \( \mathcal{C} \), est associé un ensemble \( \mathcal{C}(X, Y) \) (ou \( \mathrm{Hom}_{\mathcal{C}}(X, Y) \)), dont les éléments sont appelés morphismes (ou flèches) de \( X \) dans \( Y \);
			\item À chaque triplet \( (X, Y, Z) \) d'objets de \( \mathcal{C} \), une application (appelée application de composition)
			\[
			\mathcal{C}(X, Y) \times \mathcal{C}(Y, Z) \to \mathcal{C}(X, Z), \quad (f, g) \mapsto g \circ f;
			\]
			\item À chaque objet \( X \in \mathcal{C} \), est associé un élément \( 1_X \in \mathcal{C}(X, X) \) appelé morphisme d'identité de \( X \).
		\end{enumerate}
		Ces données vérifient les axiomes suivants :
		\begin{itemize}
			\item \textbf{Associativité de la composition :}
			si \( X \xrightarrow{f} Y \xrightarrow{g} Z \xrightarrow{h} W \) sont des morphismes dans \( \mathcal{C} \), alors on a
			\[
			h \circ (g \circ f) = (h \circ g) \circ f.
			\]
			\item \textbf{Neutralité de l'identité :}
			pour tous \( X, Y \in |\mathcal{C}| \), et pour tout \( f \in \mathcal{C}(X, Y) \), on a
			\(f \circ 1_{X} = f \quad \text{et} \quad 1_{Y} \circ f = f.\)
		\end{itemize}
	\end{definition}
	\end{frame}
	

\begin{frame}
	\frametitle{Définition d'un foncteur contravariant}
	
	\begin{definition}
Un \textit{foncteur contravariant} est une loi de passage d'une catégorie \( \mathcal{C} \) à une catégorie \( \mathcal{D} \), \( F : \mathcal{C} \to \mathcal{D} \), qui satisfait les propriétés suivantes :
		\begin{enumerate}[label=\roman*)]
			\item À tout objet \( C \) de \( \mathcal{C} \) associe un objet \( F(C) \) de \( \mathcal{D} \),
			\item À tout morphisme \( X \xrightarrow{f} Y \) de \( \mathcal{C} \) associe un morphisme \( F(Y) \xrightarrow{F(f)} F(X) \) de \( \mathcal{D} \), et les conditions suivantes doivent être vérifiées :
			\begin{itemize}
				\item \( F(1_X) = 1_{F(X)} \) pour tout objet \( X \),
				\item \( F(g \circ f) = F(f) \circ F(g) \) pour tous morphismes \( X \xrightarrow{f} Y \xrightarrow{g} Z \).
			\end{itemize}
		\end{enumerate}
	\end{definition}
	
	
\end{frame}


\begin{frame}

	\begin{definition}
Soit \((X_i, \phi_{ij})\) un système inductif de groupes. La limite inductive ou limite directe, lorsqu'elle existe, est une famille compatible \((X, \phi_i : X_i \to X)\) avec \((X_i, \phi_{ij})\) vérifiant la propriété universelle (PU) suivante : 
		\begin{quote}
Pour toute autre famille \((X, \psi_i)_{i \in I}\) compatible avec \((X_i, \phi_{ij})\), il existe un unique homomorphisme de groupes \(u : X \to Y\) tel que le diagramme suivant soit commutatif pour tous \(i \leq j\) :
		\end{quote}
		\begin{tikzpicture}[auto, node distance=2cm]
			% Nodes
			\node (X_i) at (0, 2) {$X_i$};
			\node (X_j) at (4, 2) {$X_j$};
			\node (X) at (2, 0) {$X$};
			\node (Y) at (2, -2) {$Y$};
			
			% Arrows
			\draw[->] (X_i) -- node[above] {$\phi_{ij}$} (X_j);
			\draw[->] (X_i) -- node[right] {$\phi_i$} (X);
			\draw[->] (X_j) -- node[left] {$\phi_j$} (X);
			\draw[->] (X_i) -- node[below left] {$\psi_i$} (Y);
			\draw[->] (X_j) -- node[below right] {$\psi_j$} (Y);
			\draw[->, dashed] (X) -- node[right] {$u$} (Y);
		\end{tikzpicture}
	\end{definition}
		
\end{frame}


\begin{frame}
\begin{definition}
Soit $(X_{i}, \phi_{ij})_{i,j \in I}$ un système projectif de groupes. La limite projective ou limite inverse du système projectif \((X_{i}, \phi_{ij})_{i,j \in I}\) est une famille \((X, (\phi_{i})_{i \in I})\) de homomorphismes compatibles avec \((X_{i}, \phi_{ij})_{i,j \in I}\), vérifiant la propriété universelle suivante :
Si \((\psi _{i} : Y \rightarrow X_{i})_{i \in I}\) (\(Y \in |\mathcal{C}|\)) est une famille de morphismes compatibles, alors il existe un unique morphisme \(\psi : Y \rightarrow X\) tel que le diagramme suivant commute pour tous \(i \leq j\) :
	
	\begin{tikzpicture}[auto]
		% Nodes
		\node (Y) at (0, 2) {$Y$};
		\node (X) at (0, 0) {$X$};
		\node (X_i) at (2, -2) {$X_i$};
		\node (X_j) at (-2, -2) {$X_j$};
		
		% Arrows
		\draw[->, dashed] (Y) -- node[right] {$\psi$} (X);  % Dashed arrow
		\draw[->] (Y) -- node[left] {$\psi_j$} (X_j);
		\draw[->] (Y) -- node[right] {$\psi_i$} (X_i);
		\draw[->] (X) -- node[right] {$\phi_j$} (X_j);
		\draw[->] (X) -- node[left] {$\phi_i$} (X_i);
		\draw[->] (X_j) -- node[below] {$\phi_{ij}$} (X_i);
	\end{tikzpicture}
	
	\[
	\psi_{i} = \phi_{i} \circ \psi
	\]
	\[
	\psi_{j} = \phi_{j} \circ \psi
	\]
\end{definition}
\end{frame}

\begin{frame}
 \begin{theoreme}
Soit \( F: \mathcal{C} \to \mathcal{D} \) un foncteur contravariant entre deux catégories \( \mathcal{C} \) et \( \mathcal{D} \). Alors, l'image d'une limite inductive par le foncteur $F$ est une limite projective.
 \end{theoreme}

	1. Départ : système inductif \((X_i, f_{ij})\) dans \( \mathcal{C} \)
\begin{itemize}
	\item \( f_{ii} = \text{id}_{X_i} \)
	\item \( f_{jk} \circ f_{ij} = f_{ik} \)
	\item \( X = \varinjlim X_i \), avec \( u_i: X_i \to X \)
\end{itemize}
Par définition de la limite inductive \( X = \varinjlim X_i \), on a :
\begin{itemize}
	\item Pour toute famille compatible \( v_i: X_i \to Y \) dans \( \mathcal{C} \), c’est-à-dire \( v_j \circ f_{ij} = v_i \),
	\item Il existe un unique morphisme \( v: X \to Y \) tel que \( v \circ u_i = v_i \)
\end{itemize}

2. Application du foncteur contravariant \( F \)
\begin{itemize}
	\item \( f_{ij}: X_i \to X_j \Rightarrow F(f_{ij}): F(X_j) \to F(X_i) \)
	\item On obtient un système projectif dans \( \mathcal{D} \) et 
\end{itemize}

3. Famille compatible \((F(X), F(u_i))\)
\begin{itemize}
	\item \( u_j \circ f_{ij} = u_i \Rightarrow F(f_{ij}) \circ F(u_j) = F(u_i) \)
\end{itemize}

\end{frame}


\begin{frame}
	4. Propriété universelle
\begin{itemize}
	\item Par la propriété universelle dans \( \mathcal{C} \), il existe \( v: X \to Y \) tel que \( v \circ u_i = v_i \)
	\item En appliquant \( F \), on obtient un morphisme \( F(v): F(Y) \to F(X) \) tel que :
	\[
	F(u_i) \circ F(v) = F(v_i)
	\]
\end{itemize}

	Conclusion : \( F(X) \) satisfait la définition d'une limite projective, donc :
\[
F\left( \varinjlim X_i \right) = \varprojlim F(X_i)
\]

\end{frame}



	\section{Représentations linéaires d'un produit de deux groupes}
	
	\begin{frame}
		\frametitle{Définitions et exemples}
		
		\begin{definition} 
Soit \(\mathbb{K}\) un corps. Une représentation \(\mathbb{K}\)-linéaire d'un groupe fini \(G\) est un homomorphisme de groupes
			\[
			\rho : G \rightarrow \mathrm{GL}(V)
			\]
où \(V\) est un \(\mathbb{K}\)-espace vectoriel et \(\mathrm{GL}(V)\) est le groupe des applications linéaires bijectives de \(V\) sur lui-même.
		\end{definition}

		
		\begin{block}{Remarque} 
Si \(V\) est un \(\mathbb{K}\)-espace vectoriel de dimension finie \(n\), on dit que \(n\) est le degré de la représentation. De plus, en choisissant une base de \(V\), le groupe \(\mathrm{GL}(V)\) est isomorphe au groupe
			\[
			\mathrm{GL}(n, \mathbb{K}) = \left\{ A \in \mathrm{M}(n, \mathbb{K}) \mid \det(A) \neq 0 \right\},
			\]
où \(\mathrm{GL}(n, \mathbb{K})\) est le groupe des matrices inversibles de taille \(n \times n\) équipées de la multiplication des matrices à coefficients dans \(\mathbb{K}\), et \(\det(A)\) désigne le déterminant de la matrice \(A\).
		\end{block}
	\end{frame}
	
	

\begin{frame}
	\frametitle{Représentations linéaires et sous-représentations}
	

	\begin{definition}
	Soit \( \rho : G \to \mathrm{GL}(V) \) une représentation linéaire sur \( G \). Le caractère de \( V \), noté \( \chi_V \), est la fonction
	\[
	\chi_V : G \to \mathbb{C}
	\]
	définie pour tout \( g \in G \) par
	\[
	\chi_V(g) := \operatorname{Tr}(\rho(g)),
	\]
	où \( \operatorname{Tr} \) désigne la trace.
\end{definition}

\begin{theoreme}  
	Soit \( G \) un groupe. Les propriétés suivantes sont équivalentes :
	\begin{enumerate}
		\item[(i)] \( G \) est abélien.
		\item[(ii)] Toutes les représentations irréductibles de \( G \) sont de degré 1.
	\end{enumerate}
\end{theoreme}
	
\end{frame}


\section{Représentations linéaires d'un produit arbitraire des groupes}

\begin{frame}
\begin{proposition}
Soit $(V_i)_{i\in I}$ une famille non vide d’espaces vectoriels sur un même corps commutatif  $\mathbb{K}$. Soit $(u_i)_{i\in I}$ une famille non vide de vecteurs non nuls tels que, pour tout $i\in I$, le vecteur $u_i$ appartienne à $V_i$. Pour tous sous-ensembles finis $J$ et $K$ de $I$ tels que $J\subseteq K$, considérons l'application linéaire injective  
	$$\begin{array}{rlll}
		\varphi_{J,K}: \underset{i\in J}\otimes V_i& \longrightarrow& \underset{i\in K}\otimes V_i\\
		\underset{i\in J}\otimes v_i&\longmapsto& \underset{i\in J}\otimes v_i\otimes(\underset{i\in K\setminus J}\otimes u_i).
	\end{array}
	$$
	Soit $\mathcal{F}(I)$ l'ensemble de tous les sous-ensembles finis de $I$. 
	\begin{enumerate}[label=\roman*)]
		\item Le système $(\underset{i\in J}\otimes V_i, \varphi_{J,K})_{J\in \mathcal{F}(I)}$ est inductif.
		\item La limite inductive du système inductif $(\underset{i\in J}\otimes V_i, \varphi_{J,K})_{J\in \mathcal{F}(I)}$, notée $\underset{i\in I}\otimes V_i$, est le produit tensoriel infini des espaces vectoriels $(V_i)_{i\in I}$.
	\end{enumerate}
\end{proposition}
\end{frame}


\begin{frame}
\begin{proposition}
Soit $(V_i)_{i\in I}$ une famille non vide d’espaces vectoriels sur un même corps commutatif  $\mathbb{K}$. Soit $(u_i)_{i\in I}$ une famille non vide de vecteurs non nuls tels que, pour tout $i\in I$, le vecteur $u_i$ appartienne à $V_i$. Pour tous sous-ensembles finis $J$ et $K$ de $I$ tels que $J\subseteq K$, l’application  
	$$\begin{array}{rlll}
		\psi_{J,K}: GL(\underset{i\in K}\otimes V_i)& \longrightarrow& GL(\underset{i\in J}\otimes V_i)\\
		f&\longmapsto& \psi_{J,K} (f)= f_{J,K}
	\end{array}
	$$ 
	avec $f_{J,K}$ défini comme suit : pour tout $f\in GL(\underset{i\in K}\otimes V_i)$ et tout $\underset{i\in J}\otimes v_i\in \underset{i\in J}\otimes V_i$,  
	$$f_{J,K}(\underset{i\in J}\otimes v_i)= \underset{i\in J}\otimes v'_i \text{ si } f((\underset{i\in J}\otimes v_i)\otimes (\underset{i\in K\setminus J}\otimes t_i))= (\underset{i\in J}\otimes v'_i)\otimes (\underset{i\in K\setminus J}\otimes v'_i)= \underset{i\in K}\otimes v'_i,$$  
	pour un certain $\underset{i\in K\setminus J}\otimes t_i\in \underset{i\in K\setminus J}\otimes V_i$, est un homomorphisme de groupes.
\end{proposition}
\end{frame}

\begin{frame}
\begin{proposition}
Le système $\big(GL(\underset{i \in J}{\otimes} V_i), \psi_{J,K} \big)$ est projectif, avec pour limite projective
	\[
	\varprojlim_{J \in \mathcal{F}(I)} GL\left(\underset{i \in J}{\otimes} V_i\right),
	\]
	où $\mathcal{F}(I)$ désigne l’ensemble des sous-ensembles finis de $I$.
\end{proposition}

\end{frame}


\begin{frame}{Esquisse de la preuve}
	Pour chaque \( J \in \mathcal{F}(I) \), on considère le groupe :
	\[
	GL\left( \bigotimes_{i \in J} V_i \right)
	\]
	

	Pour tout \( J \subseteq K \), on définit un morphisme :
	\[
	\psi_{J,K} :
	GL\left( \bigotimes_{i \in K} V_i \right)
	\longrightarrow
	GL\left( \bigotimes_{i \in J} V_i \right)
	\]
On vérifie deux propriétés fondamentales :
	\begin{itemize}
		\item Identité : \( \psi_{J,J} = \mathrm{id} \)
		\item Compatibilité : \( \psi_{J,K} \circ \psi_{K,L} = \psi_{J,L} \) pour \( J \subseteq K \subseteq L \)
	\end{itemize}
	
	On obtient un système projectif :
	\[
	\left( GL\left( \bigotimes_{i \in J} V_i \right),\ \psi_{J,K} \right)
	\]	
Comme \( \mathbf{Grp} \) est complète, la limite projective existe :
\[
\varprojlim_{J \in \mathcal{F}(I)} GL\left( \bigotimes_{i \in J} V_i \right)
\]
\end{frame}

\begin{frame}
\begin{proposition}
Soit la correspondance \( GL : \text{Vect}_{\otimes V_i} \rightarrow \text{Grp} \) entre les catégories 
	\( \text{Vect}_{\otimes V_i} \) et \( \text{Grp} \) où \( \text{Grp} \) désigne la catégorie des groupes. \( GL \) est un foncteur contravariant.
	
\end{proposition}

	\begin{proposition}
Soit $J$ un sous-ensemble fini d'un ensemble $I$ et $(G_i)_{i\in J}$ une famille finie de groupes finis. Soit $(V_i)_{i\in J}$ une famille finie d'espaces vectoriels de dimension finie sur le même corps $F$ et pour chaque $i \in J$, soit $\varphi_i: G_i\rightarrow GL(V_i)$ une représentation linéaire de $G_i$ dans $V_i$. Alors, le produit tensoriel $\varphi_{J}= \underset{i\in J}\otimes \varphi_i$ des applications $(\varphi_i)_{i\in J}$ défini par : 
	$$\begin{array}{llll}
		\varphi_{J}= \underset{i\in J}\otimes \varphi_i: \underset{i\in J}\Pi G_i&\longrightarrow& GL(\underset{i\in J}\otimes V_i)\\ 
		(g_i)_{i\in J}&\longmapsto& \underset{i\in J}\otimes \varphi_i ((g_i)_{i\in J})= \underset{i\in J}\otimes \varphi_i(g_i)                                                                                                                                                                                                         
	\end{array}
	$$ 
	avec
	$$\begin{array}{rlll} 
		\underset{i\in J}\otimes \varphi_i(g_i): \underset{i\in J}\otimes V_i&\longrightarrow& \underset{i\in J}\otimes V_i\\
		\underset{i\in J}\otimes v_i&\longmapsto&(\underset{i\in J}\otimes \varphi_i(g_i))(\underset{i\in J}\otimes v_i)=\underset{i\in J}\otimes \varphi_i(g_i)(v_i)
	\end{array} 
	,$$ est une représentation linéaire du produit direct fini $\underset{i\in J}\prod G_i$ des groupes $(G_i)_{i\in J}$ dans le produit tensoriel fini $\underset{i\in J}\otimes V_i$.
\end{proposition}

\end{frame}


\begin{frame}
	\begin{proposition}
Soient $(I, \leq)$ un ensemble dirigé et $(G_i)_{i\in I}$ une famille non vide de groupes finis. Pour tout sous-ensemble fini $J$ de $I$, définissons $G_J = \underset{i\in J}\prod G_i$. Si $J$ et $K$ sont des sous-ensembles de $I$ tels que $J\subseteq K$, alors nous considérons la projection :
	$$\begin{array}{rlll}\varphi_{J, K}: G_K&\longrightarrow &G_J\\
		(g_i)_{i\in K}&\longmapsto&(g_i)_{i\in J}.
	\end{array}
$$ Le système $(G_J,\varphi_{J, K})$ est projectif avec limite projective $\underset{i\in I} \prod G_i$.
	\end{proposition}
	
	\begin{lemma}
Soit $(V_i)_{i\in I}$ une famille non vide d'espaces vecteurs de dimension finie sur le même corps \( \mathbb{K} \) et soit \( \underset{i\in I}\otimes V_i \) le produit tensoriel infini des espaces vectoriels \( (V_i)_{i\in I} \) où chaque \( V_i \) (\( i \in I \)) est un espace vectoriel non nul. Soit \( \mathcal{F}(I) \) l'ensemble des sous-ensembles finis de \( I \). Alors, 
		
		\[
		\underset{\overleftarrow{J \in \mathcal{F}(I)}}{\lim} GL\left( \underset{i \in J}{\otimes} V_i \right) = GL\left( \underset{\longrightarrow}{\lim}_{J \in \mathcal{F}(I)} \left( \underset{i \in J}{\otimes} V_i \right) \right)
		\]
	\end{lemma}
	
\end{frame}


\begin{frame}
\begin{theoreme} 
Soit \( (G_i)_{i \in I} \) une famille non vide de groupes et \( (\varphi_i)_{i \in I} \) une famille de représentations linéaires, où chaque  
	\[
	\varphi_i : G_i \to \operatorname{GL}(V_i)
	\]
	est une représentation du groupe \( G_i \) sur un espace vectoriel \( V_i \).  
	Alors, la représentation linéaire sur le produit tensoriel des espaces \( V_i \) est donnée par :
	\[
	\varphi_I = \bigotimes_{i \in I} \varphi_i : \prod_{i \in I} G_i \longrightarrow \operatorname{GL} \left( \bigotimes_{i \in I} V_i \right).
	\]
\end{theoreme}
\end{frame}

\begin{frame}
	\textbf{\underline{Preuve }  :}\\ 
Soient \( J , K \in \mathcal{F}(I) \) avec $\mathcal{F}(I)$ l'ensemble de partie finie de \( I \) tels que \( J \subseteq K \). Le diagramme suivant commute :

$$
\xymatrix{
	\underset{i\in K} \prod G_i \ar@{->}[rr]^{\varphi_{K}} \ar@{->}[dd]_{\varphi_{_{J,K}}} && GL(\underset{i\in K} \otimes V_i) \ar@{->}[dd]^{\psi_{_{J,K}}} \\
	&& \\
	\underset{i\in J} \prod G_i \ar@{->}[rr]^{\varphi_{J}} && GL(\underset{i\in J} \otimes V_i)
}
$$

En effet, soient \( J \) et \( K \) deux sous-ensembles finis de \( I \) tels que \( J \subseteq K \). Soit \( (g_i)_{i \in K} \) un élément de \( \underset{i\in K} \prod G_i \). Il est clair que :

$$
\psi_{J,K} \circ \varphi_{K} ((g_i)_{i\in K}) = \psi_{J, K}(\varphi_{K} ((g_i)_{i\in J})) = \psi_{J,K}(\underset{i\in J} \otimes \varphi_{i}(g_i))
$$
et

$$
\varphi_{J} \circ \varphi_{J,K} ((g_i)_{i\in K}) = \varphi_{J}(\varphi_{J,K}((g_i)_{i\in K})) = \varphi_{J} ((g_i)_{i\in J}) = \underset{i\in J} \otimes \varphi_{i}(g_i).
$$

Soit \( \underset{i\in J} \otimes v_i \) un élément de \( \underset{i\in J} \otimes V_i \). Alors \( (\underset{i\in J} \otimes v_i) \otimes (\underset{i\in K \setminus J} \otimes u_i) \in \underset{i\in K} \otimes V_i \). 
\end{frame}

\begin{frame}
	Ainsi,
	$$
	\underset{i\in K} \otimes \varphi_{i}(g_i) \left( (\underset{i\in J} \otimes v_i) \otimes (\underset{i\in K \setminus J} \otimes u_i) \right) = (\underset{i\in J} \otimes \varphi_{i}(g_i)(v_i)) \otimes (\underset{i\in K \setminus J} \otimes \varphi_{i}(g_i)(u_i)).
	$$
	Cela découle de la définition de \( \psi_{J,K} \) que :
	$$
	\psi_{J,K}(\underset{i\in K} \otimes \varphi_{i}(g_i))(\underset{i\in J} \otimes v_{i}) = \underset{i\in J} \otimes \varphi_{i}(g_i)(v_i) = \underset{i\in J} \otimes \varphi_{i}(g_i)(\underset{i\in J} \otimes v_i).
	$$
	Par conséquent, le diagramme ci-dessus commute. Ainsi il existe un unique homomorphisme de groupes :
	$$
	\varphi_I = 	\underset{\overleftarrow{J \in \mathcal{F}(I)}}{\lim} ( \underset{i \in J}{\bigotimes} \varphi_i) : \underset{i\in I} \prod G_i \longrightarrow GL(\underset{i\in I} \otimes V_i),
	$$
	qui est une représentation linéaire du produit direct infini \( \underset{i\in I} \prod G_i \) des groupes \( G_i \) dans \( \underset{i\in I} \otimes V_i \). Le théorème est donc démontré.
\end{frame}

\begin{frame}
 	\begin{definition} 
 	Soient \( G_1 \) et \( G_2 \) des groupes, et \( \rho_1 \) et \( \rho_2 \) des représentations linéaires respectives de \( G_1 \) et \( G_2 \), avec \( \chi_1 \) et \( \chi_2 \) les caractères associés à ces représentations. Le caractère \( \chi \) du produit tensoriel des représentations \( \rho_1 \otimes \rho_2 \) de \( G_1 \times G_2 \) est défini par la formule :
 	
 	\[
 	\chi(g_1, g_2) = \chi_1(g_1) \chi_2(g_2),
 	\]
 	pour tout \( g_1 \in G_1 \) et \( g_2 \in G_2 \).
 \end{definition}
 
 \begin{proposition}
 	Soit $J$ un ensemble fini et  \( \varphi_J \) la représentation définie par :
 	\[
 	\varphi_{J} = \underset{i \in J}{\bigotimes} \varphi_i : \prod_{i \in J} G_i \longrightarrow GL\left( \underset{i \in J}{\bigotimes} V_i \right)
 	\]
 	où \(\forall i \in J\), \( \varphi_i : G_i \to GL(V_i) \) est une représentation de chaque groupe \( G_i \) et \( V_i \) est l'espace vectoriel associé.\\
 \end{proposition}	
 
\end{frame}

\begin{frame}

Le caractère $\chi_{\varphi_J}$ de la représentation $\varphi_J$ du produit direct fini $\underset{i\in J}\prod G_i$ est donné par :
\[
\chi_{\varphi_J}((g_i)_{i\in J}) = \prod_{i\in J} \chi_{\varphi_i}(g_i), \quad \forall (g_i)_{i\in J} \in \underset{i\in J}\prod G_i,
\]
où  $\chi_{\varphi_i}$ est le caractère de la représentation $\varphi_i$ du groupe $G_i$.

 \begin{proposition}
	Soit \( \varphi_I \) la représentation définie par :
	\[
	\varphi_I = \underset{i \in I}{\bigotimes} \varphi_i : \prod_{i \in I} G_i \longrightarrow GL\left( \underset{i \in I}{\bigotimes} V_i \right),
	\]
	où \( \varphi_i : G_i \to GL(V_i) \) est une représentation linéaire de chaque groupe \( G_i \), et \( V_i \) est l'espace vectoriel complexe associé.\\
	À chaque représentation linéaire \( \varphi_i \), associons le caractère (une fonction de i)
	\[
	\chi_{\varphi_i} : G_i \to \mathbb{C}
	\]
	défini pour tout \( g_i \in G_i \) par :
	\[
	\chi_{\varphi_i}(g_i) := \operatorname{Tr}(\varphi_i(g_i)).
	\]
\end{proposition}
\end{frame}

\begin{frame}
	Définissons la suite \( (a_i)_{i \in \mathbb{N} \setminus \{0\} } \), dont chaque terme est donné par :
\[
a_i := \chi_{\varphi_i}(g_i), \quad i \in \mathbb{N} \setminus \{0\}.
\]
Considérons la suite des produits partiels associée à \( (a_i)_{i \in \mathbb{N} \setminus \{0\}} \), définie par :
\[
P_N = \prod_{i=1}^{N} \chi_{\varphi_i}(g_i), \quad N \in \mathbb{N} \setminus \{0\}.
\]
Si la suite \( (P_N)_{N \in \mathbb{N} \setminus \{0\}} \) converge, alors le caractère \( \chi_{\varphi_I} \) de la représentation tensorielle infinie \( \varphi_I \) est défini par :
\[
\chi_{\varphi_I}((g_i)_{i \in I}) = \lim_{N \to \infty} \prod_{i=1}^{N} \chi_{\varphi_i}(g_i) = \prod_{i \in I} \chi_{\varphi_i}(g_i).
\]	
Dans le cas contraire, le caractère \( \chi_{\varphi_I} \) n’est pas défini.
\end{frame}


\begin{frame}
	\begin{theoreme} 
Soit \( (G_i)_{i \in I} \) une famille non vide de groupes finis d'ordre premier chacun , \((V_i)_{i \in I} \) une famille  non vide d'espaces vectoriels sur  \(\mathbb{K}\), et  
	\[ (
	\varphi_i : G_i \to \operatorname{GL}(V_i) )_{i \in I}
	\]
	une famille de représentations linéaires
	Soit \( G =  \prod_{i \in I} G_i \)
	\begin{enumerate} [label=\roman*)]
		\item Si chaque \(G_i\) est un groupe cyclique, alors toute représentation irréductible de \( G\) est de degré 1.
		\item Si chaque \(G_i\) est un groupe d'ordre premier, alors toute représentation irréductible de \( G\) est de degré 1.
		\item Si presque toutes les représentations \( \varphi_i \) sont triviales, alors la représentation induite sur le produit tensoriel des espaces \( V_i \)
		\[
		\varphi_I = \bigotimes_{i \in I} \varphi_i : G = \prod_{i \in I} G_i \longrightarrow \operatorname{GL} \left( \bigotimes_{i \in I} V_i \right),
		\]
		est irréductible si et seulement si chaque représentation \( \varphi_i \) non triviale est irréductible ainsi que leur produit tensoriel.
	\end{enumerate} 	
\end{theoreme}
\end{frame}


\begin{frame}{Définition de \texorpdfstring{$\widehat{\mathbb{Z}}$}{Z-chapeau}}
	\begin{definition}
Le complété profini \( \widehat{\mathbb{Z}} \) est défini comme la limite projective du système :
		\[
		\left(\mathbb{Z}/n\mathbb{Z},\; \varphi_{m,n}\right),
		\]
		où :
		\begin{enumerate}[label=\roman*)]
			\item \( \mathbb{Z}/n\mathbb{Z} \) est le groupe des classes d’équivalence modulo \( n \),
			\item \( \varphi_{m,n} : \mathbb{Z}/m\mathbb{Z} \to \mathbb{Z}/n\mathbb{Z} \) est la surjection canonique définie lorsque \( n \mid m \).
		\end{enumerate}
	\end{definition}
\end{frame}

\begin{frame}
	\begin{proposition}
	Soit \( N \) un sous-ensemble fini de \( \mathbb{N} \) et soit \( (\mathbb{Z}/n\mathbb{Z})_{n \in N} \) une famille finie de groupes cycliques. Soit \( (V_n)_{n \in N} \) une famille finie non vide d'espaces vectoriels de dimension finie, et pour chaque \( n \in N \), soit 
	\[
	\varphi_n: \mathbb{Z}/n\mathbb{Z} \to \operatorname{GL}(V_n)
	\]
	une représentation linéaire.\\  
	Alors, on a la représentation linéaire :
	\[
	\varphi_N = \bigotimes_{n\in N} \varphi_n: \prod_{n \in N} \mathbb{Z}/n\mathbb{Z} \to \operatorname{GL} \left( \bigotimes_{n \in N} V_n \right)
	\]
	tel que :
	\[
	\varphi_N((g_n)_{n \in N}) = \bigotimes_{n \in N} \varphi_n(g_n).
	\]
	De plus, en passant à la limite inverse, on a la représentation linéaire sur un produit tensoriel infini
	\[
	\varphi_\mathbb{N} = \bigotimes_{n\in \mathbb{N}} \varphi_n: 	\varprojlim_{N \in \mathcal{F}(\mathbb{N})} \prod \mathbb{Z}/n\mathbb{Z} = \prod_{n \in \mathbb{N}} \mathbb{Z}/n\mathbb{Z} \longrightarrow \operatorname{GL}\left( \bigotimes_{n \in \mathbb{N}} V_n \right).
	\]
\end{proposition}
\end{frame}

\begin{frame}
	\begin{theoreme}
	L'application
	\[
	\Phi : \widehat{\mathbb{Z}} \to \varprojlim_{N \in \mathcal{F}(\mathbb{N})} \prod_{n \in N} \mathbb{Z}/n\mathbb{Z}
	\]
	définie par :
	\[
	\Phi((x_n)_{n \geq 1}) = (x_N)_{N \in \mathcal{F}(\mathbb{N})}, \quad \text{où } x_N = (x_n)_{n \in N}
	\]
	est un isomorphisme de groupes .
\end{theoreme}


Ainsi, d’après les résultats précédents, nous obtenons la représentation linéaire suivante :
\[
\varphi_{\mathbb{N}} : \widehat{\mathbb{Z}} = \prod_{n \in \mathbb{N}} \mathbb{Z}/n\mathbb{Z} \longrightarrow \operatorname{GL} \left( \bigotimes_{n \in \mathbb{N}} V_n \right).
\]
\end{frame}

\begin{frame}
\begin{definition}
	Un caractère de $\widehat{\mathbb{Z}}$ est un homomorphisme de groupe continu :
	\[
	\chi : \widehat{\mathbb{Z}} \to \mathbb{C}^\times.
	\]
	
\end{definition}

\begin{theoreme}
	L'ensemble des caractères continus de $\widehat{\mathbb{Z}}$ est isomorphe à $\mathbb{Q}/\mathbb{Z}$ via la dualité de Pontryagin :
	\[
	\operatorname{Hom}_{\text{cont}}(\widehat{\mathbb{Z}}, \mathbb{C}^\times) \simeq \mathbb{Q}/\mathbb{Z}.
	\]
\end{theoreme}


\begin{proposition}
	\begin{enumerate} [label=\roman*)] 
		\item Toutes les représentations irréductibles de $\widehat{\mathbb{Z}}$ sont de degré $1$.
		\item Soient $\operatorname{Irr}(\widehat{\mathbb{Z}})$ l'ensemble des représentations irréductibles de $\widehat{\mathbb{Z}}$ et $\operatorname{Hom}_{\text{cont}}(\widehat{\mathbb{Z}}, \mathbb{C}^\times)$ l'ensemble de ses caractères continus. Il existe un isomorphisme :
		\[
		\Phi : \operatorname{Irr}(\widehat{\mathbb{Z}}) \to \operatorname{Hom}_{\text{cont}}(\widehat{\mathbb{Z}}, \mathbb{C}^\times)),
		\]
		\item $\widehat{\mathbb{Z}}$ possède une infinité non dénombrable de représentations irréductibles.
	\end{enumerate}
\end{proposition}

\end{frame}

	\section{Conclusion}
	\begin{frame}{Conclusion}
	
	Merci pour votre aimable attention, J'en ai fini.
	\end{frame}
	
	
\end{document}
