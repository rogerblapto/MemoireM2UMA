\documentclass[9pt]{beamer}
\usepackage[french]{babel} % Pour la langue française
\usepackage[utf8]{inputenc}
\usepackage[T1]{fontenc}
\usepackage{lmodern}
\usepackage{amsmath, amssymb,amsthm}
\usepackage{babel}
\usepackage{enumitem} % Pour gérer la numérotation personnalisée des listes
\usepackage[all,cmtip]{xy}  % Chargement du paquet xy pour les diagrammes commutatifs
\usepackage{tikz}
\usetikzlibrary{cd} % Pour les diagrammes commutatifs






\newtheorem{proposition}{Proposition}
\newtheorem{mynotation}{Notation}
\newtheorem{propriety}{Propriété}
\newtheorem{remark}{Remarque}

% Définir l'environnement "Théorème" en français
\newtheorem{theoreme}{Théorème}






% Choix du thème
\usetheme{Madrid}

% Titre et informations
\title{Représentation linéaire, caractère et \\ 
	représentations linéaires irréductibles \\ 
	d’un groupe infini.}

\author[SOUNKOUA Roger]{DEPARTEMENT DE MATHÉMATIQUES - INFORMATIQUE \\
	 Mémoire présenté par : SOUNKOUA Roger \\
Sous la direction du \\
\textbf{Dr. Gilbert MANTIKA}\\
(Chargée de Cours, UMa, FS)\\
Pr. Dieukam \\
(Maître de Conférences, UMa, FS)
}
\date{\today}

% Personnalisation du pied de page sur 2 lignes
\setbeamertemplate{footline}{
	\begin{beamercolorbox}[wd=\paperwidth,ht=2.5ex,dp=1ex,left]{footline}
		\hspace{0.5cm} \insertshorttitle  \hfill \insertshortauthor \hspace{0.5cm}
	\end{beamercolorbox}
	\begin{beamercolorbox}[wd=\paperwidth,ht=2.5ex,dp=1ex,left]{footline}
		\hspace{0.5cm} \insertframenumber{} / \inserttotalframenumber 
		\hfill \insertshortinstitute \hspace{0.5cm} \insertdate
	\end{beamercolorbox}
}

% Personnalisation de la couleur
\definecolor{myblue}{rgb}{0.1, 0.2, 0.8}
\setbeamercolor{footline}{bg=myblue, fg=white}


\begin{document}
	
	
	
	
	% Première diapo : titre
	\begin{frame}
		\titlepage
	\end{frame}
	

	
	\section*{Introduction}
	\begin{frame}{Introduction}
La théorie des représentations linéaires des groupes est un outil fondamental permettant de représenter les éléments d’un groupe abstrait par des matrices inversibles sur un corps donné. Cette approche, traduisant les problèmes d’algèbre abstraite complexes en des problèmes d’algèbre linéaire plus accessibles, repose sur la notion de représentation linéaire. Etant donné un corps $\mathbb{K}$, une représentation $\mathbb{K}$-linéaire d’un groupe fini $G$ est un homomorphisme de groupes \[ \rho : G \to \text{GL}(V) \]
où $V$ est un $\mathbb{K}$-espace vectoriel et $\text{GL}(V)$ désigne le groupe des applications linéaires bijectives de $V$ sur lui-même. 
La théorie des représentations linéaires des groupes finis a été développée pour la première fois par le mathématicien allemand Ferdinand Georg Frobenius en 1897. Il a introduit la notion de représentation linéaire des groupse finis et a jetté les bases sur la théorie des caractères de ces groupes. Ces travaux a ete approfondi par le mathématicien francais Jean-Pierre Serre où il a formalisé cette théorie dans son livre intitulé \textbf{"Représentations linéaires des groupes finis"} publié en 1968. Ces deux auteurs qu'on vient de voir ont travaille dans le cadre de groupes finis et il n y a pas de resultats recents accessibles generalisant cela au groupe infinis. D'où notre thème : \textbf{représentation linéaires, caractères et représentations linéaires irréductibles d'un groupe infini.}
Le but principal de ce projet est d’étendre les représentations linéaires aux groupes infinis. Pour atteindre les objectifs de ce travail, nous allons suivre cette démarche dans la presentation:
	\end{frame}
	

	% Sommaire
\begin{frame}{Sommaire}
	\tableofcontents
\end{frame}
	
	
	\section{Préliminaires}
	
		\begin{frame}{Préliminaires}
		  \frametitle{Définition d'un Groupe}
		
		\begin{definition} 
Un groupe est un couple  $(G, *)$ où $G$ est un ensemble non vide et  
			\[
			* : G \times G \longrightarrow G
			\]
			\[
			(x, y) \longmapsto x * y
			\]
une loi vérifiant :
			\begin{enumerate} [label=\roman*)]
				\item $*$ est associative, c'est-à-dire, $\forall x, y, z \in G, \ (x * y) * z = x * (y * z)$ ;
				\item $G$ possède un élément neutre pour la loi $*$, c'est-à-dire, $\exists e \in G$ tel que $\forall x \in G, \ x * e = e * x = x$ ;
				\item Tout élément de $G$ est inversible (ou possède un élément symétrique) dans $G$, c'est-à-dire, $\forall x \in G, \ \exists y \in G$ tel que $x * y = y * x = e$.
			\end{enumerate}
		\end{definition}
		
		\begin{definition} 
			Si \((G, \ast)\) est un groupe tel que la loi \(\ast\) satisfasse à la propriété
			\[
			\forall x, y \in G, \ x \ast y = y \ast x,
			\]
			le groupe \((G, \ast)\) est dit \textbf{commutatif} ou encore \textbf{abélien}.
		\end{definition}
		
			
	\end{frame}
	
	
	
	\begin{frame}
		
		\begin{definition} 
Soit \( (G_i, \circ_i) \) une famille de groupes finis indexée par l'ensemble \( \{1, 2, \dots, n\} \), où \( n \in \mathbb{N}^{*} \). Le \textit{produit direct fini} de cette famille est un groupe \( \left( \prod_{i=1}^{n} G_i, \circ \right) \), défini par les propriétés suivantes : 
			\begin{enumerate}[label=\roman*)]
				\item L'ensemble sous-jacent est constitué des familles indexées par \( \{1, 2, \dots, n\} \) :  
				\[
				\prod_{i=1}^{n} G_i = \left\{ (g_i)_{i=1}^{n} \mid g_i \in G_i \text{ pour tout } i \in \{1, 2, \dots, n\} \right\}.
				\]
				\item La loi de composition \( \circ \) est définie composante par composante :  
				\[
				(g_i)_{i=1}^{n} \circ (g_i')_{i=1}^{n} = (g_i \circ_i g_i')_{i=1}^{n},
				\]  
				où \( \circ_i \) désigne l'opération du groupe \( G_i \) pour chaque \( i \in \{1, 2, \dots, n\} \).
				\item L'élément neutre de \( \prod_{i=1}^{n} G_i \) est la famille \( (e_i)_{i=1}^{n} \), où \( e_i \) est l'élément neutre de \( G_i \) pour tout \( i \in \{1, 2, \dots, n\} \).
				\item L'inverse d'une famille \( (g_i)_{i=1}^{n} \in \prod_{i=1}^{n} G_i \) est donné par :  
				\[
				(g_i)_{i=1}^{n}{}^{-1} = (g_i^{-1})_{i=1}^{n},
				\]  
				où \( g_i^{-1} \) est l'inverse de \( g_i \) dans \( G_i \) pour chaque \( i \in \{1, 2, \dots, n\} \).
			\end{enumerate}
		\end{definition}
	\end{frame}
	
	
	\begin{frame}
		
	\begin{remark}
Soit \( (G_i, \circ_i) \) une famille de groupes finis indexée par \( i = 1, \dots, n \).  
Le cardinal du produit direct de ces groupes vérifie la relation suivante :
		\[
		\left| \prod_{i=1}^{n} G_i \right| = \prod_{i=1}^{n} |G_i|,
		\]
		où \( |G_i| \) désigne le cardinal du groupe \( G_i \).
	\end{remark}
	

\begin{propriety}
Si chaque \( (G_i, \circ_i) \) est un groupe abélien, alors \( \left( \prod_{i=1}^{n} G_i, \circ \right) \) est aussi un groupe abélien.
\end{propriety}

		
		\begin{definition}
Soient \( (G, \cdot) \) et \( (G', *) \) deux groupes. Un homomorphisme de groupes de \( G \) dans \( G' \) est une application \( f : G \rightarrow G' \) vérifiant :
			\[
			\forall (x, y) \in G \times G, \quad f(x \cdot y) = f(x) * f(y).
			\]
		\end{definition}
		
		
		
	\end{frame}
	
	
	
	
	\begin{frame}{Préliminaires}
	\begin{definition}
		Une catégorie \( \mathcal{C} \) consiste en les données suivantes :
		\begin{enumerate}[label=\roman*)]
			\item Une classe \( |\mathcal{C}| \), dont les éléments sont appelés objets de \( \mathcal{C} \);
			\item À chaque couple d'objets \( (X, Y) \) de \( \mathcal{C} \), est associé un ensemble \( \mathcal{C}(X, Y) \) (ou \( \mathrm{Hom}_{\mathcal{C}}(X, Y) \)), dont les éléments sont appelés morphismes (ou flèches) de \( X \) dans \( Y \);
			\item À chaque triplet \( (X, Y, Z) \) d'objets de \( \mathcal{C} \), une application (appelée application de composition)
			\[
			\mathcal{C}(X, Y) \times \mathcal{C}(Y, Z) \to \mathcal{C}(X, Z), \quad (f, g) \mapsto g \circ f;
			\]
			\item À chaque objet \( X \in \mathcal{C} \), est associé un élément \( 1_X \in \mathcal{C}(X, X) \) appelé morphisme d'identité de \( X \).
		\end{enumerate}
		Ces données vérifient les axiomes suivants :
		\begin{itemize}
			\item \textbf{Associativité de la composition :}
			si \( X \xrightarrow{f} Y \xrightarrow{g} Z \xrightarrow{h} W \) sont des morphismes dans \( \mathcal{C} \), alors on a
			\[
			h \circ (g \circ f) = (h \circ g) \circ f.
			\]
			\item \textbf{Neutralité de l'identité :}
			pour tous \( X, Y \in |\mathcal{C}| \), et pour tout \( f \in \mathcal{C}(X, Y) \), on a
			\(f \circ 1_{X} = f \quad \text{et} \quad 1_{Y} \circ f = f.\)
		\end{itemize}
	\end{definition}
	\end{frame}
	

\begin{frame}
	\frametitle{Définition d'un foncteur contravariant}
	
	\begin{definition}
Un \textit{foncteur contravariant} est une loi de passage d'une catégorie \( \mathcal{C} \) à une catégorie \( \mathcal{D} \), \( F : \mathcal{C} \to \mathcal{D} \), qui satisfait les propriétés suivantes :
		\begin{enumerate}[label=\roman*)]
			\item À tout objet \( C \) de \( \mathcal{C} \) associe un objet \( F(C) \) de \( \mathcal{D} \),
			\item À tout morphisme \( X \xrightarrow{f} Y \) de \( \mathcal{C} \) associe un morphisme \( F(Y) \xrightarrow{F(f)} F(X) \) de \( \mathcal{D} \), et les conditions suivantes doivent être vérifiées :
			\begin{itemize}
				\item \( F(1_X) = 1_{F(X)} \) pour tout objet \( X \),
				\item \( F(g \circ f) = F(f) \circ F(g) \) pour tous morphismes \( X \xrightarrow{f} Y \xrightarrow{g} Z \).
			\end{itemize}
		\end{enumerate}
	\end{definition}
	
\begin{definition} 
Un ensemble \((I, \leq)\) est dit ordonné filtrant si \((I, \leq)\) est un ensemble partiellement ordonné et si pour tous \(i, j \in I\), il existe \(k \in I\) vérifiant \(i \leq k\) et \(j \leq k\).
\end{definition}
	
\end{frame}


\begin{frame}
	\frametitle{Définitions}
	
	\begin{definition}
Soit \((I, \leq)\) un ensemble ordonné filtrant. Un système inductif de groupes sur \(I\) est la donnée d'un couple \((X_i, \phi_{ij})_{i,j \in I}\) où \(X_i\) sont les groupes et les \(\phi_{ij} : X_i \to X_j\) (pour \(i \leq j\)) sont des homomorphismes de groupes, vérifiant :
	\begin{enumerate}[label=\roman*)]
		\item Pour tout \(i \in I\), \(\phi_{ii} = \mathrm{Id}_{X_i}\) ;
		\item Pour tous \((i, j, k) \in I^3\), \(i \leq j \leq k \Rightarrow \phi_{jk} \circ \phi_{ij} = \phi_{ik}\).
	\end{enumerate}
Ce qui se traduit par le diagramme commutatif suivant :
	\[
	\xymatrix{
		X_i \ar[rr]^{\phi_{ij}} \ar[dr]_{\phi_{ik}} & & X_j \ar[dl]^{\phi_{jk}} \\
		& X_k &
	}
	\]
\end{definition}
	
\end{frame}

\begin{frame}
	

	
	\begin{definition}
Soit \(X\) un groupe et \((X_i, \phi_{ij})\) un système inductif de groupes. La famille \((X, \phi_i : X_i \rightarrow X)\) est dite compatible avec \((X_i, \phi_{ij})\) si pour tous \(i, j \in I\) tels que \(i \leq j\), on a \(\phi_i = \phi_j \circ \phi_{ij}\). Ce qui est illustré par le diagramme commutatif suivant :
		\[
		\xymatrix{
			X_i \ar[r]^{\phi_{ij}} \ar[dr]_{\phi_i} & X_j \ar[d]^{\phi_j} \\
			& X
		}
		\]
	\end{definition}
	
\end{frame}

\begin{frame}

	\begin{definition}
Soit \((X_i, \phi_{ij})\) un système inductif de groupes. La limite inductive ou limite directe, lorsqu'elle existe, est une famille compatible \((X, \phi_i : X_i \to X)\) avec \((X_i, \phi_{ij})\) vérifiant la propriété universelle (PU) suivante : 
		\begin{quote}
Pour toute autre famille \((X, \psi_i)_{i \in I}\) compatible avec \((X_i, \phi_{ij})\), il existe un unique homomorphisme de groupes \(u : X \to Y\) tel que le diagramme suivant soit commutatif pour tous \(i \leq j\) :
		\end{quote}
		\begin{tikzpicture}[auto, node distance=2cm]
			% Nodes
			\node (X_i) at (0, 2) {$X_i$};
			\node (X_j) at (4, 2) {$X_j$};
			\node (X) at (2, 0) {$X$};
			\node (Y) at (2, -2) {$Y$};
			
			% Arrows
			\draw[->] (X_i) -- node[above] {$\phi_{ij}$} (X_j);
			\draw[->] (X_i) -- node[right] {$\phi_i$} (X);
			\draw[->] (X_j) -- node[left] {$\phi_j$} (X);
			\draw[->] (X_i) -- node[below left] {$\psi_i$} (Y);
			\draw[->] (X_j) -- node[below right] {$\psi_j$} (Y);
			\draw[->, dashed] (X) -- node[right] {$u$} (Y);
		\end{tikzpicture}
	\end{definition}
	
	\begin{proposition}
La limite inductive, lorsqu'elle existe, est unique à isomorphisme unique près.
	\end{proposition}	
\end{frame}


\begin{frame}
	\begin{mynotation}
La limite inductive \((X, \phi_i)_{i \in I}\) d'un système inductif \((X_i, \phi_{ij})_{j \in I}\) est notée \(X = \varinjlim X_i\).
	\end{mynotation}
	
	Soit $(I, \leq)$ un ensemble ordonné filtrant.
	
	\begin{definition}
Un système projectif de groupes sur \((I, \leq)\) est un couple \((X_{i}, \phi_{ij})_{i,j \in I}\) où les \(X_{i}\) sont les groupes et les \(\phi_{ij}: X_{j} \rightarrow X_{i}\) (\(i \leq j\)) sont les homomorphismes de groupes vérifiant :
		\begin{enumerate}[label=\roman*)]
			\item \(\phi_{ii} = \text{Id}_{X_{i}}\) pour tout \(i \in I\);
			\item pour tout \((i,j,k) \in I^{3}\) tels que \(i \leq j \leq k\), on a \(\phi_{ik} = \phi_{ij} \circ \phi_{jk}\).
		\end{enumerate}
Autrement dit, le diagramme suivant est commutatif :
		\[
		\xymatrix{
			X_{k} \ar[rr]^{\phi_{jk}} \ar[dr]_{\phi_{ik}} & & X_{j} \ar[dl]^{\phi_{ij}} \\
			& X_{i} &
		}
		\]
	\end{definition}

\end{frame}

\begin{frame}
\begin{definition}
Soient \(X\) un groupe et \((X_{i}, \phi_{ij})_{i,j \in I}\) un système projectif de groupes. La famille de homomorphismes \((\phi_{i} : X \rightarrow X_{i})_{i \in I}\), qu'on note \((X , \phi_{i})\), est dite compatible avec le système projectif \((X_{i}, \phi_{ij})_{i,j \in I}\) si pour tous \(i,j \in I\) tels que \(i \leq j\), on a : \(\phi_{ij} \circ \phi_{j} = \phi_{i}\).
	Ce qui se traduit par le diagramme commutatif suivant :
	\[
	\xymatrix{
		X \ar[rr]^{\phi_{j}} \ar[dr]_{\phi_{i}} & & X_{j} \ar[dl]^{\phi_{ij}} \\
		& X_{i} &
	}
	\]
\end{definition}
\end{frame}

\begin{frame}
\begin{definition}
Soit $(X_{i}, \phi_{ij})_{i,j \in I}$ un système projectif de groupes. La limite projective ou limite inverse du système projectif \((X_{i}, \phi_{ij})_{i,j \in I}\) est une famille \((X, (\phi_{i})_{i \in I})\) de homomorphismes compatibles avec \((X_{i}, \phi_{ij})_{i,j \in I}\), vérifiant la propriété universelle suivante :
Si \((\psi _{i} : Y \rightarrow X_{i})_{i \in I}\) (\(Y \in |\mathcal{C}|\)) est une famille de morphismes compatibles, alors il existe un unique morphisme \(\psi : Y \rightarrow X\) tel que le diagramme suivant commute pour tous \(i \leq j\) :
	
	\begin{tikzpicture}[auto]
		% Nodes
		\node (Y) at (0, 2) {$Y$};
		\node (X) at (0, 0) {$X$};
		\node (X_i) at (2, -2) {$X_i$};
		\node (X_j) at (-2, -2) {$X_j$};
		
		% Arrows
		\draw[->, dashed] (Y) -- node[right] {$\psi$} (X);  % Dashed arrow
		\draw[->] (Y) -- node[left] {$\psi_j$} (X_j);
		\draw[->] (Y) -- node[right] {$\psi_i$} (X_i);
		\draw[->] (X) -- node[right] {$\phi_j$} (X_j);
		\draw[->] (X) -- node[left] {$\phi_i$} (X_i);
		\draw[->] (X_j) -- node[below] {$\phi_{ij}$} (X_i);
	\end{tikzpicture}
	
	\[
	\psi_{i} = \phi_{i} \circ \psi
	\]
	\[
	\psi_{j} = \phi_{j} \circ \psi
	\]
\end{definition}
\end{frame}

\begin{frame}
\begin{proposition}[Unicité de la limite projective]
Si une limite projective d'un système projectif existe, elle est unique à isomorphisme près.
\end{proposition}

\begin{mynotation}
Une telle limite est notée $\varprojlim_{I} X_i$ ou  $\varprojlim_{i \in I} X_i$.
\end{mynotation}

 \begin{definition}[Groupe profini]
La limite projective d'un système projectif de groupes finis est appelée groupe profini.
\end{definition}


 \frametitle{Proposition sur le Complété Profini}
\begin{proposition} 
Le complété profini d’un groupe est unique à isomorphisme près, c’est-à-dire si \( (\widehat{G_1}, j_1) \) et \( (\widehat{G_2}, j_2) \) sont deux complétés de \( G \), alors il existe un isomorphisme \( \widehat{\alpha} : \widehat{G_1} \to \widehat{G_2} \) tel que \( \widehat{\alpha} j_1 = j_2 \).
\end{proposition}

\end{frame}

\begin{frame}
 \begin{theorem}
Soit \( F: \mathcal{C} \to \mathcal{D} \) un foncteur contravariant entre deux catégories \( \mathcal{C} \) et \( \mathcal{D} \). Alors, l'image d'une limite inductive par le foncteur $F$ est une limite projective.
 \end{theorem}

	1. Départ : système inductif \((X_i, f_{ij})\) dans \( \mathcal{C} \)
\begin{itemize}
	\item \( f_{ii} = \text{id}_{X_i} \)
	\item \( f_{jk} \circ f_{ij} = f_{ik} \)
	\item \( X = \varinjlim X_i \), avec \( u_i: X_i \to X \)
\end{itemize}
Par définition de la limite inductive \( X = \varinjlim X_i \), on a :
\begin{itemize}
	\item Pour toute famille compatible \( v_i: X_i \to Y \) dans \( \mathcal{C} \), c’est-à-dire \( v_j \circ f_{ij} = v_i \),
	\item Il existe un unique morphisme \( v: X \to Y \) tel que \( v \circ u_i = v_i \)
\end{itemize}

2. Application du foncteur contravariant \( F \)
\begin{itemize}
	\item \( f_{ij}: X_i \to X_j \Rightarrow F(f_{ij}): F(X_j) \to F(X_i) \)
	\item On obtient un système projectif dans \( \mathcal{D} \) et 
\end{itemize}

3. Famille compatible \((F(X), F(u_i))\)
\begin{itemize}
	\item \( u_j \circ f_{ij} = u_i \Rightarrow F(f_{ij}) \circ F(u_j) = F(u_i) \)
\end{itemize}

\end{frame}


\begin{frame}
	4. Propriété universelle
\begin{itemize}
	\item Par la propriété universelle dans \( \mathcal{C} \), il existe \( v: X \to Y \) tel que \( v \circ u_i = v_i \)
	\item En appliquant \( F \), on obtient un morphisme \( F(v): F(Y) \to F(X) \) tel que :
	\[
	F(u_i) \circ F(v) = F(v_i)
	\]
\end{itemize}

	Conclusion : \( F(X) \) satisfait la définition d'une limite projective, donc :
\[
F\left( \varinjlim X_i \right) = \varprojlim F(X_i)
\]

\end{frame}



	\section{Représentations linéaires d'un produit de deux groupes.}
	\subsection{Définitions et exemples}
	
	\begin{frame}
		\frametitle{Définitions et exemples}
		
		\begin{definition} 
Soit \(\mathbb{K}\) un corps. Une représentation \(\mathbb{K}\)-linéaire d'un groupe fini \(G\) est un homomorphisme de groupes
			\[
			\rho : G \rightarrow \mathrm{GL}(V)
			\]
où \(V\) est un \(\mathbb{K}\)-espace vectoriel et \(\mathrm{GL}(V)\) est le groupe des applications linéaires bijectives de \(V\) sur lui-même.
		\end{definition}

		
		\begin{block}{Remarque} 
Si \(V\) est un \(\mathbb{K}\)-espace vectoriel de dimension finie \(n\), on dit que \(n\) est le degré de la représentation. De plus, en choisissant une base de \(V\), le groupe \(\mathrm{GL}(V)\) est isomorphe au groupe
			\[
			\mathrm{GL}(n, \mathbb{K}) = \left\{ A \in \mathrm{M}(n, \mathbb{K}) \mid \det(A) \neq 0 \right\},
			\]
où \(\mathrm{GL}(n, \mathbb{K})\) est le groupe des matrices inversibles de taille \(n \times n\) équipées de la multiplication des matrices à coefficients dans \(\mathbb{K}\), et \(\det(A)\) désigne le déterminant de la matrice \(A\).
		\end{block}
	\end{frame}
	
	

\begin{frame}
	\frametitle{Représentations linéaires et sous-représentations}
	
	\begin{definition} \cite{serre1971representation}
Soient \( (V, \rho)_{G} \) et \( (W, \psi)_{G} \) deux représentations linéaires. Un opérateur d'entrelacement, ou morphisme de représentations, est une application linéaire \( \alpha : V \to W \) telle que
		\[
		\alpha \circ \rho(g) = \psi(g) \circ \alpha, \quad \forall g \in G.
		\]
On dit que \( \alpha \) est équivariante. Lorsque \( \alpha : V \to W \) est un isomorphisme, on dit que les représentations \( (V, \varphi)_{G} \) et \( (W, \psi)_{G} \) sont isomorphes.
	\end{definition}
	

	
	\begin{definition} 
Soit \( W \) un sous-espace vectoriel de \( V \) et \( G \) un groupe. On dit que \( W \) est stable (ou invariant) sous l'action de \( G \), ou encore \( G \)-stable, si pour tout \( g \in G \) et tout \( w \in W \), on a \( \rho_g(w) \in W \).
	\end{definition}
	

	
	\begin{definition}
Une sous-représentation de \( (V,\rho)_{G} \) est la restriction \( \rho_W : G \rightarrow \mathrm{GL}(W) \) où \( W \) est stable sous \( G \). Elle est définie par
		\[
		\rho_W(g) = \rho(g)|_W, \quad \forall g \in G.
		\]
	\end{definition}
	
\end{frame}


\begin{frame}
	
\begin{proposition}
Soit \( \rho : G \rightarrow \mathrm{GL}(V) \) une représentation linéaire sur \( G \) telle que \( W \) soit un sous-espace vectoriel de \( V \) stable sous l'action de \( G \). Alors, l'application restreinte \( \rho_W : G \rightarrow \mathrm{GL}(W) \) définie par \( \rho_W(g) = \rho(g)|_W \) pour tout \( g \in G \) est un homomorphisme de groupes.
\end{proposition}


	
\begin{lemma}
Soient \( \varphi : G \to \mathrm{GL}(V_1) \) et \( \psi : G \to \mathrm{GL}(V_2) \) deux représentations linéaires de \( G \). Soit \( f : V_1 \to V_2 \) un morphisme de représentations linéaires. Alors :
		\begin{enumerate}[label=\roman*)]
			\item \( \rho_{\ker(f)} : G \to \mathrm{GL}(\ker(f)) \) est une sous-représentation linéaire de \( \varphi : G \to \mathrm{GL}(V_1) \) ;
			\item L'image \( \mathrm{im}(f) \), \( \rho_{\mathrm{im}(f)} : G \to \mathrm{GL}(\mathrm{im}(f)) \), est une sous-représentation linéaire de \( \psi : G \to \mathrm{GL}(V_2) \) ;
			\item \( V_1 / \ker(f) \cong \mathrm{im}(f) \) au sens de représentations linéaires de \( G \).
		\end{enumerate}
\end{lemma}
\end{frame}	




\begin{frame}
	\frametitle{Caractère et Représentation Unitaire}
	
	\begin{definition}
Soit \( \rho : G \to \mathrm{GL}(V) \) une représentation linéaire sur \( G \). Le caractère de \( V \), noté \( \chi_V \), est la fonction
		\[
		\chi_V : G \to \mathbb{C}
		\]
		définie pour tout \( g \in G \) par
		\[
		\chi_V(g) := \operatorname{Tr}(\rho(g)),
		\]
		où \( \operatorname{Tr} \) désigne la trace.
	\end{definition}
	
	
	\begin{theorem} 
Soit \( (\rho, V)_G \) une représentation linéaire. On peut munir \( V \) d'un produit hermitien \( (.,.)_V \) qui rend la représentation linéaire \( (\rho, V)_G \) unitaire.
	\end{theorem}
	
\end{frame}


\begin{frame}
\begin{proposition}
Soient \( \rho_V : G \rightarrow \mathrm{GL}(V) \) et \( \rho_W : G \rightarrow \mathrm{GL}(W) \) deux représentations linéaires sur \( G \) de degrés \( n \) et \( m \) (\( n, m \in \mathbb{N}^* \)), et de caractères \( \chi_V \) et \( \chi_W \) respectivement. On a :
	\begin{enumerate}[label=\roman*)]
		\item \( \chi_V(1) = \dim V \);
		\item \( \chi_{V \oplus W} = \chi_V + \chi_W \);
		\item \( \chi_V(g^{-1}) = \overline{\chi_V(g)} \) \quad pour \( \forall g \in G \).
	\end{enumerate}
\end{proposition}

\begin{theorem}[Frobenius] 
Pour tout groupe fini \( G \), le nombre de représentations irréductibles non-isomorphes deux à deux de \( G \) est exactement égal au nombre \( c(G) \) de classes de conjugaison de \( G \).
\end{theorem}

\begin{proposition} 
Deux représentations d’un groupe \( G \) sont isomorphes si et seulement si elles ont le même caractère.
\end{proposition}

\end{frame}



\begin{frame}
\begin{definition}
	On dit qu'une représentation linéaire \( \rho : G \rightarrow \mathrm{GL}(V) \) est \emph{irréductible} si l'espace vectoriel \( V \) n'est pas réduit à \( \{0\} \) et si \( V \) ne possède aucun sous-espace invariant par \( \rho \) autre que \( \{0\} \) et \( V \).
\end{definition}

\begin{theorem} 
	Soit \( \rho : G \to \mathrm{GL}(V) \) une représentation linéaire de \( G \) dans \( V \) et soit \( W \) un sous-espace vectoriel de \( V \) stable sous \( G \). Alors, il existe un complément \( W^0 \) de \( W \) dans \( V \) qui est stable sous \( G \).
\end{theorem}

\begin{theorem}[Théorème de Maschke] 
	Toute représentation linéaire \( \rho : G \to \mathrm{GL}(V) \) sur \( G \) dans un espace vectoriel complexe de dimension finie se décompose en somme directe de représentations irréductibles.
\end{theorem}

\end{frame}



\begin{frame}
\begin{theorem}[Lemme de Schur] 
	Soient \( \rho_1 : G \rightarrow V_1 \) et \( \rho_2 : G \rightarrow V_2 \) deux représentations irréductibles de \( G \). \\
	Soit \( f : V_1 \rightarrow V_2 \) une application linéaire vérifiant
	\[
	\forall g \in G, \quad f \circ \rho_1(g) = \rho_2(g) \circ f.
	\]
	On a les propriétés suivantes :
	\begin{enumerate} [label=\roman*)]
		\item[(i)] Si \( \rho_1 \) et \( \rho_2 \) ne sont pas isomorphes, alors \( f = 0 \).
		\item[(ii)] Si \(V_1 = V_2 \) et \( \rho_1 = \rho_2 \), alors \( f \) est une homothétie.
	\end{enumerate}
\end{theorem}

\begin{corollary}
	Soit \( h  : V \to W \) une application linéaire. Posons :
	\[
	h_0 = \frac{1}{N} \sum_{g \in G} (\rho_W(g))^{-1} \circ h \circ \rho_V(g),
	\]
	où \( N = \mathrm{Card}(G) \).
	Alors :
	\begin{enumerate}[label=\roman*)]
		\item Si \( \rho_V \) et \( \rho_W \) ne sont pas isomorphes, on a \( h_0 = 0 \).
		\item Si \( V = W \) et \( \rho_V = \rho_W \), alors \( h_0 \) est une homothétie de rapport \( \frac{1}{n} \text{Tr}(h) \), où \( n = \dim(V) \).
	\end{enumerate}
\end{corollary}
\end{frame}



\begin{frame}
\begin{remark} 
	Une traduction matricielle du corollaire précédent est: 
	si \( \varphi \) et \( \psi \) sont des fonctions \( G \to \mathbb{C} \), alors
	\[
	\langle \varphi, \psi \rangle = \frac{1}{g} \sum_{t \in G} \varphi(t^{-1})\psi(t) 
	= \frac{1}{g} \sum_{t \in G} \varphi(t)\psi(t^{-1}). \tag{1}
	\]
\end{remark}

\begin{proposition} 
	Pour \( g \in G \), soient \( (r_{i_1 j_1}(g)) \) et \( (u_{i_2 j_2}(g)) \) les matrices respectives de \( \rho_V(g) \) et \( \rho_W(g) \) dans les bases \( \mathcal{B}_1 \), \( \mathcal{B}_2 \) de \( V_1 \), \( V_2 \). Alors :
	\begin{enumerate}[label=\roman*)]
		\item Si \( \rho_V \) et \( \rho_W \) ne sont pas isomorphes, on a \( \langle u_{i_2 j_2}, r_{j_1 i_1} \rangle = 0 \) pour tous indices \( i_1, j_1, i_2, j_2 \).
		\item Si \( V_1 = V_2 \) est de dimension \( n \) et \( \rho_V = \rho_W \) (auquel cas on prend \( \mathcal{B}_1 = \mathcal{B}_2 \) et on a \( r_{ij} = u_{ij} \) pour tous indices \( i, j \)), alors \( \langle r_{i_2 j_2}, r_{j_1 i_1} \rangle = 0 \) si \( i_1 \neq i_2 \) ou \( j_1 \neq j_2 \), et \( \langle r_{ij}, r_{ji} \rangle = \frac{1}{n} \) pour tous indices \( i, j \).
	\end{enumerate}
\end{proposition}


\end{frame}


\begin{frame}
	\begin{theoreme}
		\begin{enumerate}[label=\roman*)]
			\item Soit \( \chi \) le caractère d’une représentation irréductible \( \rho \) de \( G \). Alors, \( (\chi | \chi) = 1 \).
			\item Soient \( \chi_1 \) et \( \chi_2 \) les caractères de deux représentations irréductibles non isomorphes \( \rho_1 \) et \( \rho_2 \). Alors, \( (\chi_1 | \chi_2) = 0 \).
		\end{enumerate}
	\end{theoreme}

\begin{theoreme}  
	Soit \( G \) un groupe. Les propriétés suivantes sont équivalentes :
	\begin{enumerate}
		\item[(i)] \( G \) est abélien.
		\item[(ii)] Toutes les représentations irréductibles de \( G \) sont de degré 1.
	\end{enumerate}
\end{theoreme}
\end{frame}




	\section{Résultats}
\begin{frame}{Convention générale}
Dans la suite, sauf mention contraire, tous les espaces vectoriels seront supposés de dimension finie et définis sur un corps commutatif \( \mathbb{K} \), où :
	\[
	\mathbb{K} \in \{ \mathbb{R}, \mathbb{C} \}.
	\]
\end{frame}

	% --- Propriété universelle ---
\begin{frame}{Propriété Universelle du Produit Tensoriel}
	\begin{block}{\textbf{Propriété Universelle}}
		Soient \( V_1, V_2, \ldots, V_p \) et \( T \) des espaces vectoriels, et
		\[
		\otimes : V_1 \times \cdots \times V_p \to T
		\]
		une application \( p \)-linéaire. Cette application a la \textbf{propriété universelle} si :
		\begin{enumerate}[label=\roman*)]
			\item Les vecteurs \( x_1 \otimes \cdots \otimes x_p \), pour \( x_i \in V_i \), engendrent \( T \).
			\item Toute application \( p \)-linéaire \( \varphi : V_1 \times \cdots \times V_p \to H \) (où \( H \) est un espace vectoriel quelconque) s’écrit :
			\[
			\varphi(x_1, \ldots, x_p) = f(x_1 \otimes \cdots \otimes x_p)
			\]
			avec \( f : T \to H \) linéaire.
		\end{enumerate}
	\end{block}
	
\end{frame}

% --- Définition du produit tensoriel ---
\begin{frame}{Définition du Produit Tensoriel}
	\begin{definition}
		Le \textbf{produit tensoriel} des espaces \( V_1, V_2, \ldots, V_p \) est un couple \( (T, \otimes) \) tel que :
		\[
		\otimes : V_1 \times \cdots \times V_p \to T
		\]
		est une application \( p \)-linéaire vérifiant la propriété universelle.
		
		L’espace \( T \) est noté :
		\[
		V_1 \otimes \cdots \otimes V_p.
		\]
	\end{definition}
\begin{block}{\textbf{Propriété} \cite{greub2012linear}}
	Le couple \( (T, \otimes) \) est \textbf{unique à isomorphisme près}.
\end{block}

\end{frame}


	
	\section{Conclusion}
	\begin{frame}{Conclusion}
	Parvenus au terme de notre travail, où il était principalement question pour nous d'étendre les représentations linéaires aux groupes infinis, il en ressort que nous avons construit les représentations linéaires des groupes infinis vus comme produits infinis de groupes finis.
	Les caractères de tels groupes sont construits comme des suites, dont l'existence est liée aux propriétés de convergence de ces suites. Nous avons donné quelques propriétés des représentations irréductibles de ces groupes.
	Enfin, nous avons étudié le cas particulier de \(\widehat{\mathbb{Z}}\) qui est le complété profini de \( \mathbb{Z} \). Ici, nous avons construit explicitement les représentations linéaires, calculé les caractères associés, et donné quelques proprietés d’irréductibilité.\\
	Excellence M. le president du jury, très chers honorables membres du jury, à vos titres et grades respectifs, nous n'avons pas la pretention d'avoir effectue un travail parfait. Nous admettons que notre travil puiss comporter des impertions et c'est pour cette raison que nous nous soumettons a vos differentes remarques et suggestion tout en vous rasurant de les prendre en consideration pour ameliorer le present travail.\\
	
	Merci pour votre aimable attention, J'en ai fini.
	\end{frame}
	
	
\end{document}
