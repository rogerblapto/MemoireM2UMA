\documentclass[a4paper, 14pt]{report}
\usepackage[french]{babel}              %pour pour rediger un document en francais 
\usepackage[utf8]{inputenc}             %pour les acsents
\usepackage[T1]{fontenc}                %avec ca le texte est moins foncé
\usepackage{graphicx}                    %pour inserer les images
\usepackage{bibtopic}                     %pour inserer la bibliographie
\usepackage{setspace}                     %pour inserer la bibliographie
\singlespacing
\usepackage{enumitem}
\usepackage{pifont}
\usepackage{hyperref}       % table de matieres avec lien
\usepackage{titlesec}                  %pour les subsubsection
\setcounter{secnumdepth}{4}            %pour les subsubsection egalemant
\usepackage{fancyhdr}                 %PAGINATION
\usepackage{lastpage}
\usepackage[top=2.5cm, bottom=2.5cm, left=2.5cm, right=2.5cm]{geometry}
\frenchbsetup{StandardLists=true}           
\usepackage{hyphenat}
\usepackage{multirow}
\usepackage{tabularx}
\usepackage{amsmath}
\usepackage{amsfonts}
\usepackage{amssymb,amsthm}
\usepackage{tcolorbox}
\usepackage{xcolor} 
\usepackage{times}
\usepackage{caption}
\usepackage{lmodern,tikz,lipsum}
\usepackage[all,cmtip]{xy}
\usetikzlibrary{arrows.meta, positioning}
\usepackage{mathrsfs}
\usepackage{tikz-cd} 
\usepackage{mathpazo} 

\newcommand{\divides}{\mid}

% Définition du style des corollaires
\newtheorem{mycorollary}{Corollaire}[section] % Numérotation par section


\newcommand\framethispage[1][1cm]{%
	\tikz[overlay,remember picture,line width=5pt]
	\draw([xshift=(#1),yshift=(-#1)]current page.north west)rectangle
	([xshift=(-#1),yshift=(#1)]current page.south east);
}
\makeatletter
\renewcommand\listoffigures{%
	\section*{Liste des figures}%
	\@mkboth{\MakeUppercase\listfigurename}%
	{\MakeUppercase\listfigurename}%
	\@starttoc{lof}%
}
\makeatother


\titleformat{\chapter}[block] % Choix du format de titre (ici 'block' signifie que le titre est sur une ligne séparée)
{\normalfont\huge\bfseries\centering}   % Style du titre centré (police normale, taille 14pt, en gras)
{}                            % Pas de numéro de chapitre avant le titre
{0pt}                         % Espace entre le numéro du chapitre et le titre
{\titlerule[1mm]\vskip0.5ex\hspace{10pt}\bfseries\Huge} % Réduit l'espace au-dessus du titre
[\vskip1ex\titlerule]          % Ajoute une ligne en dessous du titre



% Personnalisation du pied de page
\fancyfoot[C]{\thepage}  % Numéro de page centré
\fancyfoot[L]{ \textit{Représentation linéaire, caractère et \\ 
		représentations linéaires irréductibles \\ 
		d’un groupe infini.}} % Texte à gauche dans le pied de page
\fancyfoot[R]{SOUNKOUA Roger © 2023-2024}  % Copyright et nom à droite dans le pied de page



\newtheorem{definition}{Définition}[section]
\newtheorem{remark}{Remarque}[section]
\newtheorem{example}{Exemple}[section]
\newtheorem{notation}{Notation}[section]
\newtheorem{proposition}{Proposition}[section]
\newtheorem{propriety}{Propriété}[section]
\newtheorem{theorem}{Théorème}[section]
\newtheorem{lemma}{Lemme}
\newtheorem{corollary}{Corollaire}[section] 


\newenvironment{mylemma}{\begin{lemma}\ \newline}{\end{lemma}}
\newenvironment{mydefinition}{\begin{definition}\ \newline}{\end{definition}}
\newenvironment{myremark}{\begin{remark}\ \newline}{\end{remark}}
\newenvironment{myexample}{\begin{example}\ \newline}{\end{example}}
\newenvironment{mynotation}{\begin{notation}\ \newline}{\end{notation}}
\newenvironment{myproposition}{\begin{proposition}\ \newline}{\end{proposition}}
\newenvironment{mypropriety}{\begin{propriety}\ \newline}{\end{propriety}}
\newenvironment{mytheorem}{\begin{theorem}\ \newline}{\end{theorem}}
\newenvironment{myproof}[1][Démonstration]{%
	\noindent \textbf{#1:}%
}{%
	\hfill $\square$ % Ajoute un carré à la fin de la preuve
}


% survol dans la table des matieres
\hypersetup{
	colorlinks=true,
	linkcolor=blue,
	filecolor=magenta,     
	urlcolor=cyan,
	pdfpagemode=FullScreen,
	linktoc=all,
}

\makeatletter
\renewcommand{\@pnumwidth}{3em} % Increase the space for the page numbers in the TOC
\renewcommand{\@tocrmarg}{4em}  % Increase the right margin for the TOC
\makeatother

\usepackage{etoolbox}
\makeatletter
\patchcmd{\@tocline}
{\hfil}
{\leaders\vrule height -0.4\baselineskip depth 0.4\baselineskip\hfil}
{}{}
\makeatother

% Redéfinir une taille de police spécifique
\newcommand{\applyfontsize}{%
	\fontsize{12}{12}\selectfont
}

\begin{document}
	
	\pagenumbering{roman} %POUR LA PAGINATION (pour preciser que la debut de la pagination est en romain
	
	\pagestyle{fancy}
	\renewcommand\headrulewidth{1mm}
	\renewcommand\footrulewidth{0.8mm}  %ppour le trait du bas de page 
	\renewcommand{\baselinestretch}{1.5} %pour l'interligne
	
	\framethispage[1cm]% le cadre est à 2 cm des bords de la feuille
	
	\begin{tabularx}{\textwidth}{>{\centering}XcX<{\centering}}
		
		
		REPUBLIC OF CAMEROON &  \multirow{3}{*}{$\quad\quad\quad$\includegraphics[scale=0.3]{../TPE_LATEX/img/LogoUMA.png}$\quad\quad\quad$} &   RÉPUBLIQUE DU CAMEROUN\\
		PEACE-WORK-FATHERLAND &  & PAIX-TRAVAIL-PATRIE \\
		******** &  & *********\\
		UNIVERSITY OF MAROUA  &  & UNIVERSITÉ DE MAROUA \\
		******** &  & *********\\
		FACULTY OF SCIENCES &  & FACULTÉ DES SCIENCES \\
		******** &  & *********\\
		DEPARTEMENT OF &      \multirow{3}{*}{$\quad\quad\quad$\includegraphics[scale=0.12]{../TPE_LATEX/img/logo_fs.png}$\quad\quad\quad$}
		& DEPARTEMENT DE \\
		MATHEMATICS AND  &  & MATHÉMATIQUES ET\\
		COMPUTER SCIENCES &  & D’ INFORMATIQUE \\
		******** &  & *********\\
		
		
		&   & \\
		
	\end{tabularx}
	
	\begin{center}
		\begin{tabularx}{\textwidth}{>{\centering}XcX<{\centering}}
			
			& & \\
			& DEPARTEMENT DE MATHÉMATIQUES - INFORMATIQUE & \\
			
		\end{tabularx}
	\end{center}
	\begin{tabularx}{\textwidth}{>{\centering}XcX<{\centering}}
		& & \\
		%& & \\
		
	\end{tabularx}\\
	\begin{center}
		
		\textbf{ SUJET :}
		\begin{tcolorbox}[
			colframe=blue!70,      % Couleur du contour
			colback=blue!10,       % Couleur de fond
			coltitle=black,        % Couleur du titre (si utilisé)
			boxrule=1mm,         % Épaisseur du contour
			arc=5mm,               % Arrondi des coins
			width=\textwidth ,      % Largeur de la boîte  
			center                 % Centrer la boîte
			]
			\centering
			
			{\textbf{\large \\
					REPRÉSENTATION  LINÉAIRE, CARACTÈRE ET \\
					REPRÉSENTATIONS LINÉAIRES IRRÉDUCTIBLE \\
					D'UN GROUPE INFINI.\\ }}
		\end{tcolorbox}
		
		\vspace{1cm}
		
		
		{\fontsize{14}{12}\selectfont
			
			Mémoire présenté en vue de l’obtention du diplôme de:
			\begin{center}
				\textbf{Master II mathematiques.}
			\end{center}
			
			\textbf{Spécialité}: Algèbre et Géometrie (ALG).
			
			\begin{center}
				\textbf{Option}: Algèbre.
			\end{center}
			
			\begin{center}
				Par
			\end{center}
			
			\begin{center}
				\textbf{SOUNKOUA Roger}
			\end{center}
			
			\begin{center}
				\textbf{Matricule }: 21A1754FS
			\end{center}
			
			\begin{center}
				Licence (en mathématiques)
			\end{center}
			
			\begin{center}
				Sous la Direction de:
			\end{center}
			
			\begin{center}
				\textbf{ Dr Gilbert MANTIKA }
			\end{center}
			\begin{center}
				Chargé de cours de l'Université de Maroua
			\end{center}
			\begin{center}
				
				\begin{tabularx}{\textwidth}{>{\centering}XcX<{\centering}}
					
					& & \\
					& & \\
					
				\end{tabularx}\\
				
				
				\begin{tabularx}{\textwidth}{>{\centering}XcX<{\centering}}
					
					& & \\
					& & \\
				\end{tabularx}\
				
				\textbf{Année académique 2023-2024}
				
			\end{center} \thispagestyle{empty}
			%\pagenumbering{none}
		}
		
		\normalsize
		
		
		
		
	\end{center} \thispagestyle{empty}
	%\pagenumbering{none}
	
	\newpage
	\begin{onehalfspace} %interligne
		\lhead{}
		\rhead{}
		\chead{}
		
		
		\chapter*{DEDICACE}
		\addcontentsline{toc}{chapter}{DEDICACE. . . . . . . . . . . . . . . . . . . . . . . . . . . . . . . . . . . . . . .  . . . . . . . . .  . . . .  . . . . . .  . . .  .}
		
		\begin{center}
			\textbf{\textbf{ \LARGE À ma maman Massa Salomé.}}
			\vspace{1cm} % Espace vertical
		\end{center}
		
		\chapter*{REMERCIEMENTS}
		\addcontentsline{toc}{chapter}{REMERCIEMENTS. . . . . . . . . . . . . . . . . . . . . . . . . . . . . . . . . . . . . . .  . . . . . . . . .  . . . .  . .}
		% Partie avec la taille personnalisée
		{
			\applyfontsize % Application locale de la taille de police 14pt
			
			Ce travail a été réalisé au Laboratoire de Mathématiques de la Faculté des Sciences de l’Université de Maroua, sous l’encadrement du Dr Gilbert Mantika. Je tiens à lui exprimer ma profonde gratitude et mes sincères remerciements pour m’avoir offert l’opportunité de travailler sous sa direction.
			
			Je souhaite également exprimer mes remerciements à :
			\begin{itemize}
				\item Mme le Doyen de la Faculté des Sciences de l’Université de Maroua, Pr Ngo Bum Elisabeth, pour ses efforts constants visant à assurer une excellente qualité des enseignements, ainsi que pour l'attention particulière qu'elle nous a accordée à chaque fois que nous l'avons sollicitée ;
				\item Pr Joseph Dongho, Chef du Département de Mathématiques et Informatique, pour son soutien et sa disponibilité tout au long de cette période ;
				\item Dr Luc Emery Diekouam Fotso, pour ses enseignements, ses encouragements et ses précieux conseils ;
				\item Dr Aminatou Pecha, pour ses enseignements et son accompagnement bienveillant ;
				\item Dr Kemajou Théophile, pour ses encouragements constants à poursuivre dans le domaine de la géométrie ;
				\item Ma mère Massa Salomé qui a sacrifié certains de ses besoins personnels pour la réussite de mes études ;
				\item Mon père Koge André pour son soutien indéfectible et ses nombreuses aides ;
				\item Mes frères et sœurs, pour leurs conseils avisés et leur soutien financier ;
				\item Mes camarades de promotion, pour leur esprit d’entraide et leur amitié.
			\end{itemize}
			
			Je remercie également tous les enseignants de la Faculté des Sciences de l’Université de Maroua ainsi que toutes les personnes, de près ou de loin, qui ont contribué à la réalisation de ce travail.
			
			
		}
		
		% Partie avec la taille personnalisée
		{
			\applyfontsize % Application locale de la taille de police 14pt
			
			%\chapter*{Table de matières}
			\tableofcontents
		}
		
		
		\chapter*{RESUMÉ}
		\addcontentsline{toc}{chapter}{RESUMÉ. . . . . . . . . . . . . . . . . . . . . . . . . . . . . . . . . . . . . . .  . . . . . . . . .  . . . .  . . . . . .  . . .  . .}
		% Partie avec la taille personnalisée
		{
			\applyfontsize % Application locale de la taille de police 12pt
			Ce mémoire explore les représentations linéaires des groupes infinis, en mettant un accent particulier sur les groupes profinis. Ces groupes, définis comme des limites projectives de groupes finis, possèdent des propriétés topologiques et algébriques remarquables, telles que la compacité et la discontinuité totale. Dans une première partie, les bases des représentations linéaires des groupes finis, incluant les caractères et les représentations irréductibles, ont été établies pour constituer un socle théorique. La seconde partie étend ces concepts aux groupes infinis à l’aide des produits tensoriels infinis, permettant de définir et d’étudier les représentations de ces structures complexes. L’étude approfondie du complété profini \( \widehat{\mathbb{Z}} \) illustre concrètement ces concepts et met en évidence des liens entre les propriétés topologiques et algébriques des groupes profinis. Enfin, ce travail ouvre des perspectives prometteuses, notamment en théorie de Galois, en cryptographie, et dans l’analyse des interactions entre les structures algébriques et topologiques des groupes compacts.
			
			
			\section*{Mots-clés}
			Groupes infinis, groupes profinis, complété profini, représentations linéaires, caractères et produit tensoriel.
			
		}
		
		
		\chapter*{ABSTRACT}
		\addcontentsline{toc}{chapter}{ABSTRACT. . . . . . . . . . . . . . . . . . . . . . . . . . . . . . . . . . . . . . .  . . . . . . . . .  . . . .  . . . . . .  . . . }
		% Partie avec la taille personnalisée
		{
			\applyfontsize % Application locale de la taille de police 14pt
			This thesis explores the linear representations of infinite groups, with a specific focus on profinite groups. These groups, defined as projective limits of finite groups, exhibit remarkable topological and algebraic properties such as compactness and total discontinuity. In the first part, the foundations of linear representations of finite groups, including characters and irreducible representations, are established as a theoretical basis. The second part extends these concepts to infinite groups using infinite tensor products, enabling the definition and study of representations of these complex structures.  
			A detailed study of the profinite completion \( \widehat{\mathbb{Z}} \) provides a concrete illustration of these concepts and highlights the interplay between the topological and algebraic properties of profinite groups. Finally, this work opens promising perspectives, particularly in Galois theory, cryptography, and the analysis of the interactions between algebraic and topological structures of compact groups.
			
			\section*{Keywords}
			Infinite groups, profinite groups, profinite completion, linear representations, characters and tensor product.
			
		}
		
		
		
		
		
		\chapter*{INTRODUCTION}
		% Partie avec la taille personnalisée
		{
			\applyfontsize % Application locale de la taille de police 12pt
			La théorie des représentations linéaires d’un groupe constitue un outil fondamental permettant de représenter les éléments d’un groupe abstrait par des matrices inversibles sur un corps donné. Cette approche, qui traduit des problèmes complexes d’algèbre abstraite en des problèmes d’algèbre linéaire plus accessibles, repose sur la notion de représentation linéaire. Etant donné un corps $\mathbb{K}$, une représentation $K$-linéaire d’un groupe fini $G$ est un homomorphisme de groupes \[ \rho : G \to \text{GL}(V) \]
			où $V$ est un $\mathbb{K}$-espace vectoriel et $\text{GL}(V)$ désigne le groupe des applications linéaires bijectives de $V$ sur lui-même \cite{serre1971representation}. 
			La théorie des représentations linéaires des groupes finis a été développée pour la première fois par le mathématicien allemand Ferdinand Georg Frobenius en 1897. Il introduit notamment la notion de représentation linéaire d’un groupe fini et jette les bases de la théorie des caractères des groupes \cite{minkowski1911gesammelte}. Par la suite, Jean-Pierre Serre a approfondi ces travaux et formalisé cette théorie dans son ouvrage \emph{Représentations linéaires des groupes finis}, publié en 1968, où il développe une analyse détaillée des représentations et des caractères des groupes finis \cite{serre1971representation}.
			Le but principal de ce projet est d’étendre les représentations linéaires aux groupes infinis.
			La question centrale que nous nous posons est donc :  
			comment généraliser la théorie des représentations linéaires des groupes finis aux groupes infinis, notamment au complété profini d'un groupe ?
			Pour répondre à cette interrogation, nous suivrons une démarche en deux étapes.
			Il va s’agir dans un premier temps de définir quelques concepts de base et quelques résultats immédiats relatifs aux représentations linéaires des groupes finis ; ensuite définir la notion de caractère d’une représentation linéaire d’un groupe fini et donner quelques unes de ses propriétés. Aussi, il sera question de définir les notions de représentation linéaire irréductible d’un groupe fini, puis donner quelques propriétés relatives.
			Dans un second temps, nous allons rappeler d’abord le produit tensoriel infini d’espaces vectoriels
			sur un même corps comme une limite inductive. Puis nous définirons une représentation linéaire d’un produit arbitraire de groupes finis sur un produit tensoriel infini d’espaces vectoriels sur un même corps, définir son caractère ; puis caractériser les représentations irréductibles de ce groupe infini. Enfin, nous déterminerons une représentation linéaire du complété profini d’un groupe qui est une construction d’un groupe profini.
			
		}
		\pagenumbering{arabic}
		\addcontentsline{toc}{section}{INTRODUCTION}
		
		
		
		
		\chapter{Préliminaires}
		
		% Partie avec la taille personnalisée
		{
			\applyfontsize % Application locale de la taille de police 12pt
			
			\section*{Introduction}
			
			Ce premier chapitre pose les bases essentielles pour l’étude des représentations linéaires des groupes en introduisant les concepts fondamentaux et les outils mathématiques indispensables. Nous commençons par un rappel sur la théorie des groupes \cite{schaub1997}, en définissant les groupes, sous-groupes et homomorphismes, ainsi que les notions de groupe topologique et d’espace topologique. Ensuite, nous abordons les espaces vectoriels et les applications linéaires \cite{lang2012algebra}, qui constituent le cadre naturel des représentations. Afin de généraliser ces concepts, nous introduisons les catégories et foncteurs, ainsi que les limites projectives et inductives \cite{maclane1971categories}, qui permettent d’unifier et d’étendre les constructions algébriques. Enfin, nous présentons la notion de complété profini d’un groupe, qui intervient notamment dans l’étude des représentations continues des groupes topologiques. Ces éléments constituent le socle théorique indispensable pour comprendre la structure et le comportement des représentations linéaires et serviront de fondation aux développements ultérieurs.
			\addcontentsline{toc}{section}{Introduction}
			
			\section{Groupes, sous-groupes et homomorphismes de groupes}
			\subsection{Sur les groupes}
			
			
			\begin{definition} \cite{schaub1997} \\
				Un groupe est un couple  $(G,*)$ où $G$ un ensemble non vide et $*$ une loi de composition interne
				\[
				* : G \times G \longrightarrow G
				\]
				\[
				(x, y) \longmapsto x * y
				\]
				vérifiant :
				\begin{enumerate} [label=\roman*)]
					\item $*$ est associative, c'est-à-dire,\(\forall x, y, z \in G, \ (x * y) * z = x * (y * z)\);
					\item $G$ possède un élément neutre pour la loi $*$,c'est-à-dire,\(\exists e \in G\) tel que \(\forall x \in G, \\ \ x * e = e * x = x\);
					\item tout  élément de $G$ est inversible (ou possède un  élément symetrique) dans $G$,c'est-à-dire \(\forall x \in G, \ \exists y \in G\) tel que \(x * y = y * x = e\).
				\end{enumerate}
			\end{definition}
			
			\begin{propriety} \cite{schaub1997}
				\begin{enumerate}[label=\roman*)]
					\item L'élément neutre \(e\) est unique.
					\item Un élément symétrique d'un groupe est unique.
				\end{enumerate}
			\end{propriety}
			
			\begin{example} 
				\begin{enumerate} \
					\item L’ensemble des nombres entiers relatifs muni de l’addition est un groupe, noté \((\mathbb{Z}, +)\).
					\item \((M_n(\mathbb{C}), +)\), où \(M_n(\mathbb{C})\) désigne l’ensemble des matrices \((n, n)\) à coefficients dans \(\mathbb{C}\).
					\item \((GL_n(\mathbb{C}), \times)\), où \(GL_n(\mathbb{C})\) désigne l’ensemble des matrices \((n, n)\) inversibles à coefficients dans \(\mathbb{C}\).Ce groupe est appelé groupe général linéaire.
				\end{enumerate}
			\end{example}
			
			
			\begin{definition}  \cite{ribet2004graduate}\\
				Si \((G, \ast)\) est un groupe tel que la loi \(\ast\) satisfasse à la propriété
				\[
				\forall x, y \in G, \ x \ast y = y \ast x,
				\]
				le groupe \((G, \ast)\) est dit \textbf{commutatif} ou encore \textbf{abélien}.
			\end{definition}
			
			\begin{myexample}
				Les groupes \((\mathbb{Z}, +)\), \((\mathbb{Q}^*, \times)\), \((M_n(\mathbb{C}), +)\)
				sont des groupes abéliens.
			\end{myexample}
			
			
			
			\begin{definition} \cite{schaub1997} 
				\begin{enumerate}[label=\roman*)]
					\item On appelle ordre de  \((G,*)\) le cardinal (nombre d'éléments) de \( G \).
					\item Le groupe \((G, *)\) est dit fini si l’ensemble sous-jacent \( G \) est fini, sinon il sera dit infini.
				\end{enumerate}
				
			\end{definition}
			
			\begin{mynotation}
				Dans la suite, un groupe $(G,*)$ sera noté par $G$ et le cardinal de $G$ par $|G|$ .
			\end{mynotation}
			
			\begin{definition}\cite{schaub1997} \\
				Soient \( (G, *) \) et \( (H, \cdot) \) deux groupes. Leur \textit{produit direct} est un groupe \( (G \times H, \circ) \), défini par les propriétés suivantes :
				\begin{enumerate} [label=\roman*)]
					\item \textbf{L'nsemble sous-jacent.}  \\
					C'est l'ensemble :
					\[
					G \times H = \{(g, h) \mid g \in G, h \in H\}.
					\]
					\item \textbf{La loi de composition.}  \\
					Elle est donnée par :
					\[
					(g, h) \circ (g', h') = (g * g', h \cdot h'),
					\]
					où \( * \) est l'opération de \( G \) et \( \cdot \) est celle de \( H \).
					\item \textbf{L'élément neutre.}  \\
					L'élément neutre de \( G \times H \) est \((e_G, e_H)\), où \( e_G \) est l'élément neutre de \( G \) et \( e_H \) est celui de \( H \).
					\item \textbf{Inverse.}  \\ 
					L'inverse d'un élément \((g, h) \in G \times H\) est donné par :
					\[
					(g, h)^{-1} = (g^{-1}, h^{-1}),
					\]
					où \( g^{-1} \) est l'inverse de \( g \) dans \( G \) et \( h^{-1} \) est l'inverse de \( h \) dans \( H \).
				\end{enumerate}
			\end{definition}
			
			
			\begin{propriety} \cite{schaub1997} \\
				Si \( |G| \) et \( |H| \) désignent respectivement les cardinaux de \( G \) et \( H \), alors \[
				|G \times H| = |G| \cdot |H|.
				\]
			\end{propriety}
			
			
			\begin{definition} \cite{schaub1997}\\
				Soit \( (G_i, \circ_i) \) une famille de groupes finis indexée par l'ensemble \( \{1, 2, \dots, n\} \), où \( n \in \mathbb{N}^{*} \). Le \textit{produit direct fini} de cette famille est un groupe \( \left( \prod_{i=1}^{n} G_i, \circ \right) \), défini par les propriétés suivantes : 
				\begin{enumerate}[label=\roman*)]
					\item \textbf{Ensemble sous-jacent.} \\
					L'ensemble sous-jacent est constitué des familles indexées par \( \{1, 2, \dots, n\} \) :  
					\[
					\prod_{i=1}^{n} G_i = \left\{ (g_i)_{i=1}^{n} \mid g_i \in G_i \text{ pour tout } i \in \{1, 2, \dots, n\} \right\}.
					\]
					\item \textbf{Loi de composition.} \\
					La loi de composition \( \circ \) est définie composante par composante :  
					\[
					(g_i)_{i=1}^{n} \circ (g_i')_{i=1}^{n} = (g_i \circ_i g_i')_{i=1}^{n},
					\]  
					où \( \circ_i \) désigne l'opération du groupe \( G_i \) pour chaque \( i \in \{1, 2, \dots, n\} \).
					\item \textbf{Élément neutre.} \\
					L'élément neutre de \( \prod_{i=1}^{n} G_i \) est la famille \( (e_i)_{i=1}^{n} \), où \( e_i \) est l'élément neutre de \( G_i \) pour tout \( i \in \{1, 2, \dots, n\} \).
					\item \textbf{Inverse.} \\
					L'inverse d'une famille \( (g_i)_{i=1}^{n} \in \prod_{i=1}^{n} G_i \) est donné par :  
					\[
					(g_i)_{i=1}^{{n}^{-1}} = (g_i^{-1})_{i=1}^{n},
					\]  
					où \( g_i^{-1} \) est l'inverse de \( g_i \) dans \( G_i \) pour chaque \( i \in \{1, 2, \dots, n\} \).\\
				\end{enumerate}
			\end{definition}
			
			\begin{remark} \cite{schaub1997} \\
				Les propriétés de groupe (\( \circ \) associative, existence d’un neutre et d’inverses) découlent directement des propriétés des groupes \( G_i \).\\
				La famille de groupes \( (G_i, \circ_i) \) satisfait :
				\[
				\left| \prod_{i=1}^{n} G_i \right| = \prod_{i=1}^{n} |G_i|,
				\]
				avec \( |G_i| \) le cardinal de \( G_i \).\\
			\end{remark}
			
			\begin{propriety} \cite{schaub1997} \\
				Si chaque \( (G_i, \circ_i) \) est un groupe abélien, alors \( \left( \prod_{i=1}^{n} G_i, \circ \right) \) est aussi un groupe abélien.
			\end{propriety}
			
			
			\begin{theorem} \cite{pei1996chinese} \\
				Soit $m_1, m_2, \ldots, m_r$ une suite d'entiers positifs premiers entre eux deux à deux. Alors le système de congruences :
				\[
				\begin{cases}
					x \equiv a_1 \pmod{m_1} \\
					x \equiv a_2 \pmod{m_2} \\
					\vdots \\
					x \equiv a_r \pmod{m_r}
				\end{cases}
				\]
				a une solution unique $x$ modulo $M = m_1 \times m_2 \times \cdots \times m_r$ :
				\[
				x = a_1 M_1 y_1 + a_2 M_2 y_2 + \cdots + a_r M_r y_r
				\]
				avec
				\[
				M_i = \frac{M}{m_i}, \quad y_i M_i \equiv 1 \pmod{m_i}.
				\]
			\end{theorem}
			
			
			
			\begin{theorem} \cite{schwarzweller2009chinese}
				
				Soit \( n = p_1^{e_1} p_2^{e_2} \cdots p_k^{e_k} \) la décomposition en facteurs premiers de \( n \), où les \( p_i \) sont des nombres premiers distincts et les \( e_i \) des entiers positifs. Alors, l'anneau des entiers modulo \( n \), \( \mathbb{Z}/n\mathbb{Z} \), est isomorphe au produit direct des anneaux des entiers modulo \( p_i^{e_i} \), c'est-à-dire :
				\[
				\mathbb{Z}/n\mathbb{Z} \cong \mathbb{Z}/p_1^{e_1}\mathbb{Z} \times \mathbb{Z}/p_2^{e_2}\mathbb{Z} \times \cdots \times \mathbb{Z}/p_k^{e_k}\mathbb{Z}.
				\]	
			\end{theorem}
			
			\begin{myremark}
				Ce résultat permet de réduire les calculs modulo \( n \) à des calculs modulo les puissances des facteurs premiers \( p_i^{e_i} \), et de résoudre des systèmes d'équations congruentes.
			\end{myremark}
			
			
			\begin{definition} \cite{schaub1997} \\
				Soient $(G, \cdot)$ et $(G', *)$ deux groupes. Un homomorphisme de groupes de $G$ dans $G'$ est une application $f : G \rightarrow G'$ vérifiant :
				\[
				\forall (x, y) \in G \times G, \quad f(x \cdot y) = f(x) * f(y).
				\]
			\end{definition}
			
			\begin{notation} \
				\begin{enumerate} [label=\roman*)]
					\item On note $\mathrm{Hom}(G, G')$ l’ensemble des homomorphismes de groupes de $G$ dans $G'$.
					\item  On note $\mathrm{End}(G)$ l’ensemble des homomorphismes de groupes de $G$ dans lui-même, qu’on appelle endomorphismes de $G$.
				\end{enumerate}
			\end{notation}
			
			\begin{definition} \cite{schaub1997} \\
				Un élément $f$ de $\mathrm{Hom}(G, G')$ est un isomorphisme s’il existe un morphisme réciproque $g$ de $\mathrm{Hom}(G', G)$ vérifiant $g \circ f = \mathrm{id}_G$ et $f \circ g = \mathrm{id}_{G'}$.
			\end{definition}
			
			\begin{proposition} \cite{hall2018theory} \\
				Soient $G$ et $G'$ deux groupes et $f : G \to G'$ une application.
				\begin{itemize}
					\item[(i)] $f$ est un isomorphisme si et seulement si $f$ est un morphisme bijectif.
					\item[(ii)] Si $f$ est un isomorphisme, l’application réciproque $f^{-1}$ est un isomorphisme.
				\end{itemize}
			\end{proposition}
			
			\begin{definition} \cite{schaub1997} \\
				Un groupe \( G \) est dit monogène s'il admet un unique générateur \( a \in G \), c'est-à-dire si :
				\[
				G = \langle a \rangle = \{ a^k \mid k \in \mathbb{Z} \}.
				\]
				De plus, \( G \) est dit cyclique s'il est fini.
			\end{definition}
			
			\begin{theorem} \cite{schaub1997} \\
				Si \( G \) est un groupe cyclique d’ordre \( n \geq 1 \), alors \( G \) est isomorphe au groupe additif \( \mathbb{Z}/n\mathbb{Z} \).
			\end{theorem}
			
			\begin{theorem} \cite{schaub1997} \\
				Tout groupe fini d’ordre premier est cyclique.
			\end{theorem}
			
			\subsection{Sous-groupes}
			\begin{definition} \cite{hall2018theory} \\
				Un sous-groupe d'un groupe $G$ est un sous-ensemble non vide $H$ de $G$ tel que $H$ muni  de la loi induite par celle de $G$ est un groupe.
			\end{definition}
			
			\begin{theorem}  \cite{hall2018theory} \\
				Une partie $H$ non vide de $G$ est un sous-groupe de $(G,*)$ si :
				\begin{enumerate} [label=\roman*)]
					\item pour tous $x, y \in H$, on a $x * y \in H$;
					\item si $x \in H$, alors $x^{-1} \in H$.
				\end{enumerate}
			\end{theorem}
			
			
			\begin{remark} \cite{hall2018theory} \\
				Soit $(G,*)$ un groupe.
				Un critère pratique et plus rapide pour prouver qu'un sous-ensemble non vide $H$ de $G$ est un sous-groupe de $(G,*)$ est :
				\begin{enumerate} [label=\roman*)]
					\item $H$ contient au moins un élément;
					\item pour tout $x, y \in H$, $x * y^{-1} \in H$.
				\end{enumerate}
			\end{remark}
			
			\begin{mynotation}
				Si $H$ est un sous-groupe de $G$, on notera $H < G$.
			\end{mynotation}
			
			\begin{example} \
				\begin{enumerate}
					\item $(\mathbb{Z}, +) < (\mathbb{Q}, +) < (\mathbb{R}, +) < (\mathbb{C}, +).$
					\item Pour tout groupe $G$, on considère
					\[
					Z(G) = \{g \in G \mid \forall x \in G, gx = xg\}.
					\]
					C’est un sous-groupe de $G$, appelé le centre de $G$.
				\end{enumerate}
			\end{example}
			
			\begin{definition} \cite{ribet2004graduate}\\
				Un sous-groupe \( H \) de \( G \) est \emph{distingué} (on note \( H \triangleleft G \)) si pour tout \( g \in G \), \( Hg = gH \) (on dit aussi : invariant ou normal).
			\end{definition}
			
			\begin{propriety} \cite{ribet2004graduate}\\
				Soit \( G \) un groupe abélien. Tout sous-groupe \( H \) de \( G \) est également abélien, c'est-à-dire que pour tous \( x, y \in H \), on a \( xy = yx \).
			\end{propriety}
			
			
			
			
			
			\subsection{Notion d'espace topologique}
			
			\begin{definition} \cite{bourbaki2013general} \\
				On appelle structure \textit{topologique} (ou tout simplement une topologie) sur un ensemble \( X \) un ensemble \( \mathcal{O} \) de parties de \( X \) vérifiant :
				\begin{enumerate} [label=\roman*)]
					\item \( \emptyset \in \mathcal{O} \) et \( X \in \mathcal{O} \);
					\item Toute réunion d'éléments de \( \mathcal{O} \) est dans \( \mathcal{O} \);
					\item Toute intersection finie d'éléments de \( \mathcal{O} \) est dans \( \mathcal{O} \).
				\end{enumerate}
				Les éléments de \( \mathcal{O} \) sont appelés \textit{ouverts} et ceux de  \( X \) sont appelés \textit{points}.
			\end{definition}
			
			
			\begin{definition} \cite{bourbaki2013general} \\
				Un espace topologique est un couple \( (X, \mathcal{O}) \), où \( \mathcal{O} \) est une structure topologique définie sur \( X  \).
			\end{definition}
			
			\begin{definition}  \cite{bourbaki2013general} \\
				Soit \( X \) un espace topologique et \( A \) une partie quelconque de \( X \). Un voisinage de \( A \) est tout sous-ensemble de \( X \) qui contient un ouvert contenant \( A \). Les voisinages d'un sous-ensemble \( \{ x \} \) constitué d'un seul point sont également appelés voisinages du point \( x \).
			\end{definition}
			
			
			\begin{proposition} \cite{bourbaki2013general} \\
				Un ensemble est un voisinage de chacun de ses points si et seulement si il est ouvert.
			\end{proposition}
			
			
			\begin{definition} \cite{bourbaki2013general} \\
				La clôture d'un sous-ensemble \( A \) d'un espace topologique \( X \) est l'ensemble de tous les points \( x \in X \) tels que tout voisinage de \( x \) intersecte \( A \), et est notée par \( \overline{A} \).
			\end{definition}
			
			\begin{definition}  \cite{bourbaki2013general} \\
				Un sous-ensemble \( A \) d'un espace topologique \( X \) est dit dense dans \( X \) (ou simplement dense, s'il n'y a pas d'ambiguïté sur \( X \)) si \(\overline{A} = X\), c'est-à-dire si tout ouvert non vide \( U \) de \( X \) rencontre \( A \).
			\end{definition}
			
			
			
			\begin{proposition} \cite{bourbaki2013general} \\
				Soit \( X \) un espace topologique, \( x \in X \) et \( \mathcal{B}(x) \) l'ensemble de tous les voisinages de \( x \). Les ensembles \( \mathcal{B}(x) \) satisfont les propriétés suivantes :
				\begin{enumerate}
					\item[(V$_1$)] Pour tout ensemble \( U \subset X \), si \( U \supset V \) pour un certain \( V \in \mathcal{B}(x) \), alors \( U \in \mathcal{B}(x) \).
					\item[(V$_2$)] Pour tout entier naturel non nul \( n \), si \( V_1, V_2, \dots, V_n \in \mathcal{B}(x) \), alors \[ V_1 \cap V_2 \cap \dots \cap V_n \in \mathcal{B}(x) \]  .
					\item[(V$_3$)] Pour tout \( V \in \mathcal{B}(x) \), on a \( x \in V \).
					\item[(V$_4$)] Pour tout \( V \in \mathcal{B}(x) \), il existe un ensemble \( W \in \mathcal{B}(x) \) tel que \( W \subset V \) et, pour tout \( y \in W \), on a \( V \in \mathcal{B}(y) \).
				\end{enumerate}
			\end{proposition}
			
			
			\begin{definition} \cite{bourbaki2013general} \\
				Une application \( f : E \to F \) entre espaces topologiques est dite \textit{continue en un point} \( a \in E \) si l'image réciproque de tout voisinage de \( f(a) \) est un voisinage de \( a \).\\
				Elle est dite \textit{continue} si elle est continue en tout point de \( E \).
			\end{definition}
			
			
			\begin{definition} \cite{bourbaki2013general}\\
				Soit \(\{X_i\}_{i \in I}\) une famille d'espaces topologiques. Leur produit topologique est l'ensemble  
				\[
				X = \prod_{i \in I} X_i
				\]
				muni de la topologie produit, qui est la topologie initiale engendrée par les projections canoniques :
				\[
				\pi_i : X \to X_i, \quad \text{où} \quad \pi_i((x_j)_{j \in I}) = x_i.
				\]
			\end{definition}
			
			\begin{remark} \cite{bourbaki2013general}\\
				La topologie produit \(X = \prod_{i \in I} X_i \) est la plus grossière (la plus faible) rendant toutes les projections \(\pi_i : X \to X_i\) continues.
			\end{remark}
			
			\begin{propriety}  \cite{bourbaki2013general}\\
				Une application \( f:  \prod_{i \in I} Y_i \to \prod_{i \in I} X_i \) entre deux produits topologiques est continue si et seulement si chaque application  
				\[
				\pi_i \circ f: \prod_{i \in I} Y_i \to X_i
				\]
				est continue pour tout \( i \in I \).
			\end{propriety}
			
			\begin{theorem}	\cite{bourbaki2013general}\\
				Soit \( \{ X_i \}_{i \in I} \) une famille d'espaces topologiques. Si chaque \( X_i \) est compact, alors le produit \( \prod_{i \in I} X_i \) est compact dans la topologie produit.	
			\end{theorem}
			
			\subsection{Notion de groupe topologique}
			\begin{definition} \cite{bourbaki2013general} \\
				Un groupe \((G, \cdot)\) est dit topologique si l'ensemble $G$ est muni d’une topologie et satisfait les deux axiomes suivants :
				\begin{enumerate} [label=\roman*)]
					\item L'application $(x, y) \mapsto xy$ de $G \times G$ dans $G$ est continue.
					\item L'application $x \mapsto x^{-1}$ de $G$ dans $G$ (la symétrie du groupe $G$) est continue.
				\end{enumerate}
			\end{definition}
			
			\begin{example}\
				\begin{enumerate} 
					\item $\mathbb{R}$ avec l'Addition $(\mathbb{R}, +)$ est un groupe abélien.\\
					La topologie usuelle de $\mathbb{R}$ (induite par la valeur absolue) en fait un groupe topologique.\\
					L'addition et l'opposé $x \mapsto -x$ sont continues.
					\item  $\mathbb{C}^*$ avec la multiplication $(\mathbb{C}^*, \cdot)$ est un groupe multiplicatif.\\
					La topologie usuelle de $\mathbb{C}$ en fait un groupe topologique.\\
					La multiplication et l'inversion $z \mapsto z^{-1}$ sont continues.
				\end{enumerate}
			\end{example}
			
			
			
			\section{Espaces vectoriels et applications linéaires}
			\subsection{Espaces vectoriels}
			
			\begin{definition} \cite{lang2012algebra}\\
				Un anneau est un ensemble \( A \) non vide muni de deux opérations binaires \( + \) (addition) et \( \cdot \) (multiplication), telles que :
				\begin{enumerate} [label=\roman*)]
					\item \( (A, +) \) est un groupe abélien;
					\item La multiplication est associative : \(\forall a, b, c \in A, (a \cdot b) \cdot c = a \cdot (b \cdot c) \);
					\item La multiplication est distributive par rapport à l'addition :
					\begin{enumerate}
						\item \(\forall a, b, c \in R, a \cdot (b + c) = a \cdot b + a \cdot c \);
						\item \(\forall a, b, c \in R, (a + b) \cdot c = a \cdot c + b \cdot c \).
					\end{enumerate}
				\end{enumerate}
			\end{definition}
			
			\begin{definition} \cite{lang2012algebra}\\
				Un anneau \((A, +, \cdot)\) est appelé \emph{anneau unitaire} si la multiplication possède un élément neutre noté \( 1 \).
			\end{definition}
			
			
			\begin{definition} \cite{lang2012algebra}\\
				Un anneau \((A, +, \cdot)\) est appelé \emph{commutatif} ou \emph{abélien} si la multiplication est commutative, c'est-à-dire que pour tous \( a, b \in A \), on a \( a \cdot b = b \cdot a \).
			\end{definition}
			
			\begin{example} \
				\begin{enumerate}
					\item L’anneau nul \(\{0\}\).
					\item \((\mathbb{Z}/n\mathbb{Z}, +, \cdot)\) est un anneau commutatif.
				\end{enumerate}
			\end{example}
			
			
			\begin{definition} \cite{lang2012algebra}\\
				Un corps \(\mathbb{K}\) est un anneau dans lequel tout élément non nul admet un inverse pour la multiplication.\\
				Si de plus la loi $.$ est commutative, alors \(\mathbb{K}\) est un corps commutatif.
			\end{definition}
			
			\begin{example} \
				\begin{enumerate}
					\item Le corps des nombres complexes ( \(\mathbb{C} ,+ ,\cdot \) ).
					\item Le corps des nombres réels ( \(\mathbb{R} ,+ ,\cdot \) ).
				\end{enumerate}
			\end{example}
			
			
			\begin{definition} \cite{lang2012algebra}\\
				Un corps \(\mathbb{K}\) est dit \textit{algébriquement clos} si tout polynôme non constant à coefficients dans $\mathbb{K}$ possède au moins une racine dans \(\mathbb{K}\). Autrement dit, pour tout polynôme $P(X) \in \mathbb{K}[X]$ de degré $n \geq 1$, il existe un élément $\alpha \in \mathbb{K}$ tel que $P(\alpha) = 0$.
			\end{definition}
			
			
			\begin{definition} \cite{lang2012algebra}\\
				La caractéristique d'un corps $K$, notée $\text{char}(\mathbb{K})$, est le plus petit entier positif $p$ tel que
				\[
				1 + 1 + \cdots + 1 \quad \text{(p fois)} = 0
				\]
				dans $K$, si un tel entier existe. Si aucun tel entier n'existe, on dit que la caractéristique de $K$ est 0. Autrement dit :
				\[
				\text{char}(K) = 
				\begin{cases} 
					p & \text{si l'application } x \mapsto 1 + 1 + \cdots + 1 \quad \text{(p fois)} = 0 \\
					0 & \text{si une telle relation n'existe pas.}
				\end{cases}
				\]
			\end{definition}
			
			
			\begin{mynotation}
				Un corps ( \(\mathbb{K} ,+ ,\cdot \) ) sera noté \(\mathbb{K} \).
			\end{mynotation}
			
			
			\begin{myremark}
				Dans la suite, nous allons considérer le corps \(\mathbb{K}\) commutatif.
			\end{myremark}
			
			\begin{definition} \cite{lang2012algebra}\\
				Un \(\mathbb{K}\)-espace vectoriel , ou espace vectoriel sur \(\mathbb{K}\) est un tripet $( V,+ ,\cdot)$,tel que $( V,+)$ est un groupe abélien,$\cdot$ une multiplication par les scalaires, c'est-à-dire une application
				\[
				(a, x) \in \mathbb{K} \times V \mapsto a \cdot x \in V
				\]
				vérifiant les propriétés suivantes :
				
				\begin{enumerate} [label=\roman*)]
					\item \( 1 \cdot x = x \);
					\item \( a \cdot (x + y) = a \cdot x + a \cdot y \) (pour tout \( (a, b) \in K \times K \) et \( (x, y) \in V \times E \));
					\item \( (a + b) \cdot x = a \cdot x + b \cdot x \);
					\item \( (a \cdot b) \cdot x = a\cdot (b \cdot x) \).
				\end{enumerate}
			\end{definition}
			
			\begin{mynotation}
				Un \(\mathbb{K}\)-espace vectoriel $( V,+ ,\cdot)$ sera noté $V$.
			\end{mynotation}
			
			\begin{definition} \cite{lang2012algebra}\\
				Une partie non vide \( F \) de \( V \) est un sous-espace vectoriel de \( E \) si \( F \) est un sous-groupe de \( V \) tel que pour tout \( (a, x) \in \mathbb{K} \times V \),
				\[
				(x \in F) \Rightarrow (ax \in F).
				\]
			\end{definition}
			
			\begin{definition} \cite{axler2024linear}\\
				Supposons que \( V_1, \dots, V_m \) soient des sous-espaces de \( V \).
				\begin{itemize}
					\item La somme \( V_1 + \dots + V_m \) est appelée somme directe si chaque élément de \( V_1 + \dots + V_m \) peut être écrit de manière unique comme une somme \( v_1 + \dots + v_m \), où chaque \( v_k \in V_k \).
					\item Si \( V_1 + \dots + V_m \) est une somme directe, alors \( V_1 \oplus \dots \oplus V_m \) désigne \( V_1 + \dots + V_m \), avec la notation \( \oplus \) indiquant qu'il s'agit d'une somme directe.
				\end{itemize}
			\end{definition}
			
			
			\begin{definition} \cite{farhi2024polycopie}\\
				Soit \( V \) un espace vectoriel sur \( \mathbb{C} \). Un \textbf{produit hermitien} est une application :
				\[
				\langle \cdot, \cdot \rangle : V \times V \to \mathbb{C}
				\]
				qui satisfait les propriétés suivantes pour tous \( x, y, z \in V \) et pour tout \( \lambda \in \mathbb{C} \) :
				
				\begin{enumerate} [label=\roman*)]
					\item \textbf{Linéarité à gauche} :
					\[
					\langle \lambda x + y, z \rangle = \lambda \langle x, z \rangle + \langle y, z \rangle.
					\]
					
					\item \textbf{Conjugaison à droite (anti-linéarité)} :
					\[
					\langle x, \lambda y + z \rangle = \overline{\lambda} \langle x, y \rangle + \langle x, z \rangle.
					\]
					
					\item \textbf{Hermitianité (symétrie conjuguée)} :
					\[
					\langle x, y \rangle = \overline{\langle y, x \rangle}.
					\]
					
					\item \textbf{Positivité définie} :
					\[
					\langle x, x \rangle \geq 0 \quad \text{et} \quad \langle x, x \rangle = 0 \iff x = 0.
					\]
				\end{enumerate}
			\end{definition}
			
			\begin{definition} \cite{axler2024linear}\\
				Soient \( E_1 \) et \( E_2 \) deux \(\mathbb{K}\)-espaces vectoriels.  L'ensemble produit \( E = E_1 \times E_2 \) est muni d'une structure de \(\mathbb{K}\)-espace vectoriel définie comme suit :  
				\begin{enumerate}[label=\roman*)]
					\item   
					Le vecteur nul de \( E \) est donné par :  
					\[
					0_E = (0_{E_1}, 0_{E_2}).
					\]
					
					\item 
					L'addition est définie pour tous \( (x_1, x_2), (y_1, y_2) \in E \) par :  
					\[
					(x_1, x_2) + (y_1, y_2) = (x_1 + y_1, x_2 + y_2).
					\]
					
					\item   
					Pour tout \( \lambda \in \mathbb{K} \) et \( (x_1, x_2) \in E \), on définit :  
					\[
					\lambda \cdot (x_1, x_2) = (\lambda \cdot x_1, \lambda \cdot x_2).
					\]
				\end{enumerate}
				Ainsi, \( E_1 \times E_2 \) hérite d'une structure de \( \mathbb{K} \)-espace vectoriel en considérant les opérations coordonnées.
			\end{definition}
			
			\begin{definition} \cite{axler2024linear}\\
				Soit \( E_1, E_2, \dots, E_n \) une famille d'espaces vectoriels sur un corps \( \mathbb{K} \) où \( n \in \mathbb{N}^{*} \). L'ensemble produit \( E = \prod_{i=1}^{n} E_i \) est muni d'une structure de \( \mathbb{K} \)-espace vectoriel définie comme suit :
				
				\begin{enumerate}[label=\roman*)]
					\item Le vecteur nul est donné par 
					\[
					0_E = (0_{E_i})_{i=1}^{n};
					\]
					\item L'addition est définie par 
					\[
					(x_i)_{i=1}^{n} + (y_i)_{i=1}^{n} = (x_i + y_i)_{i=1}^{n};
					\]
					\item L'opération externe est définie par 
					\[
					\lambda \cdot (x_i)_{i=1}^{n} = (\lambda \cdot x_i)_{i=1}^{n};
					\]
					\item La multiplication est définie par
					\[
					(x_i)_{i=1}^{n} \cdot (y_i)_{i=1}^{n} = (x_i \cdot y_i)_{i=1}^{n} \text{ pour } i=1, 2, \dots, n.
					\]
				\end{enumerate}
			\end{definition}
			
			
			
			\begin{definition} \cite{lang2012algebra}\\
				Soit \( V \) est \(\mathbb{K}\)-espace vectoriel . Si \( v_1, v_2, \dots, v_n \in V \) et \( c_1, c_2, \dots, c_n \in \mathbb{K} \), alors le vecteur
				\[
				c_1 v_1 + c_2 v_2 + \dots + c_n v_n
				\]
				est appelé une combinaison linéaire de \( v_1, v_2, \dots, v_n \). Les scalaires \( c_1, c_2, \dots, c_n \) sont appelés des coefficients.
			\end{definition}
			
			\begin{definition}\cite{lang2012algebra}\\
				Soit \( V \) un \(\mathbb{K}\)-espace vectoriel sur \( F \) et que \( v_1, v_2, \dots, v_n \in V \). Alors
				\[
				\text{Span}\{v_1, v_2, \dots, v_n\} = \{ c_1 v_1 + c_2 v_2 + \dots + c_n v_n \mid c_1, \dots, c_n \in \mathbb{K} \}
				\]
				est l'ensemble de toutes les combinaisons linéaires de \( v_1, v_2, \dots, v_n \) et est appelé l'enveloppe linéaire de cet ensemble de vecteurs.
			\end{definition}
			
			
			\begin{definition} \cite{lang2012algebra} \\
				Le sous-espace \( \text{Span}(A) \) est le sous-espace vectoriel de \( V \) engendré par \( A \). L'ensemble \( A \) est appelé un ensemble générateur de \( \text{Span}(A) \). Si \( \text{Span}(A) = V \), on dit que \( A \) engendre \( V \).
			\end{definition}
			
			
			\begin{definition} \cite{lang2012algebra} \\
				Soient \( v_1, v_2, \dots, v_n \) des vecteurs dans un espace vectoriel \( V \). Ils sont linéairement dépendants s'il existe des scalaires \( c_1, c_2, \dots, c_n \), dont au moins un est non nul, tels que :
				\[
				c_1 v_1 + c_2 v_2 + \dots + c_n v_n = 0.
				\]
				Ils sont linéairement indépendants si :
				\[
				c_1 v_1 + \dots + c_n v_n = 0 \implies c_1 = c_2 = \dots = c_n = 0.
				\]
			\end{definition}
			
			
			
			\begin{definition} \cite{lang2012algebra} \\
				Une liste finie de vecteurs \( v_1, \dots, v_n \) dans un espace vectoriel \( V \) est appelée une base de \( V \) si elle est à la fois indépendante et génératrice. Autrement dit, chaque vecteur \( v \in V \) peut être écrit comme une combinaison linéaire :
				\[
				v = a_1 v_1 + \dots + a_n v_n,
				\]
				(puisque \( \text{Span}\{v_1, \dots, v_n\} = V \)) et les coefficients \( a_1, \dots, a_n \) sont uniques. Ces coefficients sont appelés les coordonnées de \( v \) par rapport à la base \( v_1, \dots, v_n \).
				chc\end{definition}
			
			\begin{remark} \cite{lang2012algebra}\\
				Le nombre minimal de vecteurs qui engendrent un espace vectoriel est constant, et il est donné par la dimension de cet espace.
			\end{remark}
			
			\begin{definition} \cite{lang2012algebra} \\
				Supposons que \( V \) soit un espace vectoriel de dimension finie. Si \( V \neq \{0\} \), alors le nombre de vecteurs dans toute base de \( V \) est appelé la dimension de \( V \) et est noté \( \dim V \).
			\end{definition}
			
			
			\begin{lemma} \cite{lang2012algebra} \\
				Supposons que \( V \) soit un espace vectoriel. Si \( \{v_1, \dots, v_n\} \) engendre \( V \), alors \( \dim V \leq n \). Si \( \{w_1, \dots, w_m\} \) est une famille linéairement indépendante, alors \( m \leq \dim V \).
			\end{lemma}
			
			\begin{lemma} \cite{lang2012algebra} \\
				Supposons que \( \{v_1, \dots, v_n\} \) soit une famille linéairement indépendante et que \( v \) soit un vecteur. Alors, \( \{v, v_1, \dots, v_n\} \) est linéairement indépendante si et seulement si \( v \notin \mathrm{Span}\{v_1, \dots, v_n\} \).
			\end{lemma}
			
			
			\begin{theorem} \cite{lang2012algebra} \\
				Supposons que \( V \) soit un espace vectoriel de dimension \( n \). Alors, les affirmations suivantes sont équivalentes :
				\begin{enumerate}[label=(\alph*)]
					\item \( \{v_1, \ldots, v_n\} \) engendre \( V \).
					\item \( \{v_1, \ldots, v_n\} \) est linéairement indépendante.
					\item \( \{v_1, \ldots, v_n\} \) est une base de \( V \).
				\end{enumerate}
			\end{theorem}
			
			
			\begin{theorem} \cite{lang2012algebra} \\
				Tout espace vectoriel non nul et de dimension finie possède une base.
			\end{theorem}
			
			\begin{lemma} \cite{lang2012algebra} \\
				Tout ensemble de vecteurs linéairement indépendants dans un espace vectoriel de dimension finie \( V \) peut être étendu en une base de \( V \).
			\end{lemma}
			
			\begin{theorem} \cite{lang2012algebra} \\
				Soit \( V \) et \( W \) des espaces vectoriels de dimension finie. Alors \( V \cong W \) si et seulement si \( \dim V = \dim W \).
			\end{theorem}
			
			
			\begin{definition} \cite{lang2012algebra}\\
				Une projection est un opérateur linéaire \( p : V \to V \) définie sur un espace vectoriel \( V \) tel que :
				\[
				p \circ p = p.
				\]
			\end{definition}
			
			
			\begin{proposition} \cite{lang2012algebra}\\
				Soit \( p : V \to V \) une projection.
				\begin{enumerate}[label=\roman*)]
					\item \( \mathrm{Im}(p) = \{v \in V \mid p(v) = v\} \).
					\item L'espace \( V \) se décompose en une somme directe :
					\[
					V = \mathrm{Im}(p) \oplus \ker(p).
					\]
				\end{enumerate}
			\end{proposition}
			
			
			\textbf{\underline{Preuve :}}\\
			
			\begin{enumerate}[label=\roman*)]
				\item si \( v \in \mathrm{Im}(p) \), alors il existe \( u \in V \) tel que \( v = p(u) \). Appliquons \( p \) à \( v \) :
				\[
				p(v) = p(p(u)) = p(u) = v.
				\]
				\item  Soit	\(v = v_1 + v_2 \in V \). Définissons \( v_1 = p(v) \) et \( v_2 = v - p(v) \). Alors :
				\[
				p(v_1) = p(p(v)) = p(v) = v_1,
				\]
				ce qui implique \( v_1 \in \mathrm{Im}(p) \). De plus :
				\[
				p(v_2) = p(v - p(v)) = p(v) - P(P(v)) = p(v) - p(v) = 0,
				\]
				ce qui implique \( v_2 \in \ker(p) \). Ainsi, tout vecteur \( v \in V \) s'écrit comme :
				\[
				v = v_1 + v_2, \quad \text{avec } v_1 \in \mathrm{Im}(p) \text{ et } v_2 \in \ker(p).
				\]
				
				si \( v \in \ker(p) \cap \mathrm{Im}(p) \), alors :
				\[
				p(v) = 0 \quad \text{et} \quad p(v) = v.
				\]
				Il vient que \( v = 0 \). Ainsi, \( \ker(p) \cap \mathrm{Im}(p) = \{0\} \).\\
				Donc \(V = \mathrm{Im}(p) \oplus \ker(p).\)
				
			\end{enumerate}
			
			
			\begin{propriety} \cite{lang2012algebra}\\
				Un projecteur \( p \) décompose l'espace vectoriel \( V \) en deux sous-espaces comme suit :
				\[
				V = \mathrm{Im}(p) \oplus \ker(p).
				\]
			\end{propriety}
			
			
			
			\subsection{Applications linéaires}
			Soient $\mathbb{K}$ le corps $\mathbb{R}$ ou $\mathbb{C}$, $E$ et $F$ deux espaces vectoriels sur $\mathbb{K}$.
			
			\begin{definition} \cite{lang2012algebra}\\
				Supposons que \( V \) et \( W \) sont des espaces vectoriels sur le même corps \(\mathbb{K}\). Une application \( T : V \to W \) est dite linéaire si :
				
				\begin{enumerate}[label=\roman*)]
					\item \textbf{Additivité} :\\
					\( T(v + w) = T(v) + T(w) \) pour tout \( v, w \in V \);
					\item \textbf{Homogénéité}: \\
					\( T(cv) = cT(v) \) pour tout \( c \in \mathbb{K} \) et \( v \in V \).
				\end{enumerate}
			\end{definition}
			
			
			\begin{definition} \cite{lang2012algebra}\\
				Une application linéaire de $E$ dans $E$ s'appelle aussi un endomorphisme de $E$.
			\end{definition}
			
			\begin{mynotation}
				On note $\mathcal{L}(E, F)$ l'ensemble des applications linéaires de $E$ dans $F$, et $\mathcal{L}(E)$ si $E = F$. 
			\end{mynotation}
			
			\begin{definition} \cite{axler2024linear}\\
				Soit \( V \) un $\mathbb{K}$-espace vectoriel. L'espace vectoriel dual de \( V \), noté \( V' \), est l'ensemble des applications linéaires de \( V \) dans \( \mathbb{K}\). Autrement dit :
				\[
				V' = \{ \phi : V \to \mathbb{K} \mid \phi \text{ est linéaire} \}.
				\]
				Les éléments de \( V' \) sont appelés des formes linéaires sur \( V \).
			\end{definition}
			
			
			\begin{definition} \cite{savage2018linear}\\
				Soient \( V \) et \( W \) deux $\mathbb{K}$-espaces vectoriels de dimension finie , avec 
				\( \dim V = n \) et \( \dim W = m \), et soit \( T : V \to W \) une application linéaire.Supposons 
				que \( B = \{v_1, \dots, v_n\} \) soit une base ordonnée de \( V \), et \( D = \{w_1, \dots, w_m\} \) 
				une base ordonnée de \( W \). \\
				Pour chaque vecteur \( v_j \) de la base de \( V \), l’image \( T(v_j) \) peut s’écrire comme une 
				combinaison linéaire des vecteurs de la base de \( W \) :	
				\[
				T(v_j) = \sum_{i=1}^{m} a_{ij} w_i, \quad \text{pour} \ j = 1, \dots, n.
				\]
				Les scalaires \( a_{ij} \) sont les coordonnées de \( T(v_j) \) dans la base \( D \) de \( W \). La 
				matrice formée par les éléments \( a_{ij} \), notée \( [T]^D_B \), est appelée la matrice de l'application linéaire \( T \) relative aux bases \( B \) de \( V \) et \( D \) de \( W \). C'est une matrice de taille \( m \times n \).
			\end{definition}
			
			\begin{mynotation}
				L'ensemble des matrices de taille \( m \times n \) à coefficients dans \( \mathbb{K} \) 
				est noté \( M_{m,n}(\mathbb{K}) \). Si \( m = n \), l'ensemble des matrices carrées d'ordre \( n \) 
				sur \( F \) est noté \( M_n(\mathbb{K}) \). 
			\end{mynotation}
			
			
			\begin{definition} \cite{savage2018linear}\\
				Soit \( A \in M_n(\mathbb{K}) \), où \( n \geq 2 \). On note \( A_{i,j} \) la matrice obtenue en supprimant la \( i \)-ème ligne et la \( j \)-ème colonne de \( A \). L’élément situé à la position \( (i,j) \) dans la matrice \( A \) est noté \( a_{i,j} \).
				La fonction déterminant, notée \( \text{det} : M_n(\mathbb{K}) \to \mathbb{K} \), est définie de manière récursive selon la taille \( n \) de la matrice :
				\begin{enumerate} [label=(\alph*)]
					\item Pour \( n = 1 \), on définit :
					\[
					\text{det}(A) := a_{1,1}.
					\]
					\item  Pour \( n \geq 2 \), le déterminant de \( A \) est donné par le développement par cofacteurs le 
					long de la première colonne :
					\[
					\text{det}(A) := \sum_{i=1}^{n} (-1)^{i+1} \, \text{det}(A_{i,1}) \cdot a_{i,1}.
					\]
					Le scalaire \( (-1)^{i+j} \text{det}(A_{i,j}) \) est appelé le cofacteur de l’élément \( a_{i,j} \). 
					La valeur \( \text{det}(A) \) est appelée le déterminant de la matrice \( A \).	
				\end{enumerate}	
			\end{definition}
			
			\begin{definition} \cite{serre1971representation} \\
				Soit \( V \) un espace vectoriel ayant une base \( (e_j) \) de \( n \) éléments, et soit \( \varphi \) une application linéaire de \( V \) dans lui-même, avec une matrice \( (a_{ij}) \) relative à cette base. La trace de \( \varphi \), est :
				\[
				\text{Tr}(\varphi) = \sum_{j=1}^{n} a_{jj}.
				\]
			\end{definition}
			
			\begin{propriety} \cite{harville1997trace}
				\begin{enumerate}
					\item  \(\operatorname{Tr}(A+B)=\operatorname{Tr}(A)+\operatorname{Tr}(B)\)
					\item \( \operatorname{Tr}(A^{t})=\operatorname{Tr}(A)\)
					\item \(\operatorname{Tr}(I_{n})=n\)
					
					\item \( \operatorname{Tr}(A_1 A_2 \cdots A_k) = \operatorname{Tr}(A_{j+1} \cdots A_k A_1 \cdots A_j) \),
					(\(j=1,...,k-1\)).
				\end{enumerate}
			\end{propriety}
			
			\section{Notions de catégories, foncteurs, limites projectives et limites inductives}
			
			\subsection{Catégories}
			
			\begin{definition} \cite{maclane1971categories} \\
				Une catégorie $\mathcal{C}$ consiste en les données suivantes :
				\begin{enumerate} [label=\roman*)]
					\item Une classe $|\mathcal{C}|$, dont les éléments sont appelés objets de $\mathcal{C}$;
					\item À chaque couple d'objets $(X, Y)$ de $\mathcal{C}$, est associé un ensemble $\mathcal{C}(X, Y)$ (ou $\mathrm{Hom}_{\mathcal{C}}(X, Y)$), dont les éléments sont appelés morphismes (ou flèches) de $X$ dans $Y$;
					\item À chaque triplet $(X, Y, Z)$ d'objets de $\mathcal{C}$, une application (appelée application de composition)
					\[
					\mathcal{C}(X, Y) \times \mathcal{C}(Y, Z) \to \mathcal{C}(X, Z), \quad (f, g) \mapsto g \circ f;
					\]
					\item À chaque objet $X \in \mathcal{C}$, est associé un élément $1_X \in \mathcal{C}(X, X)$ appelé morphisme d'identité de $X$.
				\end{enumerate}
				Ces données vérifient les axiomes suivants :
				\begin{itemize}
					\item \textbf{Associativité de la composition :}
					si $X \xrightarrow{f} Y \xrightarrow{g} Z \xrightarrow{h} W$ sont des morphismes dans $\mathcal{C}$, alors on a
					\[
					h \circ (g \circ f) = (h \circ g) \circ f.
					\]
					\item \textbf{Neutralité de l'identité :}
					pour tous $X, Y \in |\mathcal{C}|$, et pour tout $f \in \mathcal{C}(X, Y)$, on a
					\[
					f \circ 1_{X} = f,
					\]
					\[
					\xymatrix{
						X \ar[rr]^{1_{X}} \ar[dr]_{f} & & X \ar[dl]^{f} \\
						& Y &
					}
					\]
					et
					\[
					1_{Y} \circ f = f,
					\]
					\[
					\xymatrix{
						X \ar[rr]^{f} \ar[dr]_{f} & & Y \ar[dl]^{1_{Y}} \\
						& Y &
					}
					\]
				\end{itemize}
			\end{definition}
			
			\begin{remark} \cite{maclane1971categories} 
				\begin{enumerate} 
					\item Le morphisme unité d'un objet $X$ d'une catégorie est unique.
					\item Si la  catégorie $\mathcal{C}$ est un ensemble, on dit que c'est une petite catégorie.
				\end{enumerate}
			\end{remark}
			
			\begin{myexample}
				La catégorie des groupes, notée \textbf{Grp}, consiste en les données suivantes :
				\begin{enumerate}[label=\roman*)]
					\item \textbf{Objets} :\\
					Les objets sont des groupes.
					\item \textbf{Morphismes} : \\
					Les morphismes sont des homomorphismes de groupes.
					\item \textbf{Composition} : \\
					La composition de deux homomorphismes de groupes est un autre homomorphisme de groupes.
					\item \textbf{Identité} :\\
					Pour chaque groupe \( G \), le morphisme d'identité est la fonction qui envoie chaque élément sur lui-même.
				\end{enumerate}
			\end{myexample}
			
			\textbf{Vérification des axiomes} :
			\begin{itemize}
				\item \textbf{Associativité} :\\
				Pour tout \( f: G_1 \to G_2 \), \( g: G_2 \to G_3 \), et \( h: G_3 \to G_4 \), on a :  
				\[
				(g \circ f) \circ h = g \circ (f \circ h).
				\]
				\item \textbf{Identité} :\\
				Pour tout homomorphisme \( f \) entre deux groupes, on a :  
				\[
				f \circ id_{G} = f \quad \text{et} \quad id_{H} \circ f = f.
				\]
			\end{itemize}
			
			
			\subsection{Foncteurs}
			Soient $\mathcal{C}$ et $\mathcal{D}$ deux catégories.
			\begin{definition} \cite{maclane1971categories} \\
				Un \textit{foncteur contravariant} est une loi de passage d'une catégorie $\mathcal{C}$ à une catégorie $\mathcal{D}$, \\
				$F : \mathcal{C} \to \mathcal{D}$, qui :
				\begin{enumerate} [label=\roman*)]
					\item à tout objet $C$ de $\mathcal{C}$ associe un objet $F(C)$ de $\mathcal{D}$,
					\item à tout morphisme $X \xrightarrow{f} Y$ de $\mathcal{C}$ associe un morphisme $F(Y) \xrightarrow{F(f)} F(X)$ de $\mathcal{D}$ satisfaisant :
					\begin{enumerate}
						\item pour tout objet $X$ de $\mathcal{C}$, on a $F(1_X) = 1_{F(X)}$ ;
						\item pour tous morphismes $X \xrightarrow{f} Y \xrightarrow{g} Z$ de $\mathcal{C}$, on a $F(g \circ f) = F(f) \circ F(g)$.
					\end{enumerate}
				\end{enumerate}
			\end{definition}
			
			\begin{definition}  \cite{cartan1999homological}\\
				Une \textbf{suite exacte courte} est une suite de morphismes dans une catégorie abélienne :
				\[
				0 \to A \xrightarrow{f} B \xrightarrow{g} C \to 0
				\]
				qui satisfait les conditions suivantes :
				\begin{itemize}
					\item \textbf{Injectivité de \( f \)} : \( f \) est injective, c'est-à-dire que \( \text{Im}(f) = \ker(g) \),
					\item \textbf{Surjectivité de \( g \)} : \( g \) est surjective, c'est-à-dire que \( \text{Im}(g) = C \),
					\item \textbf{Exactitude au milieu} : \( \text{Im}(f) = \ker(g) \).
				\end{itemize}
			\end{definition}
			
			\begin{definition} \cite{cartan1999homological}\\
				Soit \( \mathcal{C} \) et \( \mathcal{D} \) deux catégories. Un foncteur \( F : \mathcal{C} \to \mathcal{D} \) est dit \textbf{contravariant exact} si, pour toute suite courte exacte dans \( \mathcal{C} \):
				
				\[
				0 \to A \xrightarrow{f} B \xrightarrow{g} C \to 0
				\]
				
				la suite
				
				\[
				0 \to F(C) \xrightarrow{F(g)} F(B) \xrightarrow{F(f)} F(A) \to 0
				\]
				
				est également exacte dans \( \mathcal{D} \).
			\end{definition}
			
			
			
			
			\subsection{ Limites inductives}
			
			\begin{definition} \cite {maclane1971categories}\\
				Une propriété universelle (PU) est un énoncé sur les objets mathematiques qui stipule que sous certaines conditions, il existe un unique morphisme qui satisfait certaines propriétés.
			\end{definition}
			
			\begin{definition} \cite{ribes-zalesskii} \\
				Un ensemble partiellement ordonné est un couple \((I, \leq)\) où \(I\) est un ensemble non vide et \(\leq\) est une relation binaire sur \(I\) vérifiant les propriétés suivantes pour tous \(a, b, c \in I\) :
				\begin{enumerate}[label=\roman*)]
					\item \textbf{Réflexivité} : \(a \leq a\);
					\item \textbf{Anti-symétrie} : \text{si } \(a \leq b\) \text{ et } \(b \leq a\), \text{ alors } \(a = b\);
					\item \textbf{Transitivité} : \text{si } \(a \leq b\) \text{ et } \(b \leq c\), \text{ alors } \(a \leq c\).
				\end{enumerate}
			\end{definition}
			
			
			\begin{definition} \cite{ribes-zalesskii} \\
				Un ensemble $(I,\leq)$ est dit ordonné filtrant si $(I,\leq)$ est un ensemble  partiellement ordonné et si pour tous $i,j \in I$ , il existe $k \in I$ vérifiant $i \leq k$ et $j \leq k$.
			\end{definition}
			
			
			\begin{myexample}
				$(\mathbb{N}, \leq)$ est ordonné filtrant.
			\end{myexample}
			
			
			\begin{definition} \cite{ribes-zalesskii} \\
				Soit $(I, \leq)$ un ensemble ordonné filtrant. Un système inductif de groupes sur $I$ est la donnée d'un couple $(X_{i},\phi_{ij})_{i,j \in I}$ où $X_{i}$ sont les groupes et les $\phi_{ij} : X_{i} \to X_{j}$ ($i \leq j$) les homomorphismes de groupes  , vérifiant :
				\begin{enumerate} [label=\roman*)]
					\item $ \forall i \in I, \ \phi_{ii} = \mathrm{Id}_{X_{i}} $;
					\item $\forall (i, j, k) \in I^{3}, \ i \leq j \leq k \Rightarrow \phi_{jk} \circ \phi_{ij} = \phi_{ik} .$\\
					Ce qui se traduit par le diagramme commutatif suivant:
					\[
					\xymatrix{
						X_{i} \ar[rr]^{\phi_{ij}} \ar[dr]_{\phi_{ik}} & & X_{j} \ar[dl]^{\phi_{jk}} \\
						& X_{k} &
					}
					\]
					
				\end{enumerate}
			\end{definition}
			
			
			\begin{definition} \cite{ribes-zalesskii} \\
				Soient $X$ un groupe et $(X_i, \phi_{ij})$ un système inductif de groupes.La famille $(X,\phi_i :X_i \rightarrow X)$ est dite compatible avec $(X_i, \phi_{ij})$ si pour tous $i,j \in I$ tels que $i \leq j$, on a $\phi_i = \phi_j \circ \phi_{ij}$.Ce qui s'illustré par le diagramme commutatif suivant:
				\[
				\xymatrix{
					X_{i} \ar[r]^{\phi_{ij}} \ar[dr]_{\phi_i} & X_{j} \ar[d]^{\phi_j} \\
					& X
				}
				\]
				
			\end{definition}
			
			\begin{definition} \cite{ribes-zalesskii} \\
				Soit $(X_i, \phi_{ij})$ un système inductif de groupes.
				La limite inductive ou limite directe, lorsqu'elle existe,est une famille compatibile $(X,\phi_i :X_i \rightarrow X)$ avec $(X_i, \phi_{ij})$ vérifiant la PU : pour toute autre famille  $(X,\psi_i)_{i \in I}$ compatible avec $(X_i, \phi_{ij})$, il existe un unique homomorphisme de groupes $u : X \to Y$ telle que le diagramme :
				
				\begin{tikzpicture}[auto]
					
					% Nodes
					\node (X_i) at (0, 2) {$X_i$};
					\node (X_j) at (4, 2) {$X_j$};
					\node (X) at (2, 0) {$X$};
					\node (Y) at (2, -2) {$Y$};
					
					% Arrows
					\draw[->] (X_i) -- node[above] {$\phi_{ij}$} (X_j);
					\draw[->] (X_i) -- node[right] {$\phi_i$} (X);
					\draw[->] (X_j) -- node[left] {$\phi_j$} (X);
					\draw[->] (X_i) -- node[below left] {$\psi_i$} (Y);
					\draw[->] (X_j) -- node[below right] {$\psi_j$} (Y);
					\draw[->, dashed] (X) -- node[right] {$u$} (Y);
					
				\end{tikzpicture}
				soit commutatif pour tous $i \leq j$.
				
			\end{definition}
			
			\begin{proposition} \cite{ribes-zalesskii} \\
				La limite inductive lorsqu'elle existe est unique à isomorphisme unique près.
			\end{proposition}
			
			
			\textbf{\underline {Preuve:} }
			\textbf{\underline{Existence:}}\\
			Soit \((I, \leq)\) un ensemble ordonné filtrant. Considérons un système inductif \(\{X_i, \varphi_{ij}\}_{i \in I}\), où chaque \(X_i\) est un groupe et \(\varphi_{ij} : X_i \to X_j\) est un homomorphisme  de groupes pour \(i \leq j\) vérifiant :
			
			\[
			\begin{aligned}
				&\varphi_{ii} = \text{id}_{X_i}, \\
				&\varphi_{ik} = \varphi_{ij} \circ \varphi_{jk} \quad \text{pour tout} \quad i \leq j \leq k.
			\end{aligned}
			\]
			Définissons l'ensemble \(X\) par :
			\[
			X = \{(x_i)_{i \in I} \mid \forall i \leq j, \, \varphi_{ij}(x_j) = x_i\}.
			\]
			Nous avons alors \(X \subseteq \prod_{i \in I} X_i\).
			Pour chaque \(i \in I\), définissons la projection \(\pi_i : X \to X_i\) par 
			
			\[
			\pi_i((x_j)_{j \in I}) = x_i.
			\]
			Cela satisfait les conditions de compatibilité : \(\varphi_{ij} \circ \pi_j = \pi_i\).
			
			\textbf{\underline{Propriété Universelle:}}\\
			Soit \(Y\) un autre objet avec des morphismes \(\psi_i : Y \to X_i\) tels que \(\varphi_{ij} \circ \psi_j = \psi_i\) pour tout \(i \leq j\). Définissons un homomorphisme \(\psi : Y \to X\) par :
			
			\[
			\psi(y) = (\psi_i(y))_{i \in I}.
			\]
			Cette famille respecte les conditions de compatibilité, donc il existe un homomorphisme \(\psi : Y \to X\) tel que \(\pi_i \circ \psi = \psi_i\).
			
			\textbf{\underline{Unicité:}}\\
			Supposons que \(X\) et \(X'\) soient deux limites inductives du même système \(\{X_i, \varphi_{ij}\}\).
			Pour chaque \(i \in I\), considérons les homomorphismes de projection \(\pi_i : X \to X_i\) et \(\pi'_i : X' \to X_i\).
			2. Par la propriété universelle de \(X\), il existe un homomorphisme unique \(f : X' \to X\) tel que \(\pi_i \circ f = \pi'_i\).
			Par la propriété universelle de \(X'\), il existe un homomorphisme unique \(g : X \to X'\) tel que \(\pi'_i \circ g = \pi_i\).\\
			En considérant les compositions \(g \circ f : X' \to X'\) et \(f \circ g : X \to X\), nous trouvons :
			
			\[
			\begin{aligned}
				\pi'_i \circ (g \circ f) &= \pi_i \circ f = \pi'_i, \\
				\pi_i \circ (f \circ g) &= \pi'_i \circ g = \pi_i.
			\end{aligned}
			\]
			D'où \(g \circ f = \text{id}_{X'}\) et \(f \circ g = \text{id}_X\).
			Ainsi, \(f\) et \(g\) sont des isomorphismes mutuellement inverses, prouvant que \(X\) et \(X'\) sont isomorphes.\\
			Donc la limite inductive est unique à isomorphisme près, car toute limite inductive d'un même système est isomorphe.
			
			
			\begin{mynotation}
				La limite inductive $(X,\phi_i)_{i \in I}$ d'un système inductif $(X_i, \phi_{ij})_{j \in I}$ est notée $X = \varinjlim X_i$.
			\end{mynotation}
			
			\begin{example} \
				\begin{enumerate}
					
					\item Si l'ensemble filtrant \( I \) possède un plus grand élément $k$ (par exemple si \( I \) est fini et non vide), la limite inductive de tout système inductif $(X_{i},f_{ij})_{i,j \in I}$ est égale à $X_{k}$, c'est-à-dire \( \varinjlim_{i \in I} X_i = X_k \).
					\item La limite inductive du système indexé par l'ensemble vide est l'objet initial.
					\item Soit \( E \) un ensemble non vide et \( (E_n)_{n \in \mathbb{N}} \) une suite croissante de sous-ensembles de \( E \) avec les injections canoniques \( i_n: E_n \to E_{n+1} \). La limite inductive de la suite \( (E_n)_{n \in \mathbb{N}} \) s'identifie à la réunion de ces ensembles, c'est-à-dire
					
					\[
					\varinjlim_{n \in \mathbb{N}} E_n = \bigcup_{n \in \mathbb{N}} E_n.
					\]
				\end{enumerate}
			\end{example}
			
			
			\subsection{Limites projectives}
			Soit $(I,\leq)$ est dit ordonné filtrant.
			\begin{definition} \cite{maclane1971categories}\\
				Un système projectif de groupes sur $(I,\leq)$ est un couple $(X_{i},\phi_{ij})_{i,j \in I}$ où les $X_{i}$ sont les groupes et les $\phi_{ij}:X_{j} \rightarrow X_{i}$  ($i \leq j$) sont les homomorphismes de groupes vérifiant :
				\begin{enumerate}[label=\roman*)]
					\item $\phi_{ii} = \text{Id}_{X_{i}}$ pour tout $i \in I$;
					\item pour tout $(i,j,k) \in I^{3}$ tels que $i \leq j \leq k$, on a $\phi_{ik} = \phi_{ij} \circ \phi_{jk}$.
				\end{enumerate}
				Autrement dit, le diagramme
				\[
				\xymatrix{
					X_{k} \ar[rr]^{\phi_{jk}} \ar[dr]_{\phi_{ik}} & & X_{j} \ar[dl]^{\phi_{ij}} \\
					& X_{i} &
				}
				\]
				est commutatif.
			\end{definition}
			
			
			\begin{definition}  \cite{maclane1971categories}\\
				Soient $X$ un groupe et $(X_{i}, \phi_{ij})_{i,j \in I}$ un système projectif de groupes.La famille de homomorphismes $(\phi_{i} : X \rightarrow X_{i})_{i \in I}$ qu'on note $( X ,\phi_{i})$ est dite compatible avec le système projectif $(X_{i}, \phi_{ij})_{i,j \in I}$ si pour tous $i,j \in I$ tels que $i \leq j$, on a : $\phi_{ij} \circ \phi_{j} = \phi_{i}$.
				Ce qui se traduit par le diagramme commutatif suivant:
				\[
				\xymatrix{
					X \ar[rr]^{\phi_{j}} \ar[dr]_{\phi_{i}} & & X_{j} \ar[dl]^{\phi_{ij}} \\
					& X_{i} &
				}
				\]
				
			\end{definition}
			
			\begin{definition}  \cite{maclane1971categories}\\
				Soit $(X_{i},\phi_{ij})_{i,j \in I}$ un système projectif de groupes.La limite projective ou limite inverse du système projectif $(X_{i},\phi_{ij})_{i,j \in I}$  est une famille $(X,(\phi_{i})_{i \in I})$ de homomorphismes compatibles avec $(X_{i},\phi_{ij})_{i,j \in I}$ vérifiant la PU suivante: \\
				Si $(\psi _{i} : Y \rightarrow X_{i})_{i \in I}$ ($ Y \in |\mathcal{C}|$) est une famille de morphismes compatibles, alors il existe un unique morphisme $\psi : Y \rightarrow X $ tel que le diagramme suivant commute pour tous $i \leq j$:
				
				\begin{tikzpicture}[auto]
					
					% Nodes
					\node (Y) at (0, 2) {$Y$};
					\node (X) at (0, 0) {$X$};
					\node (X_i) at (2, -2) {$X_i$};
					\node (X_j) at (-2, -2) {$X_j$};
					
					% Arrows
					\draw[->, dashed] (Y) -- node[right] {$\psi$} (X);  % Dashed arrow
					\draw[->] (Y) -- node[left] {$\psi_j$} (X_j);
					\draw[->] (Y) -- node[right] {$\psi_i$} (X_i);
					\draw[->] (X) -- node[right] {$\phi_j$} (X_j);
					\draw[->] (X) -- node[left] {$\phi_i$} (X_i);
					\draw[->] (X_j) -- node[below] {$\phi_{ij}$} (X_i);
					
				\end{tikzpicture}
				
				\[ \psi_{i} = \phi_{i} \circ \psi \]
				\[ \psi_{j} = \phi_{j} \circ \psi \]
				
			\end{definition}
			
			\begin{proposition}  \cite{maclane1971categories}\\
				Si une limite projective d'un système projectif existe, elle est unique à isomorphisme près.
			\end{proposition}
			
			\textbf{\underline{Existence:}}\\
			Soit \((I, \leq)\) un ensemble ordonné filtrant.\\
			Considérons un système projectif \(\{X_i, \varphi_{ij}\}_{i, j \in I}\), où \(X_i\) sont des groupes finis, et \(\varphi_{ij} : X_j \to X_i\) sont des homomorphismes pour \(i \leq j\) vérifiant la compatibilité suivante :
			\[
			\begin{aligned}
				&\varphi_{ii} = \text{id}_{X_i}, \\
				&\varphi_{ik} = \varphi_{ij} \circ \varphi_{jk} \quad \text{pour tout} \quad i \leq j \leq k.
			\end{aligned}
			\]
			Nous cherchons à construire un objet \(X = \varprojlim X_i\) et des homomorphismes \(\pi_i : X \to X_i\) satisfaisant les conditions du système projectif.\\
			Définissons l'ensemble \(X\) par :
			
			\[
			X = \{(x_i)_{i \in I} \mid \forall i \leq j, \, \varphi_{ij}(x_j) = x_i\}.
			\]
			Cet ensemble est inclus dans \(\prod_{i \in I} X_i\) et vérifie la condition :\\
			\[
			\varphi_{ij}(x_j) = x_i \quad \text{pour tout} \quad i \leq j.
			\]
			Pour chaque \(i \in I\), définissons la projection \(\pi_i : X \to X_i\) par 
			\[
			\pi_i((x_j)_{j \in I}) = x_i.
			\]
			Cela définit un objet \(X\) avec des homomorphismes \(\pi_i\) satisfaisant \(\varphi_{ij} \circ \pi_j = \pi_i\) illustré par le diagramme suivant:
			
			\[
			\xymatrix{
				X \ar[rr]^{\pi_{j}} \ar[dr]_{\pi_i} & & X_{j} \ar[dl]^{\varphi_{ij}} \\
				& X_{i} &
			}
			\]
			
			
			\textbf{\underline{Propriété Universelle:}}\\
			Soit \(Y\) un autre objet avec des morphismes \(\psi_i : Y \to X_i\) satisfaisant \(\varphi_{ij} \circ \psi_j = \psi_i\) pour tout \(i \leq j\). Définissons un morphisme \(\psi : Y \to X\) par :
			\[
			\psi(y) = (\psi_i(y))_{i \in I}.
			\]
			La famille \((\psi_i(y))\) est bien dans \(X\) car elle respecte les conditions de compatibilité. Ainsi, il existe un morphisme \(\psi : Y \to X\) tel que \(\pi_i \circ \psi = \psi_i\).
			
			\textbf{\underline{Unicité:}}\\
			Supposons que \(X\) et \(X'\) soient deux limites projectives du même système projectif \(\{X_i, \varphi_{ij}\}\).
			
			\begin{itemize}
				\item Soit \(\pi_i : X \to X_i\) et \(\pi'_i : X' \to X_i\) les morphismes de projection correspondants pour chaque \(i \in I\), satisfaisant les conditions de commutativité \(\varphi_{ij} \circ \pi_j = \pi_i\) et \(\varphi_{ij} \circ \pi'_j = \pi'_i\) pour tous \(i \leq j\).
				\item Par la propriété universelle de la limite projective \(X\), il existe un morphisme unique \(f : X' \to X\) tel que pour tout \(i \in I\), \(\pi_i \circ f = \pi'_i\).
				\item De même, par la propriété universelle de \(X'\), il existe un morphisme unique \(g : X \to X'\) tel que pour tout \(i \in I\), \(\pi'_i \circ g = \pi_i\).
			\end{itemize}
			
			Considérons maintenant la composition \(g \circ f : X' \to X'\). C'est un morphisme de \(X'\) vers lui-même, et pour chaque \(i\), nous avons :
			\[
			\begin{aligned}
				\pi'_i \circ (g \circ f) &= (\pi'_i \circ g) \circ f \\
				&= \pi_i \circ f \\
				&= \pi'_i.
			\end{aligned}
			\]
			Par la partie "unicité" de la propriété universelle de \(X'\), le morphisme identité \(\text{id}_{X'} : X' \to X'\) satisfait cette propriété. Ainsi, \(g \circ f = \text{id}_{X'}\).
			De même, \(f \circ g : X \to X\) est un morphisme de \(X\) vers lui-même, et pour chaque \(i\), nous avons :
			
			\[
			\begin{aligned}
				\pi_i \circ (f \circ g) &= (\pi_i \circ f) \circ g \\
				&= \pi'_i \circ g \\
				&= \pi_i.
			\end{aligned}
			\]
			Encore une fois, par la partie "unicité" de la propriété universelle de \(X\), nous concluons que \(f \circ g = \text{id}_X\).
			Ainsi, \(f\) et \(g\) sont des isomorphismes mutuellement inverses, ce qui montre que \(X\) et \(X'\) sont isomorphes.
			Donc la limite projective est unique à isomorphisme près, car toute deux limites projectives d'un même système projectif sont isomorphes.
			
			
			\begin{mynotation}
				Une telle limite est notée $\varprojlim_{I} X_i$ ou  $\varprojlim_{i \in I} X_i$.
			\end{mynotation}
			
			
			\begin{definition}  \cite{maclane1971categories}\\
				La limite projective d'un système projectif de groupe finis est appelé groupe profini.
			\end{definition}
			
			
			\begin{example}\
				\begin{enumerate} 
					\item Tout groupe fini est profini. Ainsi, \( \mathbb{Z}/7\mathbb{Z} \) est un groupe profini.
					\item La limite projective du système projectif \((\mathbb{Z}/n\mathbb{Z}, \varphi_{mn})_{N\setminus \{0\}}\) où \(\varphi_{mn}\) désigne, pour \(n\) divisant \(m\), la surjection définie par : 
					\[
					\varphi_{mn} : \mathbb{Z}/m\mathbb{Z} \longrightarrow \mathbb{Z}/n\mathbb{Z}
					\]
					\[
					\overline{x} \longmapsto \tilde{x}
					\]
					où \(\overline{x} \)	et \(\tilde{x}\) sont respectivement les classes de l’entier \(x\) modulo \(m\mathbb{Z}\) et modulo \(n\mathbb{Z}\), est un groupe profini. On le note \(\widehat{\mathbb{Z}}\) et on l’appelle le complété profini de \(\mathbb{Z}\).
				\end{enumerate}
			\end{example}
			
			
			\section{Complété Profini d'un Groupe }
			
			\begin{definition} \cite{herfort2012profinite}\\
				Un complété profini d’un groupe abstrait \( G \) est la limite projective notée $\widehat{G}$ du système projectif \( (G/N)_{N \in \mathscr{N}} \) de groupes finis, où \( \mathscr{N} \) est la collection de tous les sous-groupes normaux d’indices finis de \( G \), c’est-à-dire
				\[
				\hat{G} = \varprojlim_{N \in \mathscr{N}} G/N.
				\]
			\end{definition}
			
			\begin{proposition} \cite{herfort2012profinite}\\
				Le complété profini d’un groupe est unique à isomorphisme près, c’est-à-dire si $(\widehat{G_1}, j_1)$ et $(\widehat{G_2}, j_2)$ sont deux complétés de $G$, alors il existe un isomorphisme $\widehat{\alpha} : \widehat{G_1} \to \widehat{G_2}$ tel que $\widehat{\alpha} j_1 = j_2$.
			\end{proposition}
			
			\begin{proposition} \cite{herfort2012profinite}\\
				\[
				\hat{G} = \varprojlim_{N \in \mathscr{N}} G/N \quad \text{est un sous-groupe du produit direct} \quad \prod_{N \in \mathscr{N}} G/N.
				\]
			\end{proposition}
			
			
			\begin{propriety}  \cite{herfort2012profinite}\\
				Lorsque $G$ est doté de la topologie profinie, alors on a un homomorphisme continu $j : G \to \widehat{G}$ vérifiant la propriété universelle suivante : pour tout $\theta : G \to H$ un homomorphisme continu dans un groupe discret $H$, il existe un unique homomorphisme continu $\widehat{\theta} : \widehat{G} \to H$ tel que $\theta = \widehat{\theta} j$. On dit que le couple $(\widehat{G}, j)$ est le complété profini de $G$.
			\end{propriety}
			On a les propriétés suivantes des complétés profinis.
			
			\begin{proposition} \cite{ribes-zalesskii}\\
				Soit $(\widehat{G}, j)$ le complété profini de $G$. Alors :
				\begin{itemize}
					\item[(a)] $j(G)$ est dense dans $\widehat{G}$.
					\item[(b)] $\ker j = \bigcap_{K \in \mathcal{N}} K$.
				\end{itemize}	
			\end{proposition}
			
			
			\begin{definition} \cite{ribes-zalesskii}\\
				On dit que les groupes $G_1$ et $G_2$ sont \textit{profiniment équivalents} si leurs complétions profinies sont isomorphes, c'est-à-dire si $\widehat{G_1} \cong \widehat{G_2}$.
			\end{definition}
			
			\begin{theorem} \cite{herfort2012profinite}\\
				Des groupes de type fini avec la même collection de quotients finis ont des complétions profinies isomorphes, autrement dit sont profiniment équivalents.
			\end{theorem}
			
			\section*{Conclusion}
			\addcontentsline{toc}{section}{Conclusion}
			Ce chapitre a établi les fondements théoriques nécessaires à l’étude des représentations linéaires des groupes infinis. Nous avons introduit les notions de groupes, sous-groupes et homomorphismes, suivies des concepts d’espaces vectoriels et d’applications linéaires, essentiels pour appréhender les représentations linéaires. Enfin, les outils de la théorie des catégories, tels que les foncteurs, limites projectives et inductives, ont été présentés afin de fournir une structure générale pour les développements ultérieurs.Ces préliminaires fournissent un cadre rigoureux qui facilitera l’étude des représentations linéaires, des caractères, et des représentations irréductibles dans le contexte des groupes infinis, thème central de ce mémoire.
			
		}
		
		
		
		\chapter{Notion de représentation linéaire de groupes finis}
		
		% Partie avec la taille personnalisée
		{
			\applyfontsize % Application locale de la taille de police 12pt
			
			\section*{Introduction}
			\addcontentsline{toc}{section}{Introduction}
			Les représentations linéaires de groupes finis constituent un outil fondamental en mathématiques, notamment dans les domaines de l'algèbre, de la théorie des nombres, et de la physique mathématique. Elles permettent de traduire des problèmes abstraits en termes de matrices et d'espaces vectoriels, rendant ainsi plus accessibles l'étude des structures et des symétries.Dans ce chapitre, nous allons explorer les bases des représentations linéaires appliquées aux groupes finis. Nous commencerons par définir ce qu'est une représentation linéaire d'un groupe fini \cite{serre1971representation}, en précisant les propriétés essentielles qui s'y rattachent. Ensuite, nous aborderons le concept de caractère, un invariant crucial qui permet de distinguer les représentations et de simplifier leur étude. Enfin, nous introduirons les représentations linéaires irréductibles, qui constituent les briques élémentaires dans la décomposition de toute représentation en composantes simples.
			
			
			
			
			
			\section{Définition et propriétés d'une représentation linéaire d'un groupe fini}
			\subsection{Définitions et exemples}
			\begin{definition} \cite{serre1971representation} \\
				Soit \(\mathbb{K}\) un corps. Une \emph{représentation \(\mathbb{K}\)-linéaire} d'un groupe fini \(G\) est un homomorphisme de groupes \\
				\(\rho : G \rightarrow \mathrm{GL}(V)\) où  \(V\) est un \(\mathbb{K}\)-espace vectoriel et \(\mathrm{GL}(V)\) est le groupe des applications linéaires bijectives de \(V\) sur lui-même.
			\end{definition}
			
			\begin{remark} \cite{serre1971representation} \\
				Si \(V\) est un \(\mathbb{K}\)-espace vectoriel de dimension finie \(n\), on dit que  \(n\) est le degré de la représentation. De plus, en choisissant une base  de \(V\), le groupe \(\mathrm{GL}(V)\) est isomorphe au groupe \\
				\(\mathrm{GL}(n, \mathbb{K}) = \left\{ A \in \mathrm{M}(n, \mathbb{K}) \mid \det(A) \neq 0 \right\}\), où \(\mathrm{GL}(n, \mathbb{K})\) est le groupe des matrices inversibles de taille  \(n \times n\) équipé de la multiplication des matrices à coefficients dans \(\mathbb{K}\), $\det(A)$ pour $A \in \mathrm{M}(n, \mathbb{K})$ désignant le déterminant de la matrice $A$.	
			\end{remark}
			
			\begin{definition} \cite{cheung2018algebre}\\
				Une matrice de permutation de taille \( n \times n \) est une matrice obtenue de la matrice identité \( I_n \) en permutant ses lignes.	
			\end{definition}
			
			
			\begin{example}\
				\begin{enumerate}
					\item La représentation linéaire triviale définie par \(\forall g \in G, \ \rho(g) = \mathrm{Id}_V\).
					\item Représentation linéaire standard du groupe symétrique \( S_3 \) :\\
					Nous définissons une représentation linéaire du groupe symétrique \( S_3 \) sur \( \mathbb{R}^3 \) en associant à chaque permutation \( \sigma \in S_3 \) une matrice de permutation \( \rho(\sigma) \). Les éléments de \( S_3 = \{e, (12), (13), (23), (123), (132)\} \) sont utilisés pour définir la représentation \( \rho : S_3 \to \mathrm{GL}(3, \mathbb{R}) \) comme suit:
					
					\(	\rho(e) = \begin{pmatrix}
						1 & 0 & 0 \\
						0 & 1 & 0 \\
						0 & 0 & 1
					\end{pmatrix} 
					\) 	
					;
					\(
					\rho((12)) = \begin{pmatrix}
						0 & 1 & 0 \\
						1 & 0 & 0 \\
						0 & 0 & 1
					\end{pmatrix}
					\)
					;		
					\(
					\rho((13)) = \begin{pmatrix}
						0 & 0 & 1 \\
						0 & 1 & 0 \\
						1 & 0 & 0
					\end{pmatrix}
					\)
					
					
					\(
					\rho((23)) = \begin{pmatrix}
						1 & 0 & 0 \\
						0 & 0 & 1 \\
						0 & 1 & 0
					\end{pmatrix}
					\)
					;
					\(
					\rho((123)) = \begin{pmatrix}
						0 & 1 & 0 \\
						0 & 0 & 1 \\
						1 & 0 & 0
					\end{pmatrix}
					\)
					;		
					\(
					\rho((132)) = \begin{pmatrix}
						0 & 0 & 1 \\
						1 & 0 & 0 \\
						0 & 1 & 0
					\end{pmatrix}
					\)
					
				\end{enumerate}
			\end{example}
			
			\begin{mynotation}
				Dans la suite, une représentation linéaire \( \rho : G \to \mathrm{GL}(V) \) sera notée  par \( (V,\rho)_{G}\).
			\end{mynotation}
			
			\begin{definition} \cite{renard2009groupes}\\
				Une représentation \( (V,\rho)_{G}\) est dite unitaire si \( \rho \) prend ses valeurs dans le groupe des matrices unitaires.
			\end{definition}
			
			\begin{definition} \cite{serre1971representation} \\
				Soient \( (V, \rho)_{G} \) et \( (W, \psi)_{G} \) deux représentations linéaires. Un opérateur d'entrelacement, ou morphisme de représentations est une application linéaire \( \alpha : V \to W \) telle que \\
				\(\alpha \circ \rho(g) = \psi (g) \circ \alpha \), pour tout \( g \in G \).\\
				On dit que \( \alpha \) est équivariante.\\
				Lorsque \( \alpha : V \to W \) est un isomorphisme, on dit que les représentations \( (V, \varphi)_{G} \) et \( (W, \psi)_{G} \) sont isomorphes.
			\end{definition}
			
			
			\subsection{Sous-représentations}
			Soient \( W \) un sous-espace vectoriel de \( V \) et \( G \) un groupe.
			\begin{definition} \cite{serre1971representation} \\
				On dit que \( W \) est stable (ou invariant) sous l'action de \( G \) ou encore $G$-stable si pour tout \( g \in G \) et tout \( w \in W \), on a \( \rho_g(w) \in W \).
			\end{definition}
			
			\begin{definition} \cite{serre1971representation}\\
				Une sous-représentation de \((V,\rho)_{G} \) est la restriction \( \rho_W : G \rightarrow \mathrm{GL}(W) \) où \( W \) est stable sous \( G \). Elle est définie par :
				
				\[
				\rho_W(g) = \rho(g)|_W, \quad \forall g \in G.
				\]
				
			\end{definition}
			
			
			\begin{proposition} \cite{serre1971representation} \\
				Soit \( \rho : G \rightarrow \mathrm{GL}(V) \) une représentation linéaire sur \( G \) tel que \( W \) soit un sous-espace vectoriel de \( V \) stable sous l'action de \( G \). Alors, l'application restreinte \( \rho_W : G \rightarrow \mathrm{GL}(W) \) définie par \( \rho_W(g) = \rho(g)|_W \) pour tout \( g \in G \) est un homomorphisme de groupes.	
			\end{proposition}
			
			
			\textbf{\underline{Preuve:}}  \\
			Pour démontrer que \( \rho_W \) est un homomorphisme de groupes, il suffit de vérifier que, pour tous \( g_1, g_2 \in G \) et tout \( w \in W \), l'égalité suivante est satisfaite dans \( W \) :  
			\[
			\rho_W(g_1 g_2)(w) = \rho_W(g_1) \circ \rho_W(g_2)(w).
			\]
			Étant donné que \( \rho : G \rightarrow \mathrm{GL}(V) \) est un homomorphisme de groupes, nous avons :  
			\[
			\rho(g_1 g_2)(w) = \rho(g_1) \circ \rho(g_2)(w) \quad \text{dans } V.
			\]
			Montrons que \( \rho(g_1 g_2)(w) = \rho(g_1) \circ \rho(g_2)(w) \) appartient à \( W \).  \\
			Puisque \( W \) est stable sous l'action de \( G \), nous savons que :  
			\begin{itemize}
				\item \( \rho(g_2)(w) \in W \), car \( w \in W \) et \( G \) agit sur \( W \).  
				\item En posant \( w_0 = \rho(g_2)(w) \), on a également \( \rho(g_1)(w_0) \in W \), car \( W \) reste stable par l'action de \( G \).  
			\end{itemize}
			Ainsi, pour tous \( g_1, g_2 \in G \) et tout \( w \in W \), l'égalité suivante est vérifiée dans \( W \) :  
			\[
			\rho_W(g_1 g_2)(w) = \rho_W(g_1) \circ \rho_W(g_2)(w).
			\]  
			Par conséquent, \( \rho_W : G \rightarrow \mathrm{GL}(W) \) est un homomorphisme de groupes.
			
			\begin{myremark}  
				Il découle de ce résultat que, pour établir que \( \rho_W : G \rightarrow \mathrm{GL}(W) \) est une sous-représentation linéaire de \( \rho : G \rightarrow \mathrm{GL}(V) \), il suffit de montrer que \( W \) est un sous-espace vectorel de \(V\) stable sous l’action du groupe \( G \).  
			\end{myremark}  
			
			
			
			\begin{myexample}
				Soit \( V = \mathbb{R}^3 \) l'espace vectoriel sur lequel le groupe symétrique \( S_3 \) agit par permutation des coordonnées. Pour une permutation \( \sigma \in S_3 \), l'action de \( \sigma \) sur un vecteur \( (x_1, x_2, x_3) \in \mathbb{R}^3 \) est donnée par :
				\[
				\sigma \cdot (x_1, x_2, x_3) = (x_{\sigma(1)}, x_{\sigma(2)}, x_{\sigma(3)}).
				\]
				Cela définit une représentation linéaire \( \rho : S_3 \rightarrow \mathrm{GL}(\mathbb{R}^3) \), où chaque permutation est associée à une matrice de permutation réarrangeant les coordonnées.\\
				Considérons le sous-espace vectoriel \( W \subset V \) défini par :
				\[
				W = \text{Span}((1, 1, 1)) = \{(x, x, x) \mid x \in \mathbb{R}\}.
				\]
				
				\( \rho_W : S_3 \rightarrow \mathrm{GL}(W) \) est une sous-représentation linéaire de \((V,\rho)_{S_3} \).\\
				
				\textbf{\underline{En effet:}}\\
				Pour toute permutation \( \sigma \in S_3 \) et pour tout \( v \in W \), nous devons avoir \( \sigma \cdot v \in W \).\\
				Soient \( v = (x, x, x) \in W \) et \( \sigma \in S_3 \), alors :
				\[
				\rho_W(\sigma) ((x, x, x))  = (x, x, x) \in W.
				\]
				Donc \( W \) est stable sous l'action de \( S_3 \). Ce qui montre que  \( \rho_W : S_3 \rightarrow \mathrm{GL}(W) \) est une sous-représentation de \((V,\rho)_{S_3} \).
			\end{myexample}
			
			
			\begin{lemma} \cite{renard2009groupes}\\
				Soient \( \varphi : G \to \mathrm{GL}(V_1) \) et \( \psi : G \to \mathrm{GL}(V_2) \) deux représentations linéaires de \(G\).\\
				Soit \( f : V_1 \rightarrow V_2 \) un morphisme de représentations linéaires. Alors:
				
				\begin{enumerate} [label=\roman*)]
					\item \( \rho_{\ker(f)} : G \rightarrow \mathrm{GL}(\ker(f)) \) est une sous-représentation linéaire de \( \varphi : G \to \mathrm{GL}(V_1) \);
					\item l'image \( \mathrm{im}(f) \)  \( \rho_{\mathrm{im}(f)} : G \rightarrow \mathrm{GL}(\mathrm{im}(f)) \) est une sous-représentation linéaire de \\
					\( \psi : G \to \mathrm{GL}(V_2) \);
					\item \(V_1 / \ker(f) \cong \mathrm{im}(f)  \)
					au sens de représentations linéaires de \(G\).
				\end{enumerate}
			\end{lemma}
			
			
			
			\textbf{\underline{Preuve:}}\\
			\textbf{(i)} Soit \( v \in \ker(f) \), c’est-à-dire \( f(v) = 0 \). Pour tout \( g \in G \), nous devons montrer que \( \varphi(g) \cdot v \in \ker(f) \).\\
			Puisque \( f \) est un morphisme de représentations, nous avons :  
			\[
			f \circ \varphi(g) = \psi(g) \circ f.
			\]
			Ainsi, on a :  
			\[
			\begin{aligned}
				f(\varphi(g)(v)) &= (f(\varphi(g)))(v) \\
				&= (\psi(g) \circ f)(v) \\
				&= \psi(g) \circ f(v) \\
				&= \psi(g)(0) \\
				&= \psi(g)(0 \cdot 1) \\
				&= 0 \cdot \psi(g)(1) \\
				&= 0.
			\end{aligned}
			\]
			Donc \( \varphi(g) \cdot v \in \ker(f) \). Ce qui prouve que \( \ker(f) \) est stable par l’action de \( G \).\\
			Par ailleurs, \( \ker(f) \) est un \( \mathbb{K} \)-sous-espace vectoriel de \( V_1 \). Ainsi, 
			\[ \rho_{\ker(f)} : G \to \mathrm{GL}(\ker(f))\]  
			est une sous-représentation linéaire de \( \varphi : G \to \mathrm{GL}(V_1) \).\\
			\textbf{(ii)} Soit \( w \in \operatorname{im}(f) \), c’est-à-dire qu’il existe \( v \in V_1 \) tel que \( f(v) = w \). Nous devons montrer que, pour tout \( g \in G \), \( \psi(g) \cdot w \in \operatorname{im}(f) \).  \\
			On a:  
			\[
			\begin{aligned}
				\psi(g)(w) &= \psi(g)(f(v)) \\
				&= (\psi(g) \circ f)(v) \\
				&= (f \circ \varphi(g))(v) \\
				&= f(\varphi(g)(v)).
			\end{aligned}
			\]
			Ainsi, \( \psi(g) \cdot w \in \operatorname{im}(f) \). Ce qui prouve que \( \operatorname{im}(f) \) est stable par l’action de \( G \).\\  
			De plus, \( \operatorname{im}(f) \) est un \( \mathbb{K} \)-sous-espace vectoriel de \( V_2 \). Il s’ensuit que
			\[\rho_{\operatorname{im}(f)} : G \to \mathrm{GL}(\operatorname{im}(f))\]
			est une sous-représentation linéaire de  
			\( \psi : G \to \mathrm{GL}(V_2) \).\\	
			
			\textbf{(iii)}Soit \( f : V_1 \to V_2 \) un morphisme de représentations de \( G \).\\
			L'application \( f \) induit une application \\
			\( \tilde{f}: V_1 / \ker(f) \to \operatorname{im}(f)$ défini par $\tilde{f}([v]) = f(v) \quad \forall v \in V_1 \), \\
			où \( [v] \) désigne la classe d'équivalence de \( v \) dans l'espace quotient \( V_1 / \ker(f) \).\\
			
			\textbf{\underline{Bonne définition de \( \overline{f} \)} :}\\
			L'application \( \overline{f} \) est bien définie si pour tous \( v_1, v_2 \in V_1 \), \( v_1 \sim v_2 \) (c'est-à-dire \( v_1 - v_2 \in \ker(f) \)), nous avons \( f(v_1) = f(v_2) \).\\
			Si \( v_1 - v_2 \in \ker(f) \), alors :
			\[
			\begin{aligned}
				f(v_1-v_2)=0 & \Rightarrow f(v_1)-f(v_2)=0 \\
				&\Rightarrow f(v_1)=f(v_2).
			\end{aligned}
			\]
			Ce qui garantit que \( \overline{f}([v_1]) = \overline{f}([v_2]) \). Donc \( \overline{f} \) est bien définie.\\
			
			\textbf{\underline{Surjectivité de \( \overline{f} \)} :}\\
			Soit \( w \in \operatorname{im}(f) \). Il existe un \( v \in V_1 \) tel que \( f(v) = w \). Donc \( \overline{f}([v]) = f(v) = w \). Ainsi, chaque élément de \( \operatorname{im}(f) \) est l'image d'une classe d'équivalence dans \( V_1 / \ker(f) \), ce qui prouve que \( \overline{f} \) est surjective.\\
			
			\textbf{\underline{Injectivité de \( \overline{f} \)} :}\\
			Supposons que \( \overline{f}([v]) = 0 \), c'est-à-dire que \( f(v) = 0 \). Cela implique que \( v \in \ker(f) \). Donc \( [v] = [0] \). Par conséquent, \( \overline{f} \) est injective.\\
			L'application \( \overline{f} \) est donc un isomorphisme d'espaces vectoriels entre \( V_1 / \ker(f) \) et \( \operatorname{im}(f) \).\\
			Enfin, montrons que $\tilde{f}$ est un morphisme de représentations linéaires. \\
			Pour tout $g \in G$ et $v \in V_1$, nous avons :
			\[
			\begin{aligned}
				\tilde{f} \circ (\varphi(g) ([v])) &= \tilde{f}([\varphi(g) (v)])\\
				&= f(\varphi(g) (g)) \\
				&= \psi(g) \circ f(v) \\
				&= \psi(g) \circ\tilde{f}([v]).
			\end{aligned}
			\]
			Cela montre que \( \overline{f} \) est un morphisme de représentations linéaires.\\
			Il en resulte que l'application \( \overline{f} : V_1 / \ker(f) \to \operatorname{im}(f) \) est un isomorphisme de représentations linéaires de \( G \) et par consequent,
			\[ V_1 / \ker(f) \cong \mathrm{im}(f) \] 
			au sens de représentations linéaires de \(G\).	
			
			
			
			
			\section{Caractère d'une représentation linéaire d'un groupe fini}
			Dans la suite sauf mention contraire, \( G \) désignera un groupe fini (pas forcément commutatif), noté multiplicativement avec pour élément neutre \( 1 \), les corps considérés seront algébriquement clos de caractéristique zéro et les espaces vectoriels de dimension finie $n$.
			
			\begin{definition} \cite{serre1971representation} \\
				Soit \(\rho : G \rightarrow \mathrm{GL}(V)\) une représentation linéaire sur \( G \). Le caractère de \( V \), noté \( \chi_V \), est la fonction 
				\[
				\chi_V : G \to \mathbb{C}
				\]
				définie pour tout \( g \in G \) par
				\[
				\chi_V(g) := \operatorname{Tr}(\rho(g)),
				\]
				où \( \operatorname{Tr} \) désigne la trace.
			\end{definition}
			
			
			\begin{theorem} \cite{renard2009groupes}\\
				Soit \( (\rho, V)_{G}\) une représentation linéaire. On peut munir \( V \) d’un produit hermitien \( (.,.)_V \) qui rend la représentation linéaire \( (\rho, V)_{G}\) unitaire.\\
			\end{theorem}
			
			\textbf{\underline{Preuve:}}  \\
			Munissons \( V \) d’un produit hermitien \( (.,.)_0 \) quelconque. Définissons un nouveau produit  \( (.,.)_1 \)  par  
			\[
			(v, w)_1 = \frac{1}{|G|} \sum_{g \in G} (\rho(g) \cdot v, \rho(g) \cdot w)_0, \quad \forall v, w \in V.
			\]
			C'est un produit hermitien.\\
			\textbf{\underline{En effet:}}\\ 
			\textbf{\underline{Sesquilinéarité}}\\   
			Nous avons déjà supposé que \( (.,.)_0 \) est un produit hermitien, donc il est sesquilinéaire, c'est-à-dire linéaire à gauche et semi-linéaire à droite. 
			\[
			(\lambda v + \mu v', w)_0 = \lambda (v, w)_0 + \mu (v', w)_0, \quad \forall v, v', w \in V, \ \lambda, \mu \in \mathbb{C}.
			\]
			L'opérateur \( \rho(g) \) étant linéaire, on en déduit que la somme pondérée définissant \( (.,.)_1 \) conserve la sesquilinéarité :  
			\[
			(v, w)_1 = \frac{1}{|G|} \sum_{g \in G} (\rho(g) \cdot v, \rho(g) \cdot w)_0
			\]
			hérite donc de cette propriété.\\
			\textbf{\underline{Symétrie hermitienne}}\\  
			Par définition du produit hermitien, nous avons :  
			\[
			(v, w)_0 = \overline{(w, v)_0}.
			\]
			En appliquant cette propriété à chaque terme de la somme, on obtient :
			\begin{align*}
				(v, w)_1 
				&= \frac{1}{|G|} \sum_{g \in G} (\rho(g) \cdot v, \rho(g) \cdot w)_0 \\
				&= \frac{1}{|G|} \sum_{g \in G} \overline{(\rho(g) \cdot w, \rho(g) \cdot v)_0} \\
				&= \overline{(w, v)_1}. \\
			\end{align*}
			Donc \( (.,.)_1 \) est bien hermitien.\\
			\textbf{\underline{Positivité }}\\  
			Nous devons montrer que \( (v, v)_1 \geq 0 \) avec égalité si et seulement si \( v = 0 \). Par définition :
			\[
			(v, v)_1 = \frac{1}{|G|} \sum_{g \in G} (\rho(g) \cdot v, \rho(g) \cdot v)_0.
			\]
			Chaque terme de la somme est positif ou nul car \( (.,.)_0 \) est un produit hermitien. De plus, si \( (v, v)_1 = 0 \), alors chaque terme de la somme est nul, ce qui implique que \( (\rho(g) \cdot v, \rho(g) \cdot v)_0 = 0 \) pour tout \( g \). Puisque \( (.,.)_0 \) est un produit hermitien, cela entraîne \( \rho(g) \cdot v = 0 \) pour tout \( g \), et en particulier \( v = 0 \). Ainsi, \( (.,.)_1 \) est un produit hermitien défini positif.\\
			\textbf{\underline{Invariance par \( \rho \)} }\\
			Il reste à prouver que \( (.,.)_1 \) est invariant sous l'action de \( \rho \), c’est-à-dire que  
			\[
			(\rho(h) \cdot v, \rho(h) \cdot w)_1 = (v, w)_1, \quad \forall h \in G.
			\]
			En utilisant la définition de \( (.,.)_1 \) :
			\begin{align*}
				(\rho(h) \cdot v, \rho(h) \cdot w)_1 
				&= \frac{1}{|G|} \sum_{g \in G} (\rho(g) \cdot \rho(h) \cdot v, \rho(g) \cdot \rho(h) \cdot w)_0 \\
				&= \frac{1}{|G|} \sum_{g \in G} (\rho(gh) \cdot v, \rho(gh) \cdot w)_0 \quad (\text{en posant } g' = gh, \text{ donc } g = g' h^{-1}) \\
				&= \frac{1}{|G|} \sum_{g' \in G} (\rho(g') \cdot v, \rho(g') \cdot w)_0 \quad (\text{car } g' \text{ parcourt } G \text{ lorsque } g \text{ parcourt } G) \\
				&= (v, w)_1.
			\end{align*}
			Ainsi, \( (.,.)_1 \) est invariant par \( \rho \), ce qui montre que \( (\rho, V)_G \) est bien une représentation unitaire.
			
			
			\begin{definition} \cite{serre1971representation} \\
				Soit $G$ un groupe fini. Une fonction centrale sur $G$ est une fonction $f : G \to \mathbb{C}$ telle que, pour tous éléments $g, h \in G$, la relation suivante est satisfaite :
				\[
				f(ghg^{-1}) = f(h).
				\]
				Autrement dit, une fonction centrale est une fonction qui reste constante sur les classes de conjugaison de $G$. L'ensemble des fonctions centrales sur $G$ forme un $\mathbb{C}$-espace vectoriel noté $C(G)$, dont la dimension est égale au nombre $c(G)$ de classes de conjugaison de $G$. 	
			\end{definition}
			
			\begin{remark} \cite{renard2009groupes}\\
				Les fonctions centrales sont particulièrement importantes car tout caractère d'une représentation de $G$ est une fonction centrale. En effet, si $(V, \rho)$ est une représentation de $G$, le caractère $\chi_V$ associé satisfait la relation suivante pour tous $g, h \in G$ :
				\begin{align*}
					\chi_V(ghg^{-1}) &= \operatorname{Tr}(\rho(ghg^{-1})) \\
					&= \operatorname{Tr}(\rho(g) \rho(h)\rho(g^{-1})) \\ 
					&= \operatorname{Tr}(\rho(h)\rho(g^{-1}) \rho(g) ) \\ 
					&= \operatorname{Tr}(\rho(h)\rho(g^{-1}g)) \\
					&= \operatorname{Tr}(\rho(h)\rho(1)) \\
					&= \operatorname{Tr}(\rho(h)) \\
					&= \chi_V(h).
				\end{align*}
				ce qui montre que $\chi_V$ est constante sur les classes de conjugaison de $G$.
			\end{remark}
			
			
			\begin{proposition} \cite{serre1971representation} \\
				Soient \( \rho_V : G \rightarrow \mathrm{GL}(V) \) et \( \rho_w : G \rightarrow \mathrm{GL}(W) \) deux représentations linéaires sur \(G\) de dégres \(n\) et \(m\) (\(n,m \in \mathbb{N^*}\)), et de Caractères \( \chi_V \) et \(\chi_W \)  respectivemet.On a :
				\begin{enumerate}[label=\roman*)]
					\item \( \chi_V(1) = \dim V \);
					\item \( \chi_{V \oplus W} = \chi_V + \chi_W \);
					\item \( \chi{(g)^*} = {\chi_{(g)}^{-1}} \) \quad \text{\(\forall g \in G\)}. 
				\end{enumerate}
			\end{proposition}
			
			
			\textbf{\underline{Preuve:}} 
			\begin{enumerate}[label=\roman*)]
				\item On a:
				
				\[
				\begin{aligned}
					\chi_V(1) &= \operatorname{Tr}(\rho(1)) \\
					&= \operatorname{Tr}(\mathrm{Id}_V) \\
					&= \dim V.
				\end{aligned}
				\]
				Donc \( \chi_V(1) = \dim V \).
				\item  Soient \( \rho_V : G \rightarrow \mathrm{GL}(V) \) et \( \rho_w : G \rightarrow \mathrm{GL}(W) \) deux représentations linéaires sur \(G\) et $g \in G$.Soient $\mathcal{B}_V$ une base de $V$ et $\mathcal{B}_W$ une base de $W$, la matrice de $\rho_{V \oplus W}(s)$ s'écrit dans la base $\mathcal{B} := \mathcal{B}_V \cup \mathcal{B}_W$ :
				\[
				M(g) = \begin{pmatrix} \rho_V(s) & 0 \\ 0 & \rho_W(s) \end{pmatrix}
				\]
				où $\rho_V(s)$ est la matrice de $\rho_V(s)$ dans la base $\mathcal{B}_V$ et $\rho_W(s)$ celle de $\rho_W(s)$ dans $\mathcal{B}_W$.Ainsi,
				\[
				\begin{aligned}
					\chi_{V \oplus W}(g) &= \operatorname{Tr}(\rho_{V \oplus W}(g)) \\
					&= \operatorname{Tr}(M(g)) \\
					&= \operatorname{Tr} \begin{pmatrix} \rho_V(g) & 0 \\ 0 & \rho_W(g) \end{pmatrix}.
				\end{aligned}
				\]
				La trace d'une matrice diagonale est la somme des traces des blocs diagonaux, donc :
				\[
				\begin{aligned}
					\chi_{V \oplus W}(g) &= \operatorname{Tr}(\rho_V(g)) + \operatorname{Tr}(\rho_W(g)) \\
					&= \chi_V(g) + \chi_W(g).
				\end{aligned}
				\]
				Par consequent, \( \chi_{V \oplus W} = \chi_V + \chi_W \).
				\item On a :
				\begin{align*}
					\chi(g)^* &= \operatorname{Tr}(\rho_g)^* \\
					&= \sum \lambda_i^* \\
					&= \sum \lambda_i^{-1} \\
					&= \operatorname{Tr}(\rho_{g^{-1}}) \\
					&= \chi(g^{-1}).
				\end{align*}
				Donc \( \chi{(g)^*} = {\chi_(g)^{-1}} \) \quad \text{\(\forall g \in G\)}.	
			\end{enumerate}
			
			
			\begin{theorem}[Frobenius] \cite{serre1971representation} \\
				Pour tout groupe fini \( G \), le nombre de représentations irréductibles non-isomorphes deux à deux de \( G \) est exactement égal au nombre \( c(G) \) de classes de conjugaison de \( G \).
			\end{theorem}
			
			\begin{proposition} \cite{renard2009groupes}\\
				Deux représentations d’un groupe \( G \) sont isomorphes si et seulement si elles ont le même caractère.
			\end{proposition}
			
			\section{Représentations linéaires irréductibles}
			\begin{definition}  \cite{serre1971representation} \\
				On dit qu'une représentation linéaire \( \rho : G \rightarrow \mathrm{GL}(V) \) est \emph{irréductible} si l'espace vectoriel \( V \) n'est pas réduit à \( \{0\} \) et si \( V \) ne possède aucun sous-espace invariant par \( \rho \) autre que \( \{0\} \) et \( V \).	
			\end{definition}
			
			\begin{example} 
				\begin{enumerate} \
					\item Toute représentation de degré 1 est irréductible.
					\item Prenons le groupe cyclique d'ordre 3, \( G = C_3 = \langle g \ | \ g^3 = e \rangle \), où \( e \) est l'élément neutre.
					Considérons \( V = \mathbb{R}^2 \). Définissons une représentation \( \rho \) de \( G \) sur \( V \) comme suit :
					\[
					\rho(g) =
					\begin{pmatrix}
						\cos\left(\frac{2\pi}{3}\right) & -\sin\left(\frac{2\pi}{3}\right) \\
						\sin\left(\frac{2\pi}{3}\right) & \cos\left(\frac{2\pi}{3}\right)
					\end{pmatrix}
					=
					\begin{pmatrix}
						-\frac{1}{2} & -\frac{\sqrt{3}}{2} \\
						\frac{\sqrt{3}}{2} & -\frac{1}{2}
					\end{pmatrix}.
					\]
					Cette matrice représente une rotation dans \( \mathbb{R}^2 \) de \( 120^\circ \) dans le sens trigonométrique. Comme \( g^3 = e \), on a bien que \( \rho(g)^3 = I_2 \) (la matrice identité) et \(\det(\rho(g))=1 \neq 0\), la matrice \(\rho(g)\) est inversiblen et par consequent \( \rho \) est bien une représentation du groupe \( C_3 \).\\
					Montrons que les sous-espaces vectoriels propres de \( \mathbb{R}^2 \) qui sont : 
					\( W_1 = \{ (x, 0) \in \mathbb{R}^2 \setminus \{0\} \} \), 
					\( W_2 = \{ (0, y) \in \mathbb{R}^2 \setminus \{0\} \} \), 
					et 
					\( W_3 = \{ (z, z) \in \mathbb{R}^2 \setminus \{0\} \} \) 
					ne sont pas stables par \( \rho \).\\
					Soient \( (x, 0) \in W_1 \), \( (0, y) \in W_2 \), et \( (z, z) \in W_3 \). On a :
					\[
					\rho(g) \begin{pmatrix} x \\ 0 \end{pmatrix} = 
					\begin{pmatrix} -\frac{1}{2} & -\frac{\sqrt{3}}{2} \\ \frac{\sqrt{3}}{2} & -\frac{1}{2} \end{pmatrix} 
					\begin{pmatrix} x \\ 0 \end{pmatrix} = 
					\begin{pmatrix} -\frac{1}{2}x \\ \frac{\sqrt{3}}{2}x \end{pmatrix}
					\notin W_1.
					\]
					Ce qui montre que \(W_1\) n'est pas stable par \(\rho\).		
					\[
					\rho(g) \begin{pmatrix} 0 \\ y \end{pmatrix} = 
					\begin{pmatrix} -\frac{1}{2} & -\frac{\sqrt{3}}{2} \\ \frac{\sqrt{3}}{2} & -\frac{1}{2} \end{pmatrix} 
					\begin{pmatrix} 0 \\ y \end{pmatrix} = 
					\begin{pmatrix} -\frac{\sqrt{3}}{2}y \\ -\frac{1}{2}y \end{pmatrix}
					\notin W_2.
					\]
					Donc \(W_2\) n'est pas stable par \(\rho\).
					\[
					\rho(g) \begin{pmatrix} z \\ z \end{pmatrix} = 
					\begin{pmatrix} -\frac{1}{2} & -\frac{\sqrt{3}}{2} \\ \frac{\sqrt{3}}{2} & -\frac{1}{2} \end{pmatrix} 
					\begin{pmatrix} z \\ z \end{pmatrix} = 
					\begin{pmatrix} -\frac{\sqrt{3}}{2}z - \frac{1}{2}z \\ \frac{\sqrt{3}}{2}z - \frac{1}{2}z \end{pmatrix}
					\notin W_3.
					\]
					Ce qui montre que \(W_3\) n'est pas stable par \(\rho\).\\
					Il en resulte que \( \rho \) est irréductible.
				\end{enumerate}
			\end{example}
			
			
			\begin{theorem}\cite{serre1971representation} \\
				Soit $\rho : G \to GL(V)$ une représentation linéaire de $G$ dans $V$ et soit 
				$W$ un sous-espace vectoriel de $V$ stable sous $G$. Alors, il existe un complément $W^0$ de $W$ dans $V$ qui est stable sous $G$.\\
			\end{theorem}
			
			\textbf{\underline{Preuve:}}\\
			Soit \(  p : V \to W \) une projection linéaire.\\
			Considérons l'application
			\[
			p^0 : V \to W \quad \text{définie par} \quad  \text{\(	p^0 = \frac{1}{|G|} \sum_{g \in G} \rho(g) \circ p \circ \rho(g)^{-1}.\)} 
			\]
			où $|G|$ est l'ordre de $G$.\\
			$p^0$ est bien définie car $p$ envoie $V$ dans $W$ et que $W$ est stable sous $\rho(g)$ pour tout $g \in G$. \\
			Par construction, $p^0$ est un opérateur linéaire.\\
			Montrons que $p^0$ est une projection. \\
			Soit $x \in W$, on a:
			\[
			\begin{aligned}
				\rho(g)^{-1} (x) \in W & \implies p(\rho(g)^{-1} (x)) = \rho(g)^{-1} (x)  \quad \text{car $p$ est une projection}.\\
			\end{aligned}
			\]
			Ainsi, on a:
			\[
			\begin{aligned}
				p^0(x) &= \frac{1}{|G|} \sum_{g \in G} \rho(g) \circ p(\rho(g)^{-1}(x)) \quad \text{(par définition de \( p \))} \\
				&= \frac{1}{|G|} \sum_{g \in G} \rho(g)(\rho(g)^{-1}(x)) \quad \text{(par définition de \( p \))} \\
				&= \frac{1}{|G|} \sum_{g \in G} \rho(gg^{-1})(x) \quad \text{(par définition de \( \rho \))} \\
				&= \frac{1}{|G|} \sum_{g \in G} \rho(1)(x) \\
				&= \frac{|G|}{|G|} x \\
				&= x.
			\end{aligned}
			\]
			On obtient $p^0(x) = x$ et par suite $p^{0}(p^0(x)) = p^{0}(x)$.
			Cela montre que $p^0$ est une projection sur $W$.\\
			Le noyau de $p^0$, noté $W^0$, est défini par :
			\[
			\begin{aligned}
				W^0 &= \ker(p^0) \\
				&= \{x \in V \mid p^0(x) = 0\}.
			\end{aligned}
			\]
			Comme $p^0$ est une projection, nous avons la décomposition suivante :
			\[
			V = W \oplus W^0.
			\]
			Montrons que $W^0$ est stable sous $G$.\\
			Soient $x \in W^0$ et $h \in G$. On a :
			\[
			\begin{aligned}
				\rho(h) \circ p^0 \circ \rho(h)^{-1} &= \frac{1}{|G|} \sum_{g \in G} \rho(h) \circ \rho(g) \circ p \circ \rho(g)^{-1} \circ \rho(h) \\
				&= \frac{1}{|G|} \sum_{g \in G} \rho(hg) \circ p \circ \rho(hg)^{-1} \\
				&= \frac{1}{|G|} \sum_{g \in G} \rho(g) \circ p \circ \rho(g)^{-1} \\
				&= p^0.
			\end{aligned}
			\]
			Cela montre ainsi que \( \rho(h) \) et \( p^0 \) commutent.
			Ce qui implique que $\rho(h)(x) \in W^0$. Donc $W^0$ est stable sous $G$.
			Ainsi, $W^0$ est un complément de $W$ dans $V$ qui est stable sous $G$.
			
			
			\begin{theorem}[Théorème de Maschke] \cite{serre1971representation} \\
				Toute représentation linéaire \(\rho : G \rightarrow \mathrm{GL}(V)\) sur \( G \) dans un espace vectoriel complexe de dimension finie se décompose en somme directe de représentations irréductibles.
			\end{theorem}
			
			
			\textbf{\underline{Preuve:}}\\
			Soit \( \rho : G \rightarrow \mathrm{GL}(V) \) une représentation linéaire et \( W \) un sous-espace vectoriel stable par \( \rho \). Nous montrons que \( W \) admet un supplémentaire \( W' \) dans \( V \), également stable par \( \rho \). \\
			Soient \(  p : V \to W \) et \( p^0 : V \to W \quad \text{définie par} \quad  \text{\(	p^0 = \frac{1}{|G|} \sum_{g \in G} \rho(g) \circ p \circ \rho(g)^{-1}\)}  \)  deux projections linéaires. \\
			On a d'après le \textbf{théorème 2.3.1} \quad  \( V = W \oplus W^0 \) \quad avec \quad \( W^0 = \ker(p^0).\) (*) \\
			Nous déduisons le résultat par récurrence sur \( \dim(V) \).
			\begin{itemize}
				\item Si \( \dim(V) = 1 \), le résultat est trivial.\\
				Supposons que \( \dim(V) > 1 \).
				\item Si \( V \) est irréductible, la situation est triviale.
				\item  Si \( V \) n'est pas irréductible, alors il existe un sous-espace vectoriel \( W \) de \( V \) stable par \( \rho \) avec \( 0 < \dim(W) < \dim(V) \).
				Par (*), il existe un sous-espace \( W' \) de \( V \) stable par \( \rho \) tel que \( V = W \oplus W' \). Ainsi, en appliquant ce processus à \( W \) et \( W' \) , on a le résultat. 
			\end{itemize}
			
			
			\begin{theorem}[Lemme de Schur]	 \cite{serre1971representation} \\
				Soient \( \rho_1 : G \rightarrow V_1 \) et \( \rho_2 : G \rightarrow V_2 \) deux représentations irréductibles de \( G \). \\
				Soit \( f : V_1 \rightarrow V_2 \) une application linéaire vérifiant
				\[
				\forall g \in G, \quad f \circ \rho_1(g) = \rho_2(g) \circ f.
				\]
				On a les propriétés suivantes :
				\begin{enumerate} [label=\roman*)]
					\item[(i)] Si \( \rho_1 \) et \( \rho_2 \) ne sont pas isomorphes, alors \( f = 0 \).
					\item[(ii)] Si  \(V_1 = V_2 \) et \( \rho_1 = \rho_2 \), alors \( f \) est une homothétie.
				\end{enumerate}
			\end{theorem}
			
			
			\textbf{\underline{Preuve:}}
			\begin{enumerate} [label=\roman*)]
				\item Supposons par la contraposée que \( f \neq 0 \) et montrons que \( \rho_1 \) et \( \rho_2 \) sont isomorphes.\\
				Le sous-espace \( \ker(f) \) de \( V_1 \) est stable par \( \rho_1 \). Comme \( f \neq 0 \), on a \( \ker(f) = \{0\} \) par irréductibilité de \( \rho_1 \).\\
				De même, le sous-espace \( \mathrm{Im}(f) \) de \( V_2 \) est stable par \( \rho_2 \). Comme \( f \neq 0 \), on en déduit que \( \mathrm{Im}(f) = V_2 \) par irréductibilité de \( \rho_2 \).\\
				Il vient que \( f \) est un isomorphisme d'espaces vectoriels.\\
				Par hypothèse, \( f \) est un morphisme de représentations linéaires. \\
				D'où \( \rho_1 \) et \( \rho_2 \) sont isomorphes. \\
				Il en résulte que si \( \rho_1 \) et \( \rho_2 \) ne sont pas isomorphes, alors \( f = 0 \).	
				\item Comme \( \mathbb{C} \) est algébriquement clos, l’endomorphisme \( f \) possède au moins une valeur propre \( \lambda \). Posons alors \( f' = f - \lambda \mathrm{id} \). On a \( \rho_2(g) \circ f' = f' \circ \rho_1(g) \) pour tout \( g \in G \). Comme \( f' \) n’est pas bijectif, on a \( f' = 0 \) d'après (ii). Il en résulte que \( f = \lambda \mathrm{id} \), et par conséquent \( f \) est une homothétie.
			\end{enumerate}
			
			
			\begin{corollary} \cite{serre1971representation}\\
				Soit \( h  : V \to W \) une application linéaire. Posons :
				\[
				h_0 = \frac{1}{N} \sum_{g \in G} (\rho_W(g))^{-1} \circ h \circ \rho_V(g).
				\]
				\( N = \mathrm{Card}(G) .\)
				Alors :
				\begin{enumerate}[label=\roman*)]
					\item Si \( \rho_V \) et \( \rho_W \) ne sont pas isomorphes, on a \( h_0 = 0 \).
					\item Si \( V = W \) et \( \rho_V = \rho_W \), \( h_0 \) est une homothétie de rapport \( \frac{1}{n} \text{Tr}(h)  \), où \( n = \dim(V) \).
				\end{enumerate}
			\end{corollary}
			
			
			\textbf{\underline{Preuve:}}\\
			\begin{enumerate}[label=\roman*)]
				\item en appliquant le \textbf{lemme de Schur} à \( f = h_0 \), on a \( h_0 = 0 \).
				\item \( h_0 \) est une homothétie.De plus, \[
				\begin{aligned}
					\text{Tr}(h_0) &= \frac{1}{N} \sum_{g \in G} \text{Tr}((\rho_V(g))^{-1} \circ h \circ \rho_V(g)) \\
					&= \text{Tr}(h).
				\end{aligned}
				\]
				On en déduit bien alors que \( h^0 = \frac{\text{Tr}(h)}{n} \mathrm{I_n} \), vu que la trace de l’identité est \( n \)
			\end{enumerate}
			
			\begin{remark} \cite{renard2009groupes}\\
				Une traduction matricielle du corollaire précédent est: 
				si \( \varphi \) et \( \psi \) sont des fonctions \( G \to \mathbb{C} \), alors
				\[
				\langle \varphi, \psi \rangle = \frac{1}{g} \sum_{t \in G} \varphi(t^{-1})\psi(t) 
				= \frac{1}{g} \sum_{t \in G} \varphi(t)\psi(t^{-1}). \tag{1}
				\]
			\end{remark}
			
			\begin{proposition} \cite{serre1971representation}\\
				Pour $g \in G$, soient $(r_{i_1 j_1}(g))$ et $(u_{i_2 j_2}(g))$ les matrices respectives de $\rho_V(g)$ et $\rho_W(g)$ dans les bases $\mathcal{B}_1$, $\mathcal{B}_2$ de $V_1$, $V_2$. Alors :
				\begin{enumerate}[label=\roman*)]
					\item Si $\rho_V$ et $\rho_W$ ne sont pas isomorphes, on a $\langle u_{i_2 j_2}, r_{j_1 i_1} \rangle = 0$ pour tous indices $i_1, j_1, i_2, j_2$.
					\item Si $V_1 = V_2$ est de dimension $n$ et $\rho_V = \rho_W$ (auquel cas on prend $\mathcal{B}_1 = \mathcal{B}_2$ et on a $r_{ij} = u_{ij}$ pour tous indices $i, j$), alors $\langle r_{i_2 j_2}, r_{j_1 i_1} \rangle = 0$ si $i_1 \neq i_2$ ou $j_1 \neq j_2$, et $\langle r_{ij}, r_{ji} \rangle = \frac{1}{n}$ pour tous indices $i, j$.
				\end{enumerate}
			\end{proposition}
			
			
			\textbf{\underline{Preuve:}}\\
			Soit $h : V_1 \to V_2$ une application linéaire quelconque, de matrice $(x_{i_2i_1})$ dans les bases $\mathcal{B}_1, \mathcal{B}_2$. On peut lui associer l'application linéaire $h^0$ comme dans le corollaire precedent avec pourmatrice $(x^0_{i_2i_1})$. Alors chaque composante de la matrice s'ecrit:
			
			\[
			\begin{aligned}
				x^0_{i_2 i_1} &= \frac{1}{N} \sum_{g \in G} \sum_{j_1,j_2} u_{i_2 j_2}(g^{-1}) x_{j_2 j_1} r_{j_1 i_1}(g) \\
				&= \sum_{j_1,j_2} \left( \frac{1}{N} \sum_{g \in G} u_{i_2 j_2}(g^{-1}) r_{j_1 i_1}(g) \right) x_{j_2 j_1}
			\end{aligned}
			\]
			\( N = \mathrm{Card}(G) \) 
			
			\begin{enumerate}[label=\roman*)]
				\item Comme  \( \rho_1 \) et \( \rho_2 \) ne sont pas isomorphes, alors \( h^0 = 0 \).Il vient que la matrice $(x^0_{i_2i_1})$ est une matrice nulle et par consequent  
				\[
				x^0_{i_2 i_1} =\sum_{j_1,j_2} \left( \frac{1}{N} \sum_{g \in G} u_{i_2 j_2}(g^{-1}) r_{j_1 i_1}(g) \right) x_{j_2 j_1} =0.
				\]
				Ce qui implique :
				\[
				\frac{1}{N} \sum_{g \in G} u_{i_2j_2}(g^{-1}) r_{j_1i_1}(g) = 0.
				\]
				Donc $\langle u_{i_2j_2}, r_{j_1i_1} \rangle = 0$.\\
				
				\item On sait que $h^0$ est une homothétie de rapport $\lambda = \frac{\mathrm{Tr}\,h}{n}$. Cela signifie que  \( h^0 = \frac{\text{Tr}(h)}{n} \mathrm{I_n} \). D’où $x^0_{i_2i_1} = 0$ si $i_2 \neq i_1$ et par suite
				$\langle r_{i_2j_2}, r_{j_1i_1} \rangle = 0$ si $i_2 \neq i_1$. \\
				Si $i_2 = i_1 = i$, alors
				\[
				x^0_{ii} = \frac{1}{n} \mathrm{Tr}\,h = \frac{1}{n} \sum_{j_1=j_2} x_{j_2j_1}.
				\]
				Ce qui donne $\langle r_{ij_2}, r_{j_1i} \rangle = 0$ si $j_1 \neq j_2$ et 
				$\langle r_{ij}, r_{ji} \rangle = 1/n$ pour tous $i, j$.
				
			\end{enumerate}
			
			
			\begin{definition} \cite{serre1971representation}\\
				Soit $G$ un groupe fini de cardinal $N$. On définit un produit scalaire hermitien sur l’espace vectoriel $\mathscr{F}(G, \mathbb{C})$ des fonctions de $G$ dans $\mathbb{C}$, par la formule :
				\[
				(\varphi | \psi) := \frac{1}{N} \sum_{g \in G} \varphi(g) \overline{\psi(g)}.
				\]
				Si $\varphi$ et $\psi$ sont des caractères, les deux formules 
				coïncident car dans ce cas $\overline{\varphi(g)} = \varphi(g^{-1})$.
			\end{definition}
			
			
			\begin{theorem} \cite{serre1971representation}
				\begin{enumerate}[label=\roman*)]
					\item Soit \( \chi \) le caractère d’une représentation irréductible \( \rho \) de \( G \). Alors, \( (\chi | \chi) = 1 \).
					\item Soient \( \chi_1 \) et \( \chi_2 \) les caractères de deux représentations irréductibles non isomorphes \( \rho_1 \) et \( \rho_2 \). Alors, \( (\chi_1 | \chi_2) = 0 \).
				\end{enumerate}
			\end{theorem}
			
			\textbf{\underline{Preuve:}}\\
			\begin{enumerate} [label=\roman*)]
				\item Supposons \( \rho \) de degré \( n \), donnée sous forme matricielle \( \rho(g) = (r_{ij}(g)) \). Alors 
				\[
				\chi(g) = \sum_i r_{ii}(g),
				\]
				d'où 
				\[
				\langle \chi | \chi \rangle = \langle \chi, \chi \rangle = \sum_{i,j} \langle r_{ii}, r_{jj} \rangle.
				\]
				
				D'après la \textbf{ proposition 2.3.1}, on a \( \langle r_{ii}, r_{jj} \rangle = 0 \) si \( i \neq j \) et \( \langle r_{ii}, r_{jj} \rangle = 1/n \) si \( i = j \), ce qui donne finalement 
				\[
				\langle \chi | \chi \rangle = \left( \sum_{i,j} \delta_{ij} \right) / n = n / n = 1.
				\]
				
				\item Écrivons encore les formes matricielles respectives \( r_{ij}(g) \) et \( u_{ij}(g) \) de \( \rho_V \) et \( \rho_W \). Alors 
				\[
				\langle \chi_1 | \chi_2 \rangle = \sum_{i,j} \langle r_{ii}, u_{jj} \rangle,
				\]
				qui est nul d'après la  \textbf{ proposition 2.3.1}.
			\end{enumerate}
			
			
			\begin{theorem}  \cite{serre1971representation} \\
				Soit  \( G \) un groupe.Les propriétés suivantes sont équivalentes :
				\begin{enumerate}
					\item[(i)] \( G \) est abélien.
					\item[(ii)] Toutes les représentations irréductibles de \( G \) sont de degré 1.
				\end{enumerate}
			\end{theorem}
			
			\textbf{\underline{Preuve:}}\\
			\textbf{(i) \(\Rightarrow\) (ii)}\\
			Supposons que \( G \) est abélien. Soit \( \rho : G \to \mathrm{GL}(V) \) une représentation irréductible de \( G \), où \( \dim(V) = n \).\\
			Comme \( G \) est abélien, pour tout \( g, h \in G \), nous avons \( gh = hg \) et 
			\[
			\rho(gh) = \rho(g)\rho(h) = \rho(h)\rho(g).
			\]
			Cela signifie que les matrices \( \rho(g) \) et \( \rho(h) \) commutent pour tous \( g, h \in G \).Par conséquent, \( \dim(V) = 1 \), car \( V \) ne peut pas avoir de sous-espace propre invariant sous \( G \) (l'irréductibilité exige que \( V \) n'ait pas de sous-espaces invariants non triviaux).
			Donc toutes les représentations irréductibles de \( G \) ont degré 1.
			
			\textbf{(ii) \(\Rightarrow\) (i)}\\
			Supposons que toutes les représentations irréductibles de \( G \) ont degré 1. Prenons \( g, h \in G \) et considérons une représentation irréductible quelconque \( \rho \). Comme \( \dim(V) = 1 \), l'action de \( \rho(g) \) est simplement une multiplication par un scalaire. Plus précisément, pour \( \rho(g),\rho(h) : V \to V \), ils existent \( \lambda_g ,\lambda_h \in \mathbb{C} \) tels que :
			\[
			\rho(g)(v) = \lambda_g v \quad \text{et} \quad \rho(h)(v) = \lambda_h v \quad \text{pour tout } v \in V.
			\]
			
			Puisque \( \rho \) est un morphisme de groupes :
			\[
			\rho(gh) = \rho(g)\rho(h) \quad \text{et} \quad \rho(hg) = \rho(h)\rho(g).
			\]
			Mais comme \( \rho(g) \) et \( \rho(h) \) sont des scalaires et la multiplication étant commutative, on a :
			\[
			\rho(gh) = \rho(hg).
			\]
			Ce qui implique \( gh = hg \) et par suite \( G \) est abélien.\\ 
			
			\newpage
			
			\section*{Conclusion}
			\addcontentsline{toc}{section}{Conclusion}
			Ce chapitre a permis de poser les bases essentielles pour appréhender les représentations linéaires des groupes finis, en mettant en lumière leurs définitions, propriétés et applications. Dans un premier temps, nous avons examiné les fondements théoriques en définissant les représentations linéaires et en illustrant ces concepts à travers des exemples concrets. Par ailleurs, la notion de sous-représentations a été approfondie, soulignant leur rôle clé dans l’analyse des structures internes des représentations.
			Ensuite, l’étude des caractères associés à une représentation linéaire a révélé des invariants déterminants qui permettent de caractériser et de différencier les représentations. En effet, ces outils algébriques offrent une approche méthodique et rigoureuse pour classifier efficacement les représentations.
			Enfin, nous avons abordé les représentations linéaires irréductibles, qui forment les éléments constitutifs fondamentaux de cette théorie. Leur importance réside notamment dans leur capacité à décomposer les représentations complexes et à analyser les symétries inhérentes aux structures algébriques.
			Ainsi, ce chapitre constitue une base solide pour une compréhension approfondie des représentations linéaires, tout en ouvrant la voie aux applications plus avancées qui seront développées dans les chapitres suivants, tant dans le cadre des groupes finis que dans des contextes élargis.
			
			
			
		}
		
		
		
		
		\chapter{Produit tensoriel et représentation linéaire d'un produit de groupes finis}
		% Partie avec la taille personnalisée
		{
			\applyfontsize % Application locale de la taille de police 12pt
			
			\section*{Introduction}
			\addcontentsline{toc}{section}{Introduction}
			Le produit tensoriel et les représentations linéaires jouent un rôle crucial dans l'étude des structures algébriques, notamment dans la théorie des groupes et des espaces vectoriels. Cette partie explore les concepts fondamentaux du produit tensoriel, appliqué aux espaces vectoriels et aux groupes, ainsi que leur utilisation dans le cadre des représentations linéaires.Dans un premier temps, nous présenterons la notion de produit tensoriel d’espaces vectoriels, qui permet de construire un nouvel espace vectoriel à partir de deux espaces donnés \cite{greub2012linear}. Cela permet de décrire les interactions entre les éléments de ces espaces, tout en maintenant les propriétés d'une structure linéaire. Nous aborderons également le cas du produit tensoriel arbitraire, c'est-à-dire la généralisation de ce concept à un ensemble quelconque d'espaces vectoriels.Ensuite, nous introduirons la représentation linéaire d’un produit de groupes finis. L'objectif sera de comprendre comment les représentations de plusieurs groupes peuvent être combinées pour former une représentation globale. Nous commencerons par le produit de deux groupes finis \cite{serre1971representation} et élargirons ensuite cette étude au cas d'un produit arbitraire de groupes finis.Enfin, nous traiterons du caractère et de l’irréductibilité d’une représentation linéaire d’un produit de groupes finis. Cette analyse permet de caractériser la manière dont les représentations se décomposent et d'identifier les représentations irréductibles, qui jouent un rôle central dans la théorie des représentations.
			
			
			\section{Produit tensoriel d'espaces vectoriels}
			\subsection{Produit tensoriel de deux espaces vectoriels}
			Soient \( V_1 , V_2 \) et \(V_3\) trois \(\mathbb{K}\)-espaces vectoriels.
			\begin{definition} \cite{greub2012linear}\\
				Une application \(\varphi : V_1 \times V_2 \to V_3 \) est dite bilinéaire si elle satisfait les conditions suivantes :
				\begin{enumerate}[label=\roman*)]
					\item \(\varphi(\lambda v_1 + \mu v_2, w) = \lambda \varphi(v_1, w) + \mu \varphi(v_2, w), \quad \forall v_1, v_2 \in V_1, \, w \in V_2, \ \quad \forall \lambda, \mu \in \Gamma. \)
					
					\item \(\varphi(v, \lambda w_1 + \mu w_2) = \lambda \varphi(v, w_1) + \mu \varphi(v, w_2), \quad \forall v \in V_1, \quad \forall w_1, w_2 \in V_2.\)
				\end{enumerate}
			\end{definition}
			
			\begin{propriety}(\textbf{Propriété universelle}) \cite{greub2012linear}\\
				Soient \( E,F ,G \) et \(H\) quatre \(\mathbb{K}\)-espaces vectoriels et \(\otimes : E \times F \to G \) une application bilinéaire. \\
				On dit que $\otimes$ satisfait la \textit{propriété universelle} si elle vérifie les conditions suivantes :
				\begin{enumerate} [label=\roman*)]
					\item Les vecteurs $x \otimes y$ $(x \in E, \, y \in F)$ engendrent $G$, ou équivalemment, $\mathrm{Im} \, \otimes = G$.
					\item Si \(\varphi : E \times F \to H \) est une application bilinéaire, alors il existe une application linéaire $f : G \to H$ telle que le diagramme suivant commute.
					
					\[
					\xymatrix{
						E \times F \ar[rr]^{\otimes} \ar[dr]_{\varphi} & & G \ar[dl]^{f} \\
						& H &
					}  \quad \quad \text{(1.1)}
					\] 
					Les deux conditions ci-dessus sont équivalentes à l'unique condition suivante :
					\item pour toute application bilinéaire  $\varphi : E \times F \to H$,  il existe une unique application linéaire \(f : G \to H \text{ telle que le diagramme (1.1) commute.}\)
				\end{enumerate}
			\end{propriety}
			
			
			\begin{definition} \cite{greub2012linear}\\
				Le produit tensoriel de deux espaces vectoriels $E$ et $F$ est un couple $(G, \otimes)$, où $\otimes : E \times F \to G$ est une application bilinéaire vérifiant la propriété universelle.\\
				L'espace $G$ est noté $E \otimes F$.
			\end{definition}
			
			\begin{propriety} \cite{greub2012linear}\\
				Le couple \( (E \otimes F, \otimes) \) est unique à un isomorphisme près.
			\end{propriety}
			
			\begin{definition} \cite{axler2024linear}\\
				Si \( \dim E = n \) et \( \dim F = m \), alors la dimension du produit tensoriel \( E \otimes F \) est
				\[
				\dim(E \otimes F) = (\dim E) (\dim F) = nm.
				\]
			\end{definition}
			
			
			\begin{myexample}
				Soient \( V_1 = \mathbb{R}^2 \) et \( V_2 = \mathbb{R}^2 \), avec des bases respectives \( \{ e_1, e_2 \} \) et \( \{ f_1, f_2 \} \).\\
				Le produit tensoriel \( V_1 \otimes V_2 \) est alors un espace vectoriel de dimension \( 4 \), avec une base donnée par les éléments \( \{ e_i \otimes f_j \mid i, j = 1, 2 \} \), soit \( \{ e_1 \otimes f_1, e_1 \otimes f_2, e_2 \otimes f_1, e_2 \otimes f_2 \} \).\\
				Soit \( v_1 = 2e_1 + 3e_2 \in V_1 \) et \( v_2 = 4f_1 + f_2 \in V_2 \). Nous calculons \( v_1 \otimes v_2 \) dans \( V_1 \otimes V_2 \) en utilisant la linéarité du produit tensoriel :
				\begin{align*}
					v_1 \otimes v_2 & = (2e_1 + 3e_2) \otimes (4f_1 + f_2) \\
					& = 2e_1 \otimes (4f_1 + f_2) + 3e_2 \otimes (4f_1 + f_2) \\
					& = 2 \cdot 4 \, e_1 \otimes f_1 + 2 \cdot 1 \, e_1 \otimes f_2 + 3 \cdot 4 \, e_2 \otimes f_1 + 3 \cdot 1 \, e_2 \otimes f_2 \\
					& = 8 e_1 \otimes f_1 + 2 e_1 \otimes f_2 + 12 e_2 \otimes f_1 + 3 e_2 \otimes f_2.
				\end{align*}
				Ainsi, dans la base \( \{ e_1 \otimes f_1, e_1 \otimes f_2, e_2 \otimes f_1, e_2 \otimes f_2 \} \), on peut exprimer \( v_1 \otimes v_2 \) comme :
				
				\[
				v_1 \otimes v_2 = 8 \, e_1 \otimes f_1 + 2 \, e_1 \otimes f_2 + 12 \, e_2 \otimes f_1 + 3 \, e_2 \otimes f_2.
				\]
			\end{myexample}
			
			
			\begin{propriety} \cite{greub2012linear}\\
				Si \(E\) et \(F\) sont deux espaces vectoriels, alors \(E \otimes F \ \cong \ F \otimes E.\)
			\end{propriety}
			
			
			
			\subsection{Produit tensoriel arbitraire d'espaces vectoriels}
			
			\begin{propriety}\textbf{(Propriété universelle.)} \cite{greub2012linear}\\
				Soient \( E_i \) (\( i = 1, \ldots, p \)) des \( p \) espaces vectoriels quelconques, et soit
				\[
				\otimes : E_1 \times \cdots \times E_p \to T
				\]
				une application \( p \)-linéaire. Cette application est dite avoir la \textit{propriété universelle} si elle satisfait les conditions suivantes :
				\begin{enumerate} [label=\roman*)]
					\item Les vecteurs \( x_1 \otimes \cdots \otimes x_p \), (\( x_i \in E_i \)) engendrent \( T \).
					\item Toute application \( p \)-linéaire \( \varphi : E_1 \times \cdots \times E_p \to H \) (\( H \) étant un espace vectoriel quelconque) peut s'écrire sous la forme
					\[
					\varphi(x_1, \ldots, x_p) = f(x_1 \otimes \cdots \otimes x_p)
					\]
					ou \( f : T \to H \) est une application linéaire.
				\end{enumerate}
			\end{propriety}
			
			\begin{definition} \cite{greub2012linear}\\
				Le produit tensoriel des espaces \( E_i \) (\( i = 1, \ldots, p \)) est un couple \( (T, \otimes) \) où
				\[
				\otimes : E_1 \times \cdots \times E_p \to T
				\]
				est une application \( p \)-linéaire avec la propriété universelle. \( T \) est également appelé le produit tensoriel des espaces \( E_i \) et est noté
				\[
				E_1 \otimes \cdots \otimes E_p.
				\]
			\end{definition}
			
			\begin{propriety} \cite{greub2012linear}\\
				Le couple \( (T, \otimes) \) est unique à isomorphisme près.
			\end{propriety}
			
			
			\begin{definition} \cite{greub2012linear}\\
				Si \( E_i \) (\( i = 1, \ldots, p \)) sont des espaces vectoriels sur un corps \( \mathbb{K} \), alors la dimension du produit tensoriel \( E_1 \otimes \cdots \otimes E_p \) est donnée par 
				\[
				\dim(E_1 \otimes \cdots \otimes E_p) = \prod_{i=1}^p \dim(E_i).
				\]
			\end{definition}
			
			
			
			\begin{proposition}\cite{greub2012linear}\\
				Étant donnés trois espaces \(E_1\), \(E_2\), \(E_3\), il existe un isomorphisme linéaire  
				\[
				f : E_1 \otimes E_2 \otimes E_3 \ \cong \ (E_1 \otimes E_2) \otimes E_3
				\]
				tel que
				\[
				f(x \otimes y \otimes z) = (x \otimes y) \otimes z.
				\]
			\end{proposition}
			
			
			\begin{proposition}\textbf{(Généralisation)}  \cite{greub2012linear}\\
				Étant donnés \(p\) espaces vectoriels  \(E_1, E_2, \dots, E_p\), il existe un isomorphisme linéaire  
				\[
				f : E_1 \otimes E_2 \otimes \cdots \otimes E_p \ \cong \ ((\cdots((E_1 \otimes E_2) \otimes E_3) \cdots) \otimes E_p),
				\]
				défini récursivement, tel que pour tout \(x_1 \in E_1, x_2 \in E_2, \dots, x_p \in E_p\),
				\[
				f(x_1 \otimes x_2 \otimes \cdots \otimes x_p) = (((\cdots((x_1 \otimes x_2) \otimes x_3) \cdots) \otimes x_{p-1}) \otimes x_p).
				\]
			\end{proposition}
			
			
			
			\section{Représentation linéaire d'un produit de groupes finis}
			\subsection{Représentation linéaire d'un produit de deux groupes finis}
			\begin{definition} \cite{renard2009groupes} \\
				Soient \( \rho^1: G_1 \to \mathrm{GL}(V_1) \) et \( \rho^2: G_2 \to \mathrm{GL}(V_2) \) les représentations linéaires des groupes \( G_1 \) et \( G_2 \) respectivement. Le produit tensoriel des représentations \( \rho^1 \) et \( \rho^2 \) est une représentation linéaire 
				\[
				\rho_{V_1 \otimes V_2} : G_1 \times G_2 \to \mathrm{GL}(V_1 \otimes V_2)
				\]
				\[
				(g_1, g_2) \mapsto (x \otimes y \mapsto (\rho^1(g_1)(x) \otimes \rho^2(g_2)(y))).
				\]
			\end{definition}
			
			
			\subsection{Représentation linéaire d'un produit arbitraire de groupes finis}
			Soit $(V_i)_{i\in I}$ une famille non vide d’espaces vectoriels sur un même corps commutatif. Soit $(u_i)_{i\in I}$ une famille non vide de vecteurs non nuls tels que, pour tout $i\in I$, le vecteur $u_i$ appartient à $V_i$. Pour tous sous-ensembles finis $J$ et $K$ de $I$ tels que $J\subseteq K$, considérons l'application linéaire injective  
			$$\begin{array}{rlll}
				\varphi_{J,K}: \underset{i\in J}\otimes V_i& \longrightarrow& \underset{i\in K}\otimes V_i\\
				\underset{i\in J}\otimes v_i&\longmapsto& \underset{i\in J}\otimes v_i\otimes(\underset{i\in K\setminus J}\otimes u_i).
			\end{array}
			$$
			Soit $\mathcal{F}(I)$ l'ensemble de tous les sous-ensembles finis de $I$. Il est connu que le système $(\underset{i\in J}\otimes V_i, \varphi_{J,K})_{J\in \mathcal{F}(I)}$ est inductif. La limite inductive du système inductif $(\underset{i\in J}\otimes V_i, \varphi_{J,K})_{J\in \mathcal{F}(I)}$, notée $\underset{i\in J}\otimes^{(u_i)} V_i$, est le produit tensoriel infini des espaces vectoriels $(V_i)_{i\in I}$. Voir \cite{Guichardet}.
			
			\begin{myproposition}\label{prop1}
				Soit $(V_i)_{i\in I}$ une famille non vide d’espaces vectoriels sur un même corps commutatif $F$. Soit $(u_i)_{i\in I}$ une famille non vide de vecteurs non nuls tels que, pour tout $i\in I$, le vecteur $u_i$ appartienne à $V_i$. Pour tous sous-ensembles finis $J$ et $K$ de $I$ tels que $J\subseteq K$, l’application  
				$$\begin{array}{rlll}
					\psi_{J,K}: GL(\underset{i\in K}\otimes V_i)& \longrightarrow& GL(\underset{i\in J}\otimes V_i)\\
					f&\longmapsto& \psi_{J,K} (f)= f_{J,K}
				\end{array}
				$$ 
				avec $f_{J,K}$ défini comme suit : pour tout $f\in GL(\underset{i\in K}\otimes V_i)$ et tout $\underset{i\in J}\otimes v_i\in \underset{i\in J}\otimes V_i$,  
				$$f_{J,K}(\underset{i\in J}\otimes v_i)= \underset{i\in J}\otimes v'_i \text{ si } f((\underset{i\in J}\otimes v_i)\otimes (\underset{i\in K\setminus J}\otimes t_i))= (\underset{i\in J}\otimes v'_i)\otimes (\underset{i\in K\setminus J}\otimes v'_i)= \underset{i\in K}\otimes v'_i,$$  
				pour un certain $\underset{i\in K\setminus J}\otimes t_i\in \underset{i\in K\setminus J}\otimes V_i$, est un homomorphisme de groupes.
			\end{myproposition}
			
			
			\textbf{\underline{Preuve :}}
			\begin{enumerate}
				\item D'une part, prouvons que pour tout sous-ensemble fini $J$ et $K$ de $I$ tels que $J\subseteq K$, l'application $\psi_{J,K}$ est bien définie. Pour cela, nous procéderons selon les étapes suivantes :\\
				$\bullet$ Pour tout $f\in GL(\underset{i\in K}\otimes V_i)$ et tout $\underset{i\in J}\otimes v_i\in \underset{i\in J}\otimes V_i$, si $f((\underset{i\in J}\otimes v_i)\otimes (\underset{i\in K\setminus J}\otimes t_i))= \underset{i\in K}\otimes w_i$ et $f((\underset{i\in J}\otimes v_i)\otimes (\underset{i\in K\setminus J}\otimes t'_i))= \underset{i\in K}\otimes w'_i$ pour tout $\underset{i\in K\setminus J}\otimes t_i$ et $\underset{i\in K\setminus J}\otimes t'_i$ dans $\underset{i\in K\setminus J}\otimes V_i$, alors $\underset{i\in J}\otimes w_i= \underset{i\in J}\otimes w'_i$. En effet, $f_{J,K}(\underset{i\in J}\otimes v_i)= \underset{i\in J}\otimes w_i$ et $f_{J,K}(\underset{i\in J}\otimes v_i)= \underset{i\in J}\otimes w'_i$. Ainsi, $\underset{i\in J}\otimes w_i= \underset{i\in J}\otimes w'_i$.\\
				$\bullet$ Soit $f\in GL(\underset{i\in K}\otimes V_i)$. Pour tout $\underset{i\in J}\otimes v_i\in \underset{i\in J}\otimes V_i$, nous avons $(\underset{i\in J}\otimes v_i)\otimes (\underset{i\in K\setminus J}\otimes u_i)\in \underset{i\in K}\otimes V_i$. Puisque $f$ est bijective, il existe un unique $\underset{i\in K}\otimes w_i\in \underset{i\in K}\otimes V_i$ tel que :
				$$\begin{array}{rlll}
					f(\underset{i\in K}\otimes w_i)&=& f((\underset{i\in J}\otimes w_i)\otimes (\underset{i\in K\setminus J}\otimes w_i))\\
					&=& (\underset{i\in J}\otimes v_i)\otimes (\underset{i\in K\setminus J}\otimes u_i)
				\end{array}$$
				Il en découle, par définition de $f_{J,K}$, que $f_{J,K}(\underset{i\in J}\otimes w_i))= \underset{i\in J}\otimes v_i$. Cela signifie clairement que l'application $f_{J,K}$ est surjective.\\
				$\bullet$ Maintenant, considérons $\underset{i\in J}\otimes v_i\neq \underset{i\in J}\otimes t_i$ dans $\underset{i\in J}\otimes V_i$. Alors, nous avons $(\underset{i\in J}\otimes v_i)\otimes (\underset{i\in K\setminus J}\otimes u_i)\neq (\underset{i\in J}\otimes t_i)\otimes (\underset{i\in K\setminus J}\otimes u_i)$ dans $\underset{i\in K}\otimes V_i$. Et pour tout $f\in GL(\underset{i\in K}\otimes V_i)$, on observe facilement que $f((\underset{i\in J}\otimes v_i)\otimes (\underset{i\in K\setminus J}\otimes u_i))\neq f((\underset{i\in J}\otimes t_i)\otimes (\underset{i\in K\setminus J}\otimes u_i))$ puisque $f$ est injective. Il en résulte que $f_{J,K}(\underset{i\in J}\otimes v_i)\neq f_{J,K}(\underset{i\in J}\otimes t_i)$. Ainsi, $f_{J,K}$ est injective.\\
				
				$\bullet$
				Prouvons que $f_{J,K}$ est linéaire. Considérons $\underset{i\in J}\otimes v_i$ et $\underset{i\in J}\otimes t_i$, deux éléments de $\underset{i\in J}\otimes V_i$, et soit $\alpha\in F$.
				
				Prouvons que :
				\[
				f_{J,K}(\underset{i\in J}\otimes v_i+ \alpha \underset{i\in J}\otimes t_i)= f_{J,K}(\underset{i\in J}\otimes v_i)+ \alpha f_{J,K}(\underset{i\in J}\otimes t_i).
				\]
				Considérons $f_{J,K}(\underset{i\in J}\otimes v_i)= \underset{i\in J}\otimes v'_i$ et $f_{J,K}(\underset{i\in J}\otimes t_i)= \underset{i\in J}\otimes t'_i$, où  
				\[
				f((\underset{i\in J}\otimes v_i)\otimes (\underset{i\in K\setminus J}\otimes h_i))= \underset{i\in K}\otimes v'_i
				\]
				et  
				\[
				f((\underset{i\in J}\otimes t_i)\otimes (\underset{i\in K\setminus J}\otimes k_i))= \underset{i\in K}\otimes t'_i
				\]
				pour certains éléments $\underset{i\in K\setminus J}\otimes h_i$ et $\underset{i\in K\setminus J}\otimes k_i$ de $\underset{i\in K\setminus J}\otimes V_i$. Nous avons :
				
				\[
				\begin{array}{rlll}
					f(((\underset{i\in J}\otimes v_i)\otimes (\underset{i\in K\setminus J}\otimes h_i))+ \alpha ((\underset{i\in J}\otimes t_i)\otimes (\underset{i\in K\setminus J}\otimes k_i)))=\\
					f((\underset{i\in J}\otimes v_i)\otimes (\underset{i\in K\setminus J}\otimes h_i))+ \alpha f((\underset{i\in J}\otimes t_i)\otimes (\underset{i\in K\setminus J}\otimes k_i))
				\end{array}
				\]
				puisque $f$ est linéaire. Et
				\[
				\begin{array}{rlll}
					f((\underset{i\in J}\otimes v_i)\otimes (\underset{i\in K\setminus J}\otimes h_i))+ \alpha f((\underset{i\in J}\otimes t_i)\otimes (\underset{i\in K\setminus J}\otimes k_i))&=& (\underset{i\in K}\otimes v'_i)+(\underset{i\in K}\otimes \alpha t'_i))\\
					&=& \underset{i\in K}\otimes (v'_i+ \alpha t'_i)\\
					&=& (\underset{i\in J}\otimes (v'_i+ \alpha t'_i))\otimes (\underset{i\in K\setminus J}\otimes (v'_i+ \alpha t'_i)) 
				\end{array}
				\]
				De plus,
				\[
				\begin{array}{rlll}
					f(((\underset{i\in J}\otimes v_i)\otimes (\underset{i\in K\setminus J}\otimes h_i))+ \alpha ((\underset{i\in J}\otimes t_i)\otimes (\underset{i\in K\setminus J}\otimes k_i)))&=& f(\underset{i\in J}\otimes (v_i + \alpha t_i)+\underset{i\in K\setminus J}\otimes (h_i+ \alpha k_i))
				\end{array}
				\]
				Ainsi,
				\[
				\begin{array}{rlll}
					f_{J,K}(\underset{i\in J}\otimes (v_i+ \alpha t_i))&=& \underset{i\in J}\otimes (v'_i+ \alpha t'_i)\\
					&=& (\underset{i\in J}\otimes v'_i)+ \alpha (\underset{i\in J}\otimes t'_i))\\
					&=& f_{J,K}(\underset{i\in J}\otimes v_i)+ \alpha f_{J,K}(\underset{i\in J}\otimes t_i) 
				\end{array}
				\]
				Par conséquent, $f_{J,K} \in GL(\underset{i\in K}\otimes V_i)$ pour tout $f \in GL(\underset{i\in K}\otimes V_i)$. \\  
				Ainsi, $\psi_{J,K}$ est bien défini.
				
				
				\item Prouvons maintenant que $\psi_{J,K}$ est un homomorphisme de groupe qui est surjectif. 
				Soit $g \in GL(\underset{i\in J}\otimes V_i)$. Il est clair que 
				$f = g\otimes 1_{\underset{i\in K\setminus J}\otimes V_i}$ appartient à 
				$GL(\underset{i\in K}\otimes V_i)$.  
				
				Pour tout $\underset{i\in J}\otimes v_i \in \underset{i\in J}\otimes V_i$, on a  
				$(\underset{i\in J}\otimes v_i)\otimes (\underset{i\in K\setminus J}\otimes u_i) 
				\in \underset{i\in K}\otimes V_i$, et  
				
				\[
				f((\underset{i\in J}\otimes v_i)\otimes (\underset{i\in K\setminus J}\otimes u_i))= 
				g(\underset{i\in J}\otimes v_i)\otimes (\underset{i\in K\setminus J}\otimes u_i).
				\]
				Ainsi, $f_{J,K} (\underset{i\in J}\otimes v_i) = g(\underset{i\in J}\otimes v_i)$ et  
				$\psi_{J,K} (f) = g$. Cela montre que $\psi_{J,K}$ est surjectif.  
				
				Il reste à prouver que $\psi_{J,K}$ est un homomorphisme de groupe. Considérons  
				$f, g \in GL(\underset{i\in J}\otimes V_i)$. Nous devons montrer que  
				\[
				\psi_{J,K}(f\circ g) = \psi_{J,K}(f) \circ \psi_{J,K}(g).
				\]
				En effet, $(f\circ g)\otimes 1_{\underset{i\in K\setminus J}\otimes V_i} 
				\in GL(\underset{i\in K}\otimes V_i)$ et, comme vu précédemment,  
				
				\[
				\psi_{J,K}((f\circ g)\otimes 1_{\underset{i\in K\setminus J}\otimes V_i}) 
				= f\circ g = \psi_{J,K}(f) \circ \psi_{J,K}(g).
				\]
				Cela conclut la démonstration.
				
			\end{enumerate}
			
			
			\begin{myproposition}\label{prop2}
				Le système $\big(GL(\underset{i\in J}\otimes V_i),\psi_{J,K}\big)$ est projectif avec une limite projective $\underset{\overleftarrow{J\in \mathcal{F}(I)}}\lim GL(\underset{i\in J}\otimes V_i)$, où $\mathcal{F}(I)$ est l'ensemble des sous-ensembles finis de $I$.
			\end{myproposition}
			
			\textbf{\underline{Preuve :}}
			\begin{enumerate}
				\item Montrons que $\psi_{J,J}= id_{GL(\underset{i\in J}\otimes V_i)}$, c'est-à-dire que $\psi_{J,J}(f)= f$ pour tout $f\in GL(\underset{i\in J}\otimes V_i)$. Soit $f\in GL(\underset{i\in J}\otimes V_i)$ et soient $\underset{i\in J}\otimes v_i$ et $\underset{i\in J}\otimes v'_i$ deux éléments de $\underset{i\in J}\otimes V_i$ tels que $f(\underset{i\in J}\otimes v_i)= \underset{i\in J}\otimes v'_i$. Par définition de $\psi_{J,J}(f)$, on a :
				$$
				\psi_{J,J}(f)(\underset{i\in J}\otimes v_i)= f_{J,J} (\underset{i\in J}\otimes v_i)= \underset{i\in J}\otimes v'_i.
				$$
				Par conséquent, $f_{J,J} (\underset{i\in J}\otimes v_i)= f(\underset{i\in J}\otimes v_i)$. Il est donc clair que $\psi_{J,J}(f)= f$, comme voulu.\\ 
				
				\item Soient $J, K$ et $L$ des sous-ensembles finis de $I$ tels que $J\subseteq K$ et $K\subseteq L$. Montrons que le diagramme suivant commute : 
				$$
				\xymatrix{  GL(\underset{i\in L}\otimes V_i)\ar@{->}[rr]^{\psi_{K,L}}\ar@{->}[rrdd]^{\psi_{J,L}}&& GL(\underset{i\in K}\otimes V_i)\ar@{->}[dd]^{\psi_{J,K}} \\
					&&\\
					& & GL(\underset{i\in J}\otimes V_i)  }
				$$
				C'est-à-dire que pour tout $f\in GL(\underset{i\in L}\otimes V_i)$ et tout $\underset{i\in J}\otimes v_i\in \underset{i\in J}\otimes V_i$, on a :
				$$
				\psi_{J,K}\circ \psi_{K,L}(f)(\underset{i\in J}\otimes v_i)= \psi_{J,L}(f)(\underset{i\in J}\otimes v_i).
				$$
				Soit $f\in GL(\underset{i\in L}\otimes V_i)$ et soit $\underset{i\in J}\otimes v_i$ un élément de $\underset{i\in J}\otimes V_i$. Posons :
				$$
				\underset{i\in L}\otimes v_i = (\underset{i\in J}\otimes v_i)\otimes (\underset{i\in L\setminus J}\otimes u_i)\in \underset{i\in L}\otimes V_i.
				$$
				Alors, on a :
				$$
				f(\underset{i\in J}\otimes v_i)= f((\underset{i\in J}\otimes v_i)\otimes (\underset{i\in L\setminus J}\otimes u_i))= \underset{i\in L}\otimes v'_i.
				$$
				Il est clair que :
				$$
				\psi_{J,L}(f)(\underset{i\in J}\otimes v_i)= f_{J,L}(\underset{i\in J}\otimes v_i)= \underset{i\in J}\otimes v'_i.
				$$
				En considérant :
				$$
				\underset{i\in L}\otimes v_i = ((\underset{i\in J}\otimes v_i)\otimes (\underset{i\in K\setminus J}\otimes u_i))\otimes (\underset{i\in L\setminus K}\otimes u_i),
				$$
				nous voyons que :
				$$
				f_{K,L}((\underset{i\in J}\otimes v_i)\otimes (\underset{i\in K\setminus J}\otimes u_i))= (\underset{i\in J}\otimes v'_i)\otimes (\underset{i\in K\setminus J}\otimes v'_i).
				$$
				En utilisant à nouveau la définition de $f_{J,K}$, il s'ensuit que :
				$$
				\psi_{J,K}\circ \psi_{K,L}(f)(\underset{i\in J}\otimes v_i)= \psi_{J,K}(f_{K,L})(\underset{i\in J}\otimes v_i)= \underset{i\in J}\otimes v'_i.
				$$
				Ainsi, nous avons :
				$$
				\psi_{J,K}\circ \psi_{K,L}(f)(\underset{i\in J}\otimes v_i)= \psi_{J,L}(f)(\underset{i\in J}\otimes v_i).
				$$
				Ceci montre que le diagramme ci-dessus commute, puisque $f\in GL(\underset{i\in L}\otimes V_i)$ et $\underset{i\in L}\otimes v_i\in \underset{i\in L}\otimes V_i$ sont choisis arbitrairement.\\ 
				
				Ainsi, le système $\big(GL(\underset{i\in J}\otimes V_i),\psi_{J,K}\big)$ est projectif.\\ 
				
				Enfin, en utilisant le fait que les groupes $GL(\underset{i\in J}\otimes V_i)$ sont des objets de la catégorie $Grp$ des groupes, qui est complète, il s'ensuit que la limite du système projectif $\big(GL(\underset{i\in J}\otimes V_i),\psi_{J,K}\big)$ existe. La proposition est donc démontrée.
			\end{enumerate}
			
			
			\begin{myproposition}\label{prop3} 
				Soit $J$ un sous-ensemble fini d'un ensemble $I$ et $(G_i)_{i\in J}$ une famille finie de groupes finis. Soit $(V_i)_{i\in J}$ une famille finie d'espaces vectoriels de dimension finie sur le même corps $F$ et pour chaque $i \in J$, soit $\varphi_i: G_i\rightarrow GL(V_i)$ une représentation linéaire de $G_i$ dans $V_i$. Alors, le produit tensoriel $\varphi_{J}= \underset{i\in J}\otimes \varphi_i$ des applications $(\varphi_i)_{i\in J}$ défini par : 
				$$\begin{array}{llll}
					\varphi_{J}= \underset{i\in J}\otimes \varphi_i: \underset{i\in J}\Pi G_i&\longrightarrow& GL(\underset{i\in J}\otimes V_i)\\ 
					(g_i)_{i\in J}&\longmapsto& \underset{i\in J}\otimes \varphi_i ((g_i)_{i\in J})= \underset{i\in J}\otimes \varphi_i(g_i)                                                                                                                                                                                                         
				\end{array}
				$$ 
				avec
				$$\begin{array}{rlll} 
					\underset{i\in J}\otimes \varphi_i(g_i): \underset{i\in J}\otimes V_i&\longrightarrow& \underset{i\in J}\otimes V_i\\
					\underset{i\in J}\otimes v_i&\longmapsto&(\underset{i\in J}\otimes \varphi_i(g_i))(\underset{i\in J}\otimes v_i)=\underset{i\in J}\otimes \varphi_i(g_i)(v_i)
				\end{array} 
				,$$ est une représentation linéaire du produit direct fini $\underset{i\in J}\prod G_i$ des groupes $(G_i)_{i\in J}$ dans le produit tensoriel fini $\underset{i\in J}\otimes V_i$.
			\end{myproposition}
			
			\textbf{\underline{Preuve :}}\\
			Pour tout ensemble fini $J$, $i\in J$ et tout $g_{i}\in G_i$, on a $\varphi_i (g_i)\in GL(V_i)$. Il en découle, par la définition du produit tensoriel d'applications, que $\underset{i\in J}\otimes \varphi_i(g_i)\in GL(\underset{i\in J}\otimes V_i)$. On obtient par induction que $\varphi_{J}= \underset{i\in J}\otimes \varphi_i$ est un homomorphisme de groupes. Ainsi, $\varphi_{J}$ est une représentation linéaire du produit direct fini $\underset{i\in J}\prod G_i$ des groupes $(G_i)_{i\in J}$ dans l'espace vectoriel $\underset{i\in J}\otimes V_i$ comme voulu, et la démonstration est terminée.
			
			
			\begin{myproposition}
				Soient $(I, \leq)$ un ensemble dirigé et $(G_i)_{i\in I}$ une famille non vide de groupes finis. Pour tout sous-ensemble fini $J$ de $I$, définissons $G_J = \underset{i\in J}\prod G_i$. Si $J$ et $K$ sont des sous-ensembles de $I$ tels que $J\subseteq K$, alors nous considérons la projection :
				$$\begin{array}{rlll}\varphi_{J, K}: G_K&\longrightarrow &G_J\\
					(g_i)_{i\in K}&\longmapsto&(g_i)_{i\in J}.
				\end{array}
				$$ Le système $(G_J,\varphi_{J, K})$ est projectif avec limite projective $\underset{i\in I} \prod G_i$.
			\end{myproposition}
			
			\textbf{\underline{Preuve :}}\\
			Soit $(I, \leq)$ un ensemble dirigé et $(G_i)_{i\in I}$ une famille non vide de groupes finis. Pour tout sous-ensemble fini $J$ de $I$, définissons $G_J = \prod_{i \in J} G_i$. Nous allons montrer que le système $(G_J, \varphi_{J, K})$ est un système projectif et que sa limite projective est $\prod_{i \in I} G_i$.
			
			\textbf{\underline{Vérification du système projectif}}\\
			Le système projectif est défini par les ensembles $G_J$ indexés par les sous-ensembles finis $J \subseteq I$, et les applications de transition $\varphi_{J,K}$ définies pour $J \subseteq K$ par :
			\[
			\varphi_{J, K}: G_K \to G_J, \quad (g_i)_{i \in K} \mapsto (g_i)_{i \in J}.
			\]
			Les homomorphismes $\varphi_{J,K}$ sont bien définis et sont clairement des homomorphismes de groupes car ils consistent en des projections sur un sous-produit de groupes.\\
			
			\textbf{\underline{Identité :}} \\
			Pour tout $J$, on a trivialement $\varphi_{J, J} = \operatorname{id}_{G_J}$.\\
			
			\textbf{\underline{Compatibilité avec l'inclusion :}} \\
			Supposons $J \subseteq K \subseteq L$ et montrons que 
			\[
			\varphi_{J, L} = \varphi_{J, K} \circ \varphi_{K, L}.
			\]
			Pour tout élément $g = (g_i)_{i \in L} \in G_L$, on a :
			\[
			\varphi_{K, L}(g) = (g_i)_{i \in K} \in G_K,
			\]
			puis
			\[
			\varphi_{J, K}(\varphi_{K, L}(g)) = \varphi_{J, K}((g_i)_{i \in K}) = (g_i)_{i \in J}.
			\]
			Par définition de $\varphi_{J, L}$, nous avons bien $\varphi_{J, L}(g) = (g_i)_{i \in J}$. Ainsi, la compatibilité est vérifiée.\\
			Ces deux propriétés assurent que $(G_J, \varphi_{J, K})$ forme bien un système projectif.\\
			
			
			\textbf{\underline{Détermination de la limite projective}}\\
			La limite projective d'un tel système est définie comme :
			\[
			\underset{\overleftarrow{J \in \mathcal{F}(I)}}\lim G_J = \left\{ (g_i)_{i \in I} \in \prod_{i \in I} G_i \mid \forall J \subseteq K, \varphi_{J, K}((g_i)_{i \in K}) = (g_i)_{i \in J} \right\}.
			\]
			Mais ici, les projections sont simplement les projections naturelles de $\prod_{i \in I} G_i$ sur les sous-produits indexés par les parties finies de $I$. Ainsi, tout élément de $\prod_{i \in I} G_i$ satisfait trivialement la condition de compatibilité, donc :
			\[
			\underset{\overleftarrow{J \in \mathcal{F}(I)}}\lim G_J = \prod_{i \in I} G_i.
			\]
			Par construction, les projections $\varphi_J : \prod_{i \in I} G_i \to G_J$ définies par $(g_i)_{i \in I} \mapsto (g_i)_{i \in J}$ vérifient $\varphi_{J, K} \circ \varphi_K = \varphi_J$, montrant que $\prod_{i \in I} G_i$ satisfait l'universalité de la limite projective.\\
			Il en resulte que $(G_J, \varphi_{J, K})$ définit bien un système projectif, et que sa limite projective est $\prod_{i \in I} G_i$, ce qui conclut la preuve.
			
			\begin{mytheorem} \label{th1}
				Soit \( (G_i)_{i \in I} \) une famille non vide de groupes et \( (\varphi_i)_{i \in I} \) une famille de représentations linéaires, où chaque  
				\[
				\varphi_i : G_i \to \operatorname{GL}(V_i)
				\]
				est une représentation du groupe \( G_i \) sur un espace vectoriel \( V_i \).  
				Alors, la représentation induite sur le produit tensoriel des espaces \( V_i \) est donnée par :
				\[
				\varphi = \bigotimes_{i \in I} \varphi_i : \prod_{i \in I} G_i \longrightarrow \operatorname{GL} \left( \bigotimes_{i \in I} V_i \right).
				\]
			\end{mytheorem}
			
			
			Pour faire cette preuve, nous commençons par démontrer le lemme suivant :
			\begin{mylemma}\label{lem1}
				Soit $(V_i)_{i\in I}$ une famille non vide de vecteurs de dimension finie sur le même corps $F$ et soit $\underset{i\in I}\otimes^{(u_i)} V_i$ le produit tensoriel infini des espaces vectoriels $(V_i)_{i\in I}$ où chaque $u_i$ ($i\in I$) est un vecteur non nul de $V_i$. Soit $\mathcal{F}(I)$ l'ensemble des sous-ensembles finis de $I$. Alors, 
				$$\underset{\overleftarrow{J\in \mathcal{F}(I)}}\lim GL(\underset{i\in J}\otimes V_i)= GL(\underset{i\in I}\otimes^{(u_i)} V_i).$$
			\end{mylemma}
			
			\textbf{\underline{Preuve :}}\\
			Soit $\mathcal{F}(I)$, l'ensemble des sous-ensembles finis de $I$ muni de l'inclusion, vu comme une catégorie. Soit $\mathcal{V}ect_{\otimes V_i}$, l'ensemble des produits tensoriels arbitraires des espaces vectoriels $(V_i)_{i\in I}$, également vu comme une catégorie de la manière suivante : 
			\begin{enumerate}
				\item ses objets sont les produits tensoriels $\underset{i\in J}\otimes V_i$ où $J$ est un sous-ensemble quelconque de $I$ ; et
				\item pour chaque paire ordonnée d'objets $(\underset{i\in J}\otimes V_i, \underset{i\in K}\otimes V_i)$, l'ensemble des morphismes de $\underset{i\in J}\otimes V_i$ vers $\underset{i\in K}\otimes V_i$ est défini comme : 
				$$\mathcal{V}ect_{\otimes V_i} (\underset{i\in J}\otimes V_i, \underset{i\in K}\otimes V_i)= \left\{\begin{array}{ccc}
					\varphi_{J,K}  \text{ si } J\subseteq K\\
					\emptyset  \text{ sinon }. 
				\end{array}
				\right.$$
			\end{enumerate} 
			Considérons la correspondance $\varphi: \mathcal{F}(I)\rightarrow \mathcal{V}ect_{\otimes V_i}$ entre les catégories $\mathcal{F}(I)$ et $\mathcal{V}ect_{\otimes V_i}$, qui associe à chaque objet $J$ de $\mathcal{F}(I)$ l'objet $\underset{i\in J}\otimes V_i$ de $\mathcal{V}ect_{\otimes V_i}$ et à chaque flèche $\alpha: J\rightarrow K$ (c'est-à-dire $J\subseteq K$) une flèche $\varphi(\alpha)= \varphi_{J,K}: \underset{i\in J}\otimes V_i\rightarrow \underset{i\in K}\otimes V_i$ de $\mathcal{V}ect_{\otimes V_i}$. Il est évident que $\varphi$ est un foncteur covariant. En effet, en utilisant le fait que le système $(\underset{i\in J}\otimes V_i, \varphi_{J,K})_{J\in \mathcal{F}(I)}$ est inductif, nous avons :
			\begin{enumerate}
				\item pour tout $J$ dans $\mathcal{F}(I)$ et la flèche $id_{J}: J\rightarrow J$, on a clairement  $\varphi(id_{J})= \varphi_{J, J}= id_{\varphi(J)}$ ;  et
				\item pour tout $J, K, M$ dans $\mathcal{F}(I)$ tel que $J\subseteq K\subseteq M$, et les flèches $\alpha: J\rightarrow K$ et $\beta: K\rightarrow M$, on a $\varphi(\beta\circ \alpha)= \varphi(\beta)\circ \varphi(\alpha)$.
			\end{enumerate}
			Considérons également la correspondance $GL: \mathcal{V}ect_{\otimes V_i}\rightarrow \mathcal{G}rp$ entre les catégories $\mathcal{V}ect_{\otimes V_i}$ et $\mathcal{G}rp$ où $\mathcal{G}rp$ désigne la catégorie des groupes, qui associe à chaque objet $\underset{i\in J}\otimes V_i$ de $\mathcal{V}ect_{\otimes V_i}$ l'objet $GL(\underset{i\in J}\otimes V_i)$ de $\mathcal{G}rp$ et à chaque $J, K,$ dans $\mathcal{F}(I)$ tel que $J\subseteq K$ et à toute flèche $\varphi_{J,K}: \underset{i\in J}\otimes V_i\rightarrow \underset{i\in K}\otimes V_i$, la flèche $GL(\varphi(JK))= \psi(JK): GL(\underset{i\in K}\otimes V_i)\rightarrow GL(\underset{i\in J}\otimes V_i)$ de $\mathcal{V}ect_{\otimes V_i}$. Il est évident que $GL$ est un foncteur contravariant selon la proposition \ref{prop2}.\\Soit $J$, $K$ et $L$ des sous-ensembles finis de $I$ tels que la séquence courte suivante de produits tensoriels finis d'espaces vectoriels sur le même corps et des applications linéaires 
			$$\xymatrix{1\ar@{->}[rr]&&\underset{i\in J}\otimes V_i\ar@{->}[rr]^{f}&&\underset{i\in K}\otimes V_i\ar@{->}[rr]^{g}&&\underset{i\in L}\otimes V_i\ar@{->}[rr] &&1}$$
			soit exacte, c'est-à-dire que $f$ est injective, $g$ est surjective, et $\text{im}(f) = \ker(g)$. Si nous appliquons le foncteur contravariant $GL$ à cette séquence, nous obtenons une séquence de groupes et d'homomorphismes 
			$$\xymatrix{1\ar@{->}[rr]&&GL(\underset{i\in L}\otimes V_i)\ar@{->}[rr]^{G=GL(g)}&&GL(\underset{i\in K}\otimes V_i)\ar@{->}[rr]^{F=GL(f)}&&GL(\underset{i\in J}\otimes V_i)\ar@{->}[rr] &&1}.$$
			Cette nouvelle séquence courte est également exacte. \\ 
			
			\textbf{\underline{Preuve} :}\\
			
			Nous démontrons que la suite suivante est exacte :
			
			\[
			\begin{tikzcd}
				1 \arrow[r] & GL(\bigotimes_{i\in L} V_i) \arrow[r, "G"] & GL(\bigotimes_{i\in K} V_i) \arrow[r, "F"] & GL(\bigotimes_{i\in J} V_i) \arrow[r] & 1
			\end{tikzcd}
			\]
			
			où $G$ et $F$ sont définis en fonction des restrictions aux sous-ensembles d'indices.\\
			
			\textbf{\underline{Montrons que \(G\) est injective}:}\\
			
			Par définition,
			
			\[
			\ker(G) = \{ \varphi_{L} \in GL(\bigotimes_{i\in L} V_i) \mid G(\varphi_{L}) = \operatorname{id}_{\bigotimes_{i\in K} V_i} \}.
			\]
			
			Soient \(  \varphi_{L} , \varphi{'}_{L} \in GL(\bigotimes_{i\in L} V_i) \) tels que \( \varphi_{L} \neq \varphi{'}_{L} \).\\
			Montrons que \( G(\varphi_{L}) \neq G(\varphi{'}_{L})\).\\
			
			Soit 
			\[  (\underset{i\in K}\otimes v_i) \otimes(\underset{i\in L\setminus K}\otimes v_i) \in \bigotimes_{i\in L} V_i \]
			tel que \\
			
			\[
			\varphi_{L}( (\underset{i\in K}\otimes v_i) \otimes(\underset{i\in L\setminus K}\otimes v_i)) =  (\underset{i\in K}\otimes v{'}_i) \otimes(\underset{i\in L\setminus K}\otimes v{'}_i)
			\]
			et \\
			
			\[
			\varphi{'}_{L}( (\underset{i\in K}\otimes v_i) \otimes(\underset{i\in L\setminus K}\otimes v_i)) =  (\underset{i\in K}\otimes v{''}_i) \otimes(\underset{i\in L\setminus K}\otimes v{''}_i)
			\]
			
			.\\
			
			On a :
			\[
			G(\varphi_{L}) ((\underset{i\in K}\otimes v_i) \otimes(\underset{i\in L\setminus K}\otimes v_i)) = G((\underset{i\in K}\otimes v{'}_i) \otimes(\underset{i\in L\setminus K}\otimes v{'}_i)) = \underset{i\in K}\otimes v{'}_i \quad (*)
			\]
			
			
			\[
			G(\varphi_{L}) ((\underset{i\in K}\otimes v_i) \otimes(\underset{i\in L\setminus K}\otimes v_i)) = G((\underset{i\in K}\otimes v{''}_i) \otimes(\underset{i\in L\setminus K}\otimes v{''}_i)) = \underset{i\in K}\otimes v{''}_i \quad (**)
			\]
			
			\begin{align*}
				\varphi_{L} \neq \varphi'_{L}  
				&\Longrightarrow  
				\varphi_{L} (( \underset{i\in K}\otimes v_i) \otimes(\underset{i\in L\setminus K}\otimes v_i )) \neq 
				\varphi'_{L} (( \underset{i\in K}\otimes v_i) \otimes(\underset{i\in L\setminus K}\otimes v_i )) \\
				&\Longrightarrow  
				(\underset{i\in K}\otimes v{'}_i) \otimes(\underset{i\in L\setminus K}\otimes v{'}_i) \neq 
				(\underset{i\in K}\otimes v{''}_i) \otimes(\underset{i\in L\setminus K}\otimes v{''}_i) \\
				&\Longrightarrow  
				\underset{i\in K}\otimes v{'}_i \neq \otimes_{i\in K} v''_{i}   \quad (***)
			\end{align*}
			
			
			
			De \( \quad (*),\quad (**) \quad \text{et} \quad (***)\), on a \( G(\varphi_{L}) \neq G(\varphi{'}_{L})\) et par conséquent, \(G\) est injective.
			
			
			\textbf{\underline{Montrons que \(F\) est surjective}:}\\
			
			Soit $\varphi_{J} \in GL(\bigotimes_{i\in J} V_i)$. Nous voulons montrer qu'il existe $\varphi_{K} \in GL(\bigotimes_{i\in K} V_i)$ tel que \(F(\varphi_{K}) = \varphi_{J}.\)\\
			Construisons $\varphi_{K}$ comme suit :
			\[
			\varphi_{K}(v \otimes w) = \varphi_{J}(v) \otimes w, \quad \forall v \in \bigotimes_{i\in J} V_i, \quad w \in \bigotimes_{i\in K \setminus J} V_i.
			\]
			On a: \\
			\(
			F(\varphi_{K}(v \otimes w)) = F(\varphi_{J}(v) \otimes w) =  \varphi_{J}(v)
			\)
			
			Ainsi, $F$ est surjective.\\
			
			
			
			
			\textbf{\underline{Montrons que $\operatorname{im}(G) = \ker(F)$}}\\
			
			\textbf{\underline{Inclusion \( \operatorname{im}(G) \subseteq \ker(F) \)}:}\\
			Soit \(\varphi_{K} \in \operatorname{im}(G) \). Alors, il existe
			\( \varphi_{L} \in GL\left(\bigotimes_{i\in L} V_i\right) \) tel que
			\[
			G(\varphi_{L}) = \varphi_{L,K} = \varphi_{K} \in GL\left(\bigotimes_{i\in K} V_i\right).
			\]
			Nous devons montrer que :
			\[
			F(G(\varphi_{L})) = \operatorname{id}_{\bigotimes_{i\in J} V_i}.
			\]
			Par définition de \( G \) et \( F \), on a :
			\[
			F(G(\varphi_{L})) = F(\varphi_{L,K}) = \varphi_{L,J}.
			\]
			Lorsqu'on restreint  \( \varphi_{L} \) à \( J \), cela implique:
			\[
			\varphi_{L,J} = \operatorname{id}_{\bigotimes_{i\in J} V_i}.
			\]
			Donc \( \varphi_{K} \in \ker(F) \) et par suite 
			\( \operatorname{im}(G) \subseteq \ker(F).\)\\
			
			
			\textbf{\underline{Inclusion \( \ker(F) \subseteq \operatorname{im}(G) \)}:}\\
			
			Soit \( \varphi_{K} \in \ker(F) \), on a :
			\[
			F(\varphi_{K}) = \varphi_{K,J} = \operatorname{id}_{\bigotimes_{i\in J} V_i}.
			\]
			Nous devons prouver qu'il existe \( \varphi_{L} \in GL\left(\bigotimes_{i\in L} V_i\right) \) tel que :
			\[
			G(\varphi_{L}) =  \varphi_{K} .
			\]
			On définit \( \varphi_{L} \) sur \( GL(\bigotimes_{i\in L} V_i) \) en posant :
			\[
			\varphi_{L,K} = \varphi_{K} .
			\]
			Ainsi, par définition de \( G \), on obtient :
			\[
			G(\varphi_{L}) = \varphi_{L,K} = \varphi_{K}.
			\]
			
			Donc \( \ker(F) \subseteq \operatorname{im}(G) \).\\
			
			\textbf{\underline{Preuve du théorème} \ref{th1} :}\\ 
			Soient \( J \) et \( K \) deux sous-ensembles finis de \( I \) tels que \( J \subseteq K \). Le diagramme suivant commute :
			
			$$
			\xymatrix{
				\underset{i\in K} \prod G_i \ar@{->}[rr]^{\varphi_{K}} \ar@{->}[dd]_{\varphi_{_{J,K}}} && GL(\underset{i\in K} \otimes V_i) \ar@{->}[dd]^{\psi_{_{J,K}}} \\
				&& \\
				\underset{i\in J} \prod G_i \ar@{->}[rr]^{\varphi_{J}} && GL(\underset{i\in J} \otimes V_i)
			}
			$$
			
			En effet, soient \( J \) et \( K \) deux sous-ensembles finis de \( I \) tels que \( J \subseteq K \). Soit \( (g_i)_{i \in K} \) un élément de \( \underset{i\in K} \prod G_i \). Il est clair que :
			
			$$
			\psi_{J,K} \circ \varphi_{K} ((g_i)_{i\in K}) = \psi_{J, K}(\varphi_{K} ((g_i)_{i\in J})) = \psi_{J,K}(\underset{i\in J} \otimes \varphi_{i}(g_i))
			$$
			et
			
			$$
			\varphi_{J} \circ \varphi_{J,K} ((g_i)_{i\in K}) = \varphi_{J}(\varphi_{J,K}((g_i)_{i\in K})) = \varphi_{J} ((g_i)_{i\in J}) = \underset{i\in J} \otimes \varphi_{i}(g_i).
			$$
			
			Soit \( \underset{i\in J} \otimes v_i \) un élément de \( \underset{i\in J} \otimes V_i \). Alors \( (\underset{i\in J} \otimes v_i) \otimes (\underset{i\in K \setminus J} \otimes u_i) \in \underset{i\in K} \otimes V_i \). Ainsi,
			
			$$
			\underset{i\in K} \otimes \varphi_{i}(g_i) \left( (\underset{i\in J} \otimes v_i) \otimes (\underset{i\in K \setminus J} \otimes u_i) \right) = (\underset{i\in J} \otimes \varphi_{i}(g_i)(v_i)) \otimes (\underset{i\in K \setminus J} \otimes \varphi_{i}(g_i)(u_i)).
			$$
			
			Cela découle de la définition de \( \psi_{J,K} \) que :
			
			$$
			\psi_{J,K}(\underset{i\in K} \otimes \varphi_{i}(g_i))(\underset{i\in J} \otimes v_{i}) = \underset{i\in J} \otimes \varphi_{i}(g_i)(v_i) = \underset{i\in J} \otimes \varphi_{i}(g_i)(\underset{i\in J} \otimes v_i).
			$$
			
			Par conséquent, le diagramme ci-dessus commute. Ainsi, d'après \cite{deschamps}, Proposition 53 et Lemme \ref{lem1}, il existe un homomorphisme de groupes :
			
			$$
			\varphi : \underset{i\in I} \prod G_i \longrightarrow GL(\underset{i\in I} \otimes^{(u_i)} V_i),
			$$
			
			qui est une représentation linéaire du produit direct infini \( \underset{i\in I} \prod G_i \) des groupes \( G_i \) dans \( \underset{i\in I} \otimes^{(u_i)} V_i \). Le théorème est donc démontré.
			
			
			\subsection{Caractère d'une représentation linéaire d'un produit de groupes}
			
			\begin{definition} \cite{serre1971representation}\\
				Soient \( G_1 \) et \( G_2 \) des groupes finis, et \( \rho_1 \) et \( \rho_2 \) des représentations linéaires respectives de \( G_1 \) et \( G_2 \), avec \( \chi_1 \) et \( \chi_2 \) les caractères associés à ces représentations. Le \textit{caractère} \( \chi \) du produit tensoriel des représentations \( \rho_1 \otimes \rho_2 \) de \( G_1 \times G_2 \) est défini par la formule :
				
				\[
				\chi(g_1, g_2) = \chi_1(g_1) \chi_2(g_2),
				\]
				pour tout \( g_1 \in G_1 \) et \( g_2 \in G_2 \).
			\end{definition}
			
			\begin{myproposition}
				Soit \( \varphi_J \) la représentation définie par :
				\[
				\varphi = \underset{i \in J}{\bigotimes} \varphi_i : \prod_{i \in I} G_i \longrightarrow GL\left( \underset{i \in J}{\bigotimes} V_i \right)
				\]
				où \( \varphi_i : G_i \to GL(V_i) \) est une représentation de chaque groupe \( G_i \) et \( V_i \) est l'espace vectoriel associé.\\
				Le caractère $\chi_{\varphi_J}$ de la représentation $\varphi_J$ du produit direct fini $\underset{i\in J}\prod G_i$ est donné par :
				\[
				\chi_{\varphi_J}((g_i)_{i\in J}) = \prod_{i\in J} \chi_{\varphi_i}(g_i), \quad \forall (g_i)_{i\in J} \in \underset{i\in J}\prod G_i,
				\]
				où $\chi_{\varphi_i}$ est le caractère de la représentation $\varphi_i$ du groupe $G_i$.
			\end{myproposition}
			
			\textbf{\underline{Preuve} :}\\
			Par définition du caractère d'une représentation, on a :
			\[
			\chi_{\varphi_J}((g_i)_{i\in J}) = \mathrm{Tr}(\varphi_J((g_i)_{i\in J})).
			\]
			Or, par définition du produit tensoriel de représentations, on a :
			\[
			\varphi_J((g_i)_{i\in J}) = \bigotimes_{i\in J} \varphi_i(g_i).
			\]
			Comme la trace du produit tensoriel de matrices est le produit des traces, il vient que :
			\[
			\mathrm{Tr}(\varphi_J((g_i)_{i\in J})) = \prod_{i\in J} \mathrm{Tr}(\varphi_i(g_i)).
			\]
			D'où :
			\[
			\chi_{\varphi_J}((g_i)_{i\in J}) = \prod_{i\in J} \chi_{\varphi_i}(g_i).
			\]
			Ceci prouve la proposition.
			
			\begin{myproposition}
				Soit \( \varphi_I \) la représentation définie par :
				\[
				\varphi = \underset{i \in I}{\bigotimes} \varphi_i : \prod_{i \in I} G_i \longrightarrow GL\left( \underset{i \in I}{\bigotimes} V_i \right)
				\]
				où \( \varphi_i : G_i \to GL(V_i) \) est une représentation de chaque groupe \( G_i \) et \( V_i \) est l'espace vectoriel associé.
				
				Le caractère \( \chi_{\varphi_I} \) de la représentation \( \varphi_I \) du produit direct infini \( \prod_{i \in I} G_i \) est donné par :
				\[
				\chi_{\varphi_I}((g_i)_{i\in I}) = \prod_{i \in I} \chi_{\varphi_i}(g_i), \quad \forall (g_i)_{i\in I} \in \prod_{i\in I} G_i,
				\]
				où \( \chi_{\varphi_i} \) est le caractère de la représentation \( \varphi_i \) du groupe \( G_i \).
			\end{myproposition}
			
			\textbf{\underline{En effet} :} \\
			La représentation \( \varphi_I \) est définie comme une représentation induite du produit direct infini \( \prod_{i \in I} G_i \), construite à partir des représentations \( \varphi_i : G_i \to GL(V_i) \) de chaque groupe \( G_i \). Le produit tensoriel \( \bigotimes_{i \in I} V_i \) forme l'espace vectoriel associé à cette représentation induite, où l'action de chaque élément \( (g_i)_{i \in I} \) de \( \prod_{i \in I} G_i \) se décompose de manière indépendante sur les facteurs \( V_i \).
			
			Le caractère \( \chi_{\varphi_I} \) de \( \varphi_I \) appliqué à \( (g_i)_{i \in I} \in \prod_{i \in I} G_i \) est donné par la trace de l'élément \( (g_i)_{i \in I} \) dans \( \bigotimes_{i \in I} V_i \). En raison de la structure produit de \( \prod_{i \in I} G_i \) et des propriétés des représentations induites, cette trace se factorise en un produit des traces des éléments \( g_i \) sur les espaces \( V_i \). Ainsi, le caractère global \( \chi_{\varphi_I} \) est le produit des caractères \( \chi_{\varphi_i}(g_i) \) des représentations \( \varphi_i \), ce qui permet de décomposer le caractère global en un produit des caractères locaux.
			
			
			\subsection{Irréductibilité des représentations d'un produit de groupes}
			\begin{theorem} \cite{serre1971representation}
				\begin{enumerate} [label=\roman*)]
					\item Si \( \rho^1 : G_1 \to \mathrm{GL}(V_1) \) et \( \rho^2 : G_2 \to \mathrm{GL}(V_2) \) sont des représentations irréductibles de \( G_1 \) et \( G_2 \) respectivement, alors le produit tensoriel \( \rho^1 \otimes \rho^2 \) est une représentation irréductible de \( G_1 \times G_2 \).
					\item Chaque représentation irréductible de \( G_1 \times G_2 \) est isomorphe à un produit tensoriel \( \rho^1 \otimes \rho^2 \), où \( \rho^1 \) est une représentation irréductible de \( G_1 \) et \( \rho^2 \) est une représentation irréductible de \( G_2 \).
				\end{enumerate}
			\end{theorem}
			
			
			\begin{myremark}
				Les représentations irréductibles de \( G = \prod_{i \in I} G_i \) sont les caractères, c'est-à-dire les homomorphismes de groupes \( \chi : G \to \mathbb{C}^* \).
			\end{myremark}
			
			
			
			
			\begin{mylemma} \label{prop3}
				Soit \( (G_i)_{i \in I} \) une famille non vide de groupes finis abéliens et \( (\varphi_i)_{i \in I} \) une famille de représentations linéaires, où chaque  
				\[
				\varphi_i : G_i \to \operatorname{GL}(V_i)
				\]
				est une représentation du groupe \( G_i \) sur un espace vectoriel \( V_i \).  
				Alors, toute représentation irréductible de \( G =  \prod_{i \in I} G_i \) est de degré 1.
			\end{mylemma}
			
			\textbf{\underline{Preuve} :}\\   
			Étant donné que chaque \( G_i \) est un groupe abélien, leur produit \( G = \prod_{i \in I} G_i \) est également un groupe abélien.
			Il est connu que toute représentation irréductible d'un groupe abélien est de degré 1. 
			Par conséquent, toute représentation irréductible de \( G = \prod_{i \in I} G_i \) est de degré 1.
			
			\begin{mytheorem} 
				Soit \( (G_i)_{i \in I} \) une famille non vide de groupes finis qui sont d'ordre premier chacun et \( (\varphi_i)_{i \in I} \) une famille de représentations linéaires, où chaque  
				\[
				\varphi_i : G_i \to \operatorname{GL}(V_i)
				\]
				est une représentation du groupe \( G_i \) sur un espace vectoriel \( V_i \).\\
				Soit \( G =  \prod_{i \in I} G_i \)
				\begin{enumerate} [label=\roman*)]
					\item Si chaque \(G_i\) est un groupe cyclique, alors toute représentation irréductible de \( G\) est de degré 1.
					\item Si chaque \(G_i\) est un groupe d'ordre premier, alors toute représentation irréductible de \( G\) est de degré 1.
					\item Si presque toutes les représentations \( \varphi_i \) sont triviales, alors la représentation induite sur le produit tensoriel des espaces \( V_i \)
					\[
					\varphi = \bigotimes_{i \in I} \varphi_i : G = \prod_{i \in I} G_i \longrightarrow \operatorname{GL} \left( \bigotimes_{i \in I} V_i \right),
					\]
					est irréductible si et seulement si chaque représentation \( \varphi_i \) non triviale est irréductible et leur produit tensoriel est irréductible.
				\end{enumerate} 	
			\end{mytheorem}
			
			\textbf{\underline{Preuve} :}
			
			\begin{enumerate} [label=\roman*)]
				\item Chaque \( G_i \) est cyclique d'ordre fini, donc \( G_i \) est abélien.  
				Le produit \( G = \prod_{i \in I} G_i \) est un groupe abélien. par le lemme \ref{prop3},  toute représentation irréductible de \( G = \prod_{i \in I} G_i \) est de degré 1.
				\item Chaque groupe d'ordre premier est cyclique et par (ii), on a le résultat.
				
				\item  \textbf{\underline{Sens direct}}\\
				
				Supposons que \( \varphi \) est irréductible.\\ 		
				\textbf{\underline{Montrons que chaque \( \varphi_i \) non triviale est irréductible} :}\\
				Supposons par l'absurde que pour un certain \( j \in I \), la représentation non triviale \( \varphi_j \) n'est pas irréductible.  Cela signifie qu'il existe un sous-espace propre non trivial \( W_j \subset V_j \) qui est stable sous \( \varphi_j \).\\
				Soit
				\(
				W = \left( \bigotimes_{i \neq j} V_i \right) \otimes W_j \subset \bigotimes_{i \in I} V_i.
				\)
				Ce sous-espace est stable sous \( \varphi \).\\
				\textbf{En effe :}\\
				Soit \( g = (g_i)_{i \in I} \in G = \prod_{i \in I} G_i \). On a
				\[
				\varphi(g)(v) = \varphi((g_i)_{i \in I})\left( \bigotimes_{i \in I} v_i \right) = \bigotimes_{i \in I} \varphi_i(g_i)(v_i).
				\]
				
				Si \( v \in W \), alors \( v = \left( \bigotimes_{i \neq j} v_i \right) \otimes w_j \) avec \( w_j \in W_j \). Il vient que
				\[
				\varphi(g)(v) = \left( \bigotimes_{i \neq j} \varphi_i(g_i)(v_i) \right) \otimes \varphi_j(g_j)(w_j).
				\]
				
				Étant donné que \( W_j \) est stable sous \( \varphi_j \), on a \( \varphi_j(g_j)(w_j) \in W_j \). Par conséquent,
				
				\[
				\varphi(g)(v) \in \left( \bigotimes_{i \neq j} V_i \right) \otimes W_j = W.
				\]
				
				Ainsi, \( W \) est stable sous l'action de \( \varphi \). Cette stabilité contredit l'irréductibilité de \( \varphi \), car un sous-espace propre non trivial stable ne devrait pas exister dans une représentation irréductible.
				
				Ainsi, chaque \( \varphi_i \) non triviale doit être irréductible.\\
				
				\textbf{\underline{Montrons que le produit tensoriel des représentations linéaires non triviales  \( \varphi_i \)  }}\\
				\textbf{\underline{ est irréductible } :}\\			 
				Soit \( J \subset I \) l'ensemble des indices correspondant aux représentations \( \varphi_i \) non triviales.  
				On peut écrire :
				\[
				\bigotimes_{i \in I} V_i = \left( \bigotimes_{i \in J} V_i \right) \otimes \left( \bigotimes_{i \notin J} V_i \right).
				\]
				Puisque \( \bigotimes_{i \notin J} V_i \) est constitué de représentations triviales, il n’affecte pas l'irréductibilité. Par conséquent, l'irréductibilité de \( \varphi \) entraîne l'irréductibilité de \( \bigotimes_{i \in J} \varphi_i \).\\
				
				\textbf{\underline{	Sens réciproque}}\\
				
				\textbf{\underline{Montrons que \( \varphi \) est irréductible. } :}\\	
				Supposons que chaque \( \varphi_i \) non triviale est irréductible et que leur produit tensoriel est irréductible.  
				Nous devons montrer que \( \varphi \) est irréductible.
				
				Considérons un sous-espace stable \( W \subset \bigotimes_{i \in I} V_i \).
				Nous avons :
				\[
				\bigotimes_{i \in I} V_i = \left( \bigotimes_{i \in J} V_i \right) \otimes \left( \bigotimes_{i \notin J} V_i \right).
				\]
				Comme \( \bigotimes_{i \notin J} V_i \) est trivial, tout sous-espace stable sous \( \varphi \) est en réalité stable sous \( \bigotimes_{i \in J} \varphi_i \).  
				L'irréductibilité de \( \bigotimes_{i \in J} \varphi_i \) entraîne l'irréductibilité de \( \varphi \).Par hypothèse, \( \bigotimes_{i \in J} \varphi_i \) est irréductible. Donc, \( W \) doit être soit \( \{0\} \), soit \( \bigotimes_{i \in J} V_i \).  
				Cela montre que \( \varphi \) est irréductible.
			\end{enumerate}
			
			
			
			\section{Traité du cas de $\widehat{\mathbb{Z}}$, le complété profini de $\mathbb{Z}$.}
			Le groupe \( \widehat{\mathbb{Z}} \), complété profini de \( \mathbb{Z} \), est un exemple fondamental en théorie des groupes profinis. Dans cette section, nous analysons ses propriétés, ses représentations linéaires, ses caractères, et ses représentations irréductibles, en exploitant sa structure spécifique.
			
			\begin{definition} \cite{ribes-zalesskii}\\
				Le complété profini $\widehat{\mathbb{Z}}$ est la limite projective du système $(\mathbb{Z}/n\mathbb{Z}, \varphi_{m,n})$, où :
				\begin{enumerate} [label=\roman*)]
					\item $\mathbb{Z}/n\mathbb{Z}$  est le groupe des classes d’équivalence modulo $n$, et
					\item $\varphi_{m,n} : \mathbb{Z}/m\mathbb{Z} \to \mathbb{Z}/n\mathbb{Z}$ est la surjection canonique définie lorsque $n$ divise $m$.
				\end{enumerate}
			\end{definition}
			
			\begin{mynotation}
				\[
				\widehat{\mathbb{Z}} = \varprojlim_{n \in \mathbb{N}^*} \mathbb{Z}/n\mathbb{Z}.
				\]
			\end{mynotation}
			
			
			\begin{definition} \cite{maclane1971categories}\\
				Soit un système projectif d'ensembles \((E_n, \pi_{m,n})\), où :
				\begin{enumerate} [label=\roman*)]
					\item Chaque \(E_n\) est un ensemble indexé par \(n \geq 1\) , et
					\item \(\pi_{m,n} : E_m \to E_n\) est une application de transition, définie pour \(n \mid m\), satisfaisant :
					\begin{itemize} 
						\item $\pi_{n,n} = \text{id}_{E_n}, \quad \text{(identité sur \(E_n\))}$ ,
						\item $\pi_{m,n} \circ \pi_{k,m} = \pi_{k,n}, \quad \text{pour } k \mid m \mid n \quad cohérence des projections.$
					\end{itemize}
					
				\end{enumerate}
				
				\textbf{Une suite cohérente} est une suite \((x_n)_{n \geq 1} \in \prod_{n \geq 1} E_n\) telle que, pour tout \(n \mid m\), on a :
				\[
				\pi_{m,n}(x_m) = x_n.
				\]
				Dans le cas de \(\widehat{\mathbb{Z}}\), une suite \((x_n)_{n \geq 1} \in \prod_{n \geq 1} \mathbb{Z}/n\mathbb{Z}\) est \textbf{cohérente} si, pour tout \(n \mid m\), on a :
				\[
				x_m \mod n = x_n.
				\]
				Autrement dit, les éléments \(x_n \in \mathbb{Z}/n\mathbb{Z}\) doivent être compatibles entre eux via les projections modulo $n$. Cette condition garantit que les suites cohérentes forment une limite projective.Ainsi, 
				\[
				\widehat{\mathbb{Z}} = \left\{ (x_n) \in \prod_{n \geq 1} \mathbb{Z}/n\mathbb{Z} \mid \forall m, n, \text{ si } n \mid m, \text{ alors } x_m \equiv x_n \mod n \right\}.
				\]	
			\end{definition}
			
			\subsection{Propriétés du complété profini $\widehat{\mathbb{Z}}$ } 
			\begin{enumerate} [label=\roman*)] 
				\item $\widehat{\mathbb{Z}}$ est un groupe topologique compact et totalement discontinu  \cite{ribes-zalesskii}.
				
				\item $\widehat{\mathbb{Z}}$ est isomorphe au produit des groupes cycliques finis $\prod_{p \in \mathcal{P}} \mathbb{Z}_p$, où $\mathbb{Z}_p$ est le groupe des entiers $p$-adiques, et $\mathcal{P}$ est l’ensemble des nombres premiers  \cite{ribes-zalesskii}.
				
				\item L’homomorphisme canonique $\mathbb{Z} \to \widehat{\mathbb{Z}}$ est injectif, et l’image de $\mathbb{Z}$ est dense dans $\widehat{\mathbb{Z}}$ pour la topologie profinie  \cite{ribes-zalesskii}.
			\end{enumerate}
			
			
			\begin{myproposition}
				\((\widehat{\mathbb{Z}} , +) \) est un sous-groupe fermé du groupe produit \((\prod_{n \geq 1} \mathbb{Z}/n\mathbb{Z},+)\)  avec \(+\) l'addition composante par composante définie par : \\
				pour tous \( x = (x_n)_{n \geq 1} \) et \( y = (y_n)_{n \geq 1} \) dans \( \widehat{\mathbb{Z}} \), on a:
				\[
				x + y = (x_n + y_n)_{n \geq 1},
				\]
				où \( x_n + y_n \) est l'addition modulo \( n \).\\
			\end{myproposition}
			
			\textbf{\underline{Preuve:}}\\
			
			\textbf{\underline{Montrons que \( \widehat{\mathbb{Z}} \) est un sous-groupe de \( \prod_{n \geq 1} \mathbb{Z}/n\mathbb{Z}\).}}\\
			
			\textbf{La stabilité par addition} : \\ 
			Soient \( (x_n), (y_n) \in \widehat{\mathbb{Z}} \).\\
			Par définition, cela signifie que pour tout $m, n$ avec $n \mid m$, on a :
			\[
			x_m \equiv x_n \, (\text{mod } n) \quad \text{et} \quad y_m \equiv y_n \, (\text{mod } n).
			\]
			Ainsi, $(x_m + y_m) \equiv (x_n + y_n) \, (\text{mod } n)$ pour tout $n \mid m$, ce qui montre que la suite $(x_n + y_n)$ est cohérente. Par conséquent, $(x_n + y_n) \in \widehat{\mathbb{Z}} $, ce qui prouve que $\widehat{\mathbb{Z}} $ est stable par addition.\\
			
			\textbf{La présence de l'élément neutre} : \\
			La suite $(0_n)$, où $0_n$ désigne l'élément neutre dans \( \mathbb{Z}/n\mathbb{Z} \), appartient à $\widehat{\mathbb{Z}}$, car pour tout $m, n$, on a :
			\[
			0_m \equiv 0_n \, (\text{mod } n).
			\]
			Ainsi, $(0_n)$ satisfait la condition de cohérence pour toutes les projections et appartient à $\widehat{\mathbb{Z}}$, ce qui montre la présence de l'élément neutre.\\
			
			\textbf{La stabilité par opposé} :\\
			Si $(x_n) \in \widehat{\mathbb{Z}}$, alors $(-x_n)$ est également cohérent.\\
			En effet, pour tout $m, n$ avec $n \mid m$, on a :
			\[
			-x_m \equiv -x_n \, (\text{mod } n).
			\]
			Cela montre que $(-x_n) \in \widehat{\mathbb{Z}}$, donc $\widehat{\mathbb{Z}}$ est stable par opposé.\\
			Il en resulte que \( \widehat{\mathbb{Z}} \) est un sous-groupe de \( \prod_{n \geq 1} \mathbb{Z}/n\mathbb{Z} \).\\
			
			
			\textbf{\underline{Montrer que \( \widehat{\mathbb{Z}} \) est fermé.}}\\ 
			
			Montrons qes applications \(f_{m,n}\) sont continues.\\
			L'espace \(\prod_{n \geq 1} \mathbb{Z}/n\mathbb{Z}\) est muni de la topologie produit, où chaque facteur \(\mathbb{Z}/n\mathbb{Z}\) est discret. Dans cette topologie :
			\begin{itemize}
				\item Les projections \(p_k : \prod_{n \geq 1} \mathbb{Z}/n\mathbb{Z} \to \mathbb{Z}/k\mathbb{Z}\) sont continues.
				\item Les opérations algébriques (addition, soustraction) dans \(\mathbb{Z}/n\mathbb{Z}\) sont continues car \(\mathbb{Z}/n\mathbb{Z}\) est discret.
			\end{itemize}
			L'application \(f_{m,n}\) se décompose comme suit :
			\[
			f_{m,n}(x) = x_m \mod n - x_n,
			\]
			où :
			\begin{itemize}
				\item \(x \mapsto x_m\) est une projection, donc continue.
				\item \(x_m \mapsto x_m \mod n\) est un homomorphisme de groupes discret, donc continu.
				\item La soustraction dans \(\mathbb{Z}/n\mathbb{Z}\) est continue car \(\mathbb{Z}/n\mathbb{Z}\) est discret.
			\end{itemize}
			Ainsi, \(f_{m,n}\) est une composition d'applications continues, donc continue.
			\[
			\ker(f_{m,n}) = \left\{  (x_n)_n \in \prod_{n \geq 1} \mathbb{Z}/n\mathbb{Z} \mid f_{m,n}(x) = 0 \right\}.
			\]
			On a:
			\[
			\widehat{\mathbb{Z}} = \bigcap_{n \mid m} \ker(f_{m,n}).
			\]
			
			\textbf{En effet:}\\
			\( \widehat{\mathbb{Z}} \subseteq \bigcap_{n \mid m} \ker(f_{m,n}) \) \\
			Soit \( x = (x_n) \in \widehat{\mathbb{Z}} \).\\
			Par définition de \( \widehat{\mathbb{Z}} \), cela signifie que pour tout \( n \mid m \), la suite \( x \) est cohérente, c'est-à-dire que pour chaque \( n \mid m \), on a :
			\[
			x_m \equiv x_n \pmod{n}.
			\]
			Cela implique que la différence \( x_m - x_n \) est divisible par \( n \), c'est-à-dire :
			\[
			x_m - x_n \equiv 0 \pmod{n}.
			\]
			En d'autres termes, \( x \) appartient au noyau de \( f_{m,n} \), c'est-à-dire :
			\[
			f_{m,n}(x) = x_m - x_n \equiv 0 \pmod{n}.
			\]
			Ainsi, \( x \in \ker(f_{m,n}) \) pour chaque \( n \mid m \). Par conséquent, \( x \in \bigcap_{n \mid m} \ker(f_{m,n}) \). Cela montre que :
			\[
			\widehat{\mathbb{Z}} \subseteq \bigcap_{n \mid m} \ker(f_{m,n}).
			\]
			\( \bigcap_{n \mid m} \ker(f_{m,n}) \subseteq \widehat{\mathbb{Z}} \) \\
			Soit \( x = (x_n) \in \bigcap_{n \mid m} \ker(f_{m,n}) \).\\
			Cela signifie que pour chaque \( n \mid m \), on a :
			\[
			f_{m,n}(x) = x_m - x_n \equiv 0 \pmod{n}.
			\]
			En d'autres termes, pour chaque \( n \mid m \), on a :
			\[
			x_m \equiv x_n \pmod{n}.
			\]
			Cela montre que la suite \( (x_n) \) est cohérente, c'est-à-dire que \( x \in \widehat{\mathbb{Z}} \). Par conséquent :
			\[
			\bigcap_{n \mid m} \ker(f_{m,n}) \subseteq \widehat{\mathbb{Z}}.
			\]
			Donc 
			\[
			\widehat{\mathbb{Z}} = \bigcap_{n \mid m} \ker(f_{m,n}).
			\]
			L'application \(f_{m,n}\) est continue, et \(\mathbb{Z}/n\mathbb{Z}\) est discret. Dans un espace topologique, le préimage d'un singleton sous une application continue est un ensemble fermé. Par conséquent, \(f_{m,n}^{-1}(\{0\}) = \ker(f_{m,n})\) est fermé.\\
			L'intersection d'ensembles fermés est fermée, donc \( \widehat{\mathbb{Z}} \) est fermé.\\
			Il en resulte que \( \widehat{\mathbb{Z}} \) est un sous-groupe fermé du groupe produit \(\prod_{n \geq 1} \mathbb{Z}/n\mathbb{Z}\).
			
			\begin{remark}
				\begin{enumerate} [label=\roman*)] \
					\item Chaque \( \mathbb{Z}/n\mathbb{Z} \) est un groupe abélien, car l'addition dans \( \mathbb{Z}/n\mathbb{Z} \) est commutative.
					\item \( \widehat{\mathbb{Z}} \) est un sous-groupe d'un groupe abélient (\(\prod_{n \geq 1} \mathbb{Z}/n\mathbb{Z}\) , +).Donc \( \widehat{\mathbb{Z}} \) est un \textbf{groupe abélien} \cite{schaub1997}.
				\end{enumerate}
			\end{remark}
			
			
			\subsection{Représentations linéaires de $\widehat{\mathbb{Z}}$}
			\begin{myproposition}
				Soit \( N \) un sous-ensemble fini de \( \mathbb{N} \) et soit \( (\mathbb{Z}/n\mathbb{Z})_{n \in N} \) une famille finie de groupes cycliques.  
				Soit \( (V_n)_{n \in N} \) une famille finie non vide d'espaces vectoriels de dimension finie sur un corps \( F \), et pour chaque \( n \in N \), soit 
				\[
				\varphi_n: \mathbb{Z}/n\mathbb{Z} \to \operatorname{GL}(V_n)
				\]
				une représentation linéaire.\\  
				Alors, on a la représentation linéaire :
				\[
				\varphi_N = \bigotimes_{n\in N} \varphi_n: \prod_{n \in N} \mathbb{Z}/n\mathbb{Z} \to \operatorname{GL} \left( \bigotimes_{n \in N} V_n \right)
				\]
				tel que :
				\[
				\varphi_N((g_n)_{n \in N}) = \bigotimes_{n \in N} \varphi_n(g_n).
				\]
				De plus, en passant à la limite inverse, on a la représentation linéaire sur un produit tensoriel infini
				\[
				\varphi_\mathbb{N} = \bigotimes_{n\in \mathbb{N}} \varphi_n: 	\varprojlim_{N \in \mathcal{F}(\mathbb{N})} \prod \mathbb{Z}/n\mathbb{Z} = \prod_{n \in \mathbb{N}} \mathbb{Z}/n\mathbb{Z} \longrightarrow \operatorname{GL}\left( \bigotimes^{(u_n)}_{n \in \mathbb{N}} V_n \right).
				\]
			\end{myproposition}
			
			
			
			\begin{mylemma}
				Si $N, M \in \mathcal{F}(\mathbb{N})$ avec $M \subseteq N$, nous avons une projection naturelle :
				\[
				\pi_{N,M} : \prod_{n \in N} \mathbb{Z}/n\mathbb{Z} \to \prod_{m \in M} \mathbb{Z}/m\mathbb{Z},
				\]
				donnée par la restriction des coordonnées aux indices $M$.\\
				Ces projections sont compatibles, c'est-à-dire 
				\(
				\text{si } L \subseteq M \subseteq N  \text{,} \quad \text{on a :}\\	\pi_{M,L} \circ \pi_{N,M} = \pi_{N,L}.
				\)	
			\end{mylemma}
			
			
			
			\begin{mytheorem}
				L'application
				\[
				\Phi : \widehat{\mathbb{Z}} \to \varprojlim_{N \in \mathcal{F}(\mathbb{N})} \prod_{n \in N} \mathbb{Z}/n\mathbb{Z}
				\]
				définie par :
				\[
				\Phi((x_n)_{n \geq 1}) = (x_N)_{N \in \mathcal{F}(\mathbb{N})}, \quad \text{où } x_N = (x_n)_{n \in N}
				\]
				est un isomorphisme de groupes .
			\end{mytheorem}
			
			\textbf{\underline{Preuve} :}\\
			
			On sait que :
			\[
			\widehat{\mathbb{Z}} = \varprojlim_{n \in \mathbb{N}^*} \mathbb{Z}/n\mathbb{Z}.
			\]
			\[
			\widehat{\mathbb{Z}} = \left\{ (x_n)_n \in \prod_{n \geq 1} \mathbb{Z}/n\mathbb{Z} \mid \forall m, n, \text{ si } n \mid m, \text{ alors } x_m \equiv x_n \mod n \right\}.
			\]
			
			
			\[
			\varprojlim_{N \in \mathcal{F}(\mathbb{N})} \prod_{n \in N} \mathbb{Z}/n\mathbb{Z} = \left\{ (x_N)_{N \in \mathcal{F}(\mathbb{N})} \mid x_N \in \prod_{n \in N} \mathbb{Z}/n\mathbb{Z}, \quad \forall M \subseteq N, \quad \pi_{N,M}(x_N) = x_M \right\}.
			\]
			Autrement dit, chaque $x_N$ est une suite de résidus compatibles.
			
			\textbf{\underline{ Montrons que $\Phi$ est bien définie} :}\\
			Soit $(x_n)_{n \geq 1} \in \widehat{\mathbb{Z}}$  où $x_n \in \mathbb{Z}/n\mathbb{Z}$. On a :
			\[
			x_m = x_n \mod m, \quad \text{si } m \divides n.
			\]
			$\Phi ((x_n)_{n \geq 1}) = (x_N)_{N \in \mathcal{F}(\mathbb{N})}$  où :
			\[
			x_N = (x_n)_{n \in N} \in \prod_{n \in N} \mathbb{Z}/n\mathbb{Z}.
			\]
			Montrons que $(x_N)_{N \in \mathcal{F}(\mathbb{N})} \in \varprojlim_{N \in \mathcal{F}(\mathbb{N})} \prod_{n \in N} \mathbb{Z}/n\mathbb{Z}$, c'est-à-dire que pour tout $M \subseteq N$, on a :
			\[
			\pi_{N,M}(x_N) = x_M.
			\]
			Cela découle directement de la compatibilité des résidus :
			\[
			\pi_{N,M}(x_N) = (x_n)_{n \in M} = x_M.
			\]
			Donc, $\Phi$ est bien définie.
			
			\textbf{\underline{Montrons que $\Phi$ est un homomorphisme de groupes} :}\\
			Soient $x=(x_n)_{n \geq 1}, y=(y_n)_{n \geq 1} \in \widehat{\mathbb{Z}}$. Montrons que :
			\[
			\Phi(x + y) = \Phi(x) + \Phi(y).
			\]
			Dans $\widehat{\mathbb{Z}}$, l’addition est définie par :
			\[
			(x + y)_n = x_n + y_n \mod n.
			\]
			Par définition de $\Phi$, nous avons :
			\[
			\Phi(x + y) = ( (x_n + y_n)_{n \in N} )_{N \in \mathcal{F}(\mathbb{N})} \quad \text{(1)}.
			\]
			Or, dans la limite projective, la somme est définie composante par composante :
			\[
			(x_N)_{N \in \mathcal{F}(\mathbb{N})} + (y_N)_{N \in \mathcal{F}(\mathbb{N})} = ( (x_n + y_n)_{n \in N} )_{N \in \mathcal{F}(\mathbb{N})} \quad \text{(2)}.
			\]	
			Il vient que de (1) et (2) que \(	\Phi(x + y) = \Phi(x) + \Phi(y) \). Donc, $\Phi$ est bien un homomorphisme de groupes.
			
			\textbf{\underline{Montrons que $\Phi$ est injective} :}\\
			Soient $x=(x_n)_{n \geq 1}, y=(y_n)_{n \geq 1} \in \widehat{\mathbb{Z}}$.
			Montrons que :
			\[
			\Phi(x) = \Phi(y) \Rightarrow x = y.
			\]
			Si $\Phi(x) = \Phi(y)$, alors pour tout $N$,
			\[
			x_N = y_N.
			\]
			Comme $x_N = (x_n)_{n \in N}$ et $y_N = (y_n)_{n \in N}$, on a :
			\[
			x_n = y_n, \quad \forall n \in N.
			\]
			Puisque cette égalité est vraie pour tous les ensembles finis $N$, cela implique que :
			\[
			x_n = y_n, \quad \forall n \geq 1.
			\]
			Donc, $x = y$ dans $\widehat{\mathbb{Z}}$, prouvant que $\Phi$ est injective.
			
			\textbf{\underline{Montrons que $\Phi$ est surjective} :}\\
			Nous devons montrer que tout élément de la limite projective $(x_N)_{N \in \mathcal{F}(\mathbb{N})}$ provient d’un élément de $\widehat{\mathbb{Z}}$.\\
			Soit $(x_N)_{N \in \mathcal{F}(\mathbb{N})} \in \varprojlim_{N \in \mathcal{F}(\mathbb{N})} \prod_{n \in N} \mathbb{Z}/n\mathbb{Z}$. Cela signifie que pour tout $M \subseteq N$, nous avons :
			\[
			\pi_{N,M}(x_N) = x_M.
			\]
			Définissons la suite $(x_n)_{n \geq 1}$ par :
			\[
			x_n = x_{\{n\}}.
			\]
			Nous devons vérifier la compatibilité :
			\[
			x_m = x_n \mod m, \quad \text{si } m \divides n.
			\]
			Puisque $(x_N)_{N \in \mathcal{F}(\mathbb{N})} \in \varprojlim_{N \in \mathcal{F}(\mathbb{N})} \prod_{n \in N} \mathbb{Z}/n\mathbb{Z}$, on a :
			\[
			x_m = \pi_{N,\{m\}}(x_N) = x_n \mod m.
			\]
			Ainsi, la suite $(x_n)_{n \geq 1}$ est bien un élément de $\widehat{\mathbb{Z}}$ et son image par $\Phi$ est exactement $(x_N)_{N \in \mathcal{F}(\mathbb{N})}$.
			Cela prouve que $\Phi$ est surjective.\\
			Il en resulte que $\Phi$ est un isomorphisme de groupes. Donc
			\(
			\widehat{\mathbb{Z}} \simeq \varprojlim_{N \in \mathcal{F}(\mathbb{N})} \prod_{n \in N} \mathbb{Z}/n\mathbb{Z}.
			\)
			
			Ainsi, d’après les résultats précédents, nous obtenons la représentation linéaire suivante :
			\[
			\varphi_{\mathbb{N}} : \widehat{\mathbb{Z}} = \prod_{n \in \mathbb{N}} \mathbb{Z}/n\mathbb{Z} \longrightarrow \operatorname{GL} \left( \bigotimes^{(u_n)}_{n \in \mathbb{N}} V_n \right).
			\]
			
			
			\subsection{Caractères de $\widehat{\mathbb{Z}}$}
			
			\begin{mydefinition}
				Un caractère de $\widehat{\mathbb{Z}}$ est un homomorphisme de groupe continu :
				\[
				\chi : \widehat{\mathbb{Z}} \to \mathbb{C}^\times.
				\]
				
			\end{mydefinition}
			
			\begin{remark}
				\begin{enumerate}[label=\roman*)] \
					\item Chaque groupe $\mathbb{Z}/n\mathbb{Z}$ étant cyclique d'ordre $n$, ses caractères sont donnés par :
					\[
					\chi_n(a) = e^{2i\pi a / n}, \quad \forall a \in \mathbb{Z}/n\mathbb{Z}.
					\]
					\item Puisque $\widehat{\mathbb{Z}}$ est la limite projective des $\mathbb{Z}/n\mathbb{Z}$, un tel $\chi$ est entièrement déterminé par ses valeurs sur chaque quotient $\mathbb{Z}/n\mathbb{Z}$.
				\end{enumerate}
			\end{remark}
			
			\begin{theorem} \cite{ribes-zalesskii}\\
				L'ensemble des caractères continus de $\widehat{\mathbb{Z}}$ est isomorphe à $\mathbb{Q}/\mathbb{Z}$ via la dualité de Pontryagin :
				\[
				\operatorname{Hom}_{\text{cont}}(\widehat{\mathbb{Z}}, \mathbb{C}^\times) \simeq \mathbb{Q}/\mathbb{Z}.
				\]
			\end{theorem}
			
			\textbf{\underline{Preuve} :}\\
			
			Définissons une application :
			\[
			\Phi : \mathbb{Q}/\mathbb{Z} \to \operatorname{Hom}_{\text{cont}}(\widehat{\mathbb{Z}}, \mathbb{C}^\times).
			\]
			
			Soit $q \in \mathbb{Q}/\mathbb{Z}$. On peut l'écrire sous la forme $q = \frac{k}{n}$ avec $k \in \mathbb{Z}$ et $n \in \mathbb{N}^*$. Nous définissons alors un caractère :
			\[
			\chi_q : \widehat{\mathbb{Z}} \to \mathbb{C}^\times, \quad \chi_q(a) = e^{2i\pi q a}.
			\]
			
			\underline{Montrons que $\Phi$ est bien définie :}\\
			
			Soit \(q \in \mathbb{Q}/\mathbb{Z} \).\\
			
			\underline{Montrons que $\chi_q$ est un homomorphisme de groupes :}\\
			
			Soient $a, b \in \widehat{\mathbb{Z}}$.\\
			
			L'application exponentielle satisfait la relation :
			\[
			e^{2i\pi q (a + b)} = e^{2i\pi q a} e^{2i\pi q b}.
			\]
			Donc $\chi_q(a + b) = \chi_q(a) \chi_q(b)$, ce qui montre que $\chi_q$ est un homomorphisme de groupes.\\
			
			\underline{Montrons que $\chi_q$ est continu dans la topologie profinie :}\\
			
			La topologie de $\widehat{\mathbb{Z}}$ est la topologie profinie, définie comme la limite inverse des groupes $\mathbb{Z}/n\mathbb{Z}$. Nous avons $\chi_q(a) = e^{2i\pi q a}$. \\
			Comme l'application exponentielle est continue et que $q a$ est bien défini modulo $1$ (puisque $q \in \mathbb{Q}/\mathbb{Z}$), nous obtenons que $\chi_q$ est continu.
			
			
			\underline{Montrons que l'application $\Phi$ est injective :}\\
			
			Soit $q, q' \in \mathbb{Q}/\mathbb{Z}$ tels que $\chi_q = \chi_{q'}$. Cela signifie que :
			\[
			e^{2i\pi q a} = e^{2i\pi q' a}, \quad \forall a \in \widehat{\mathbb{Z}}.
			\]
			Prenons $a = 1 \in \mathbb{Z} \subseteq \widehat{\mathbb{Z}}$. Cela donne :
			\[
			e^{2i\pi q} = e^{2i\pi q'}.
			\]
			Cela implique que $q - q' \in \mathbb{Z}$, donc $q \equiv q' \mod \mathbb{Z}$. Or, par définition de $\mathbb{Q}/\mathbb{Z}$, cela signifie que $q = q'$.\\ 
			Ainsi, $\Phi$ est injective.
			
			\underline{Montrons que l'application $\Phi$ est surjective :}\\
			
			Soit $\chi \in \operatorname{Hom}_{\text{cont}}(\widehat{\mathbb{Z}}, \mathbb{C}^\times)$.  \\
			Nous devons montrer qu'il existe $q \in \mathbb{Q}/\mathbb{Z}$ tel que $\chi = \chi_q$.\\
			Le groupe $\widehat{\mathbb{Z}}$ est profini et engendré topologiquement par $1$. Ainsi, $\chi$ est entièrement déterminé par $\chi(1) \in \mathbb{C}^\times$.\\
			Puisque $\chi$ est un caractère continu et $\widehat{\mathbb{Z}}$ est la limite projective des $\mathbb{Z}/n\mathbb{Z}$, il existe une valeur $\chi(1)$ qui est une racine de l’unité :
			\[
			\chi(1) = e^{2i\pi q}, \quad \text{avec } q \in \mathbb{Q}/\mathbb{Z}.
			\]
			Vérifions que $\chi = \chi_q$\\
			Par construction, la seule possibilité est que $\chi$ soit donné par :
			\[
			\chi(a) = e^{2i\pi q a}.
			\]
			Cela montre que $\chi = \chi_q$ pour un certain $q \in \mathbb{Q}/\mathbb{Z}$. Ainsi, $\Phi$ est surjective.
			
			Il en resulte que $\Phi$ est un isomorphisme de groupes :
			\[
			\operatorname{Hom}_{\text{cont}}(\widehat{\mathbb{Z}}, \mathbb{C}^\times) \simeq \mathbb{Q}/\mathbb{Z}.
			\]
			
			
			\subsection{Représentations irréductibles de $\widehat{\mathbb{Z}}$}
			Chaque $\mathbb{Z}/n\mathbb{Z}$ est un groupe cyclique fini, et ses représentations irréductibles sont bien connues.\\
			$\mathbb{Z}/n\mathbb{Z}$ possède $n$ représentations irréductibles de degré $1$, correspondant aux caractères continus $\chi_k : \mathbb{Z}/n\mathbb{Z} \to \mathbb{C}^*$ définis par :
			\[
			\chi_k(a) = e^{2\pi i k a / n}, \quad k = 0, 1, \dots, n-1.
			\]
			
			
			\begin{proposition}
				\begin{enumerate} [label=\roman*)] \
					\item Toutes les représentations irréductibles de $\widehat{\mathbb{Z}}$ sont de degré $1$.
					\item Soient $\operatorname{Irr}(\widehat{\mathbb{Z}})$ l'ensemble des représentations irréductibles de $\widehat{\mathbb{Z}}$ et $\operatorname{Hom}_{\text{cont}}(\widehat{\mathbb{Z}}, \mathbb{C}^\times)$ l'ensemble de ses caractères continus. Il existe un isomorphisme :
					\[
					\Phi : \operatorname{Irr}(\widehat{\mathbb{Z}}) \to \operatorname{Hom}_{\text{cont}}(\widehat{\mathbb{Z}}, \mathbb{C}^\times)),
					\]
					\item $\widehat{\mathbb{Z}}$ possède une infinité non dénombrable de représentations irréductibles.
					\item Les représentations irréductibles de $\widehat{\mathbb{Z}}$ peuvent être interprétées comme des représentations sur chaque composante $\mathbb{Z}_p$.
				\end{enumerate}
			\end{proposition}
			
			
			\textbf{\underline{Preuve} :}
			\begin{enumerate} [label=\roman*)] 
				\item 
				Les représentations irréductibles des groupes finis abéliens $\mathbb{Z}/n\mathbb{Z}$ sont de degré 1, données par des caractères :
				\[
				\chi_n : \mathbb{Z}/n\mathbb{Z} \to \mathbb{C}^\times .
				\]
				Les représentations irréductibles de $\widehat{\mathbb{Z}}$ sont obtenues comme limites projectives de celles des $\mathbb{Z}/n\mathbb{Z}$, et elles sont donc aussi de degré 1. Toute représentation irréductible de $\widehat{\mathbb{Z}}$ est donc un homomorphisme continu :
				\[
				\widehat{\mathbb{Z}} \to \mathbb{C}^\times.
				\]
				\item Définissons l'application :
				\[
				\Phi : \operatorname{Irr}(\widehat{\mathbb{Z}}) \to \operatorname{Hom}_{\text{cont}}(\widehat{\mathbb{Z}}, \mathbb{C}^\times)).
				\]
				Soit $\rho : \widehat{\mathbb{Z}} \to \operatorname{GL}(V)$ une représentation irréductible. Nous avons établi que toute représentation irréductible de $\widehat{\mathbb{Z}}$ est de \textbf{degré $1$}, ce qui implique que $V$ est un espace vectoriel de dimension $1$. Par conséquent, l'image de $\rho$ est contenue dans le groupe multiplicatif des scalaires :
				\[
				\operatorname{GL}(V) \simeq \mathbb{C}^\times.
				\]
				Ainsi, $\rho$ définit un \textbf{caractère continu} de $\widehat{\mathbb{Z}}$, et on pose :
				\[
				\Phi(\rho) = \rho.
				\]
				Définissons maintenant l'application réciproque :
				\[
				\Psi : \operatorname{Hom}_{\text{cont}}(\widehat{\mathbb{Z}}, \mathbb{C}^\times) \to \operatorname{Irr}(\widehat{\mathbb{Z}}).
				\]
				Soit $\chi : \widehat{\mathbb{Z}} \to K^\times$ un caractère continu. Nous construisons une représentation linéaire associée sur un espace vectoriel $V$ de dimension $1$ sur $K$. \\ 
				Nous définissons :
				\[
				\rho_\chi : \widehat{\mathbb{Z}} \to \operatorname{GL}(V), \quad \rho_\chi(a) v = \chi(a)(v), \quad \forall v \in V.
				\]
				Puisque $\chi(a) \in \mathbb{C}^\times$, cette application définit bien une représentation irréductible de $\widehat{\mathbb{Z}}$. Nous posons alors :
				\[
				\Psi(\chi) = \rho_\chi.
				\]
				Vérifions que $\Phi$ et $\Psi$ sont réciproques.\\	
				Montrons que $\Phi \circ \Psi = \operatorname{Id}$  
				Soit $\chi \in \operatorname{Hom}_{\text{cont}}(\widehat{\mathbb{Z}}, \mathbb{C}^\times)$. Alors :
				\[
				\Phi(\Psi(\chi)) = \Phi(\rho_\chi) = \rho_\chi = \chi.
				\]
				Montrons que $\Psi \circ \Phi = \operatorname{Id}$ 
				Soit $\rho \in \operatorname{Irr}(\widehat{\mathbb{Z}})$. Alors :
				\[
				\Psi(\Phi(\rho)) = \Psi(\rho) = \rho.
				\]
				Comme chaque application est l'inverse de l'autre, nous obtenons une bijection.\\	
				Il en resulte que:
				\[ 
				\operatorname{Irr}(\widehat{\mathbb{Z}}) \simeq \operatorname{Hom}_{\text{cont}}(\widehat{\mathbb{Z}}, \mathbb{C}^\times).
				\]
				
				
				\item  Les représentations irréductibles de $\widehat{\mathbb{Z}}$ correspondent aux caractères continus.Donc
				\[ 
				\operatorname{Irr}(\widehat{\mathbb{Z}}) \simeq \operatorname{Hom}_{\text{cont}}(\widehat{\mathbb{Z}}, \mathbb{C}^\times).
				\]
				Par ailleurs, on a : 
				\[
				\operatorname{Hom}_{\text{cont}}(\widehat{\mathbb{Z}}, \mathbb{C}^\times) \simeq \mathbb{Q}/\mathbb{Z}.
				\]
				Le groupe $\mathbb{Q}/\mathbb{Z}$ est non dénombrable, ce qui prouve que $\widehat{\mathbb{Z}}$ possède une infinité non dénombrable de représentations irréductibles.
				
				\item Le groupe $\mathbb{Z}/n\mathbb{Z}$ se factorise en sous-groupes comme suit :
				\[
				\mathbb{Z}/n\mathbb{Z} \simeq \prod_{p^k \mid n} \mathbb{Z}/p^k\mathbb{Z}.
				\]
				En passant à la limite projective :
				\[
				\widehat{\mathbb{Z}} = \varprojlim \mathbb{Z}/n\mathbb{Z} \simeq \prod_{p \in \mathcal{P}} \mathbb{Z}_p.
				\]
				Ainsi, tout caractère de $\widehat{\mathbb{Z}}$ est de la forme :
				\[
				\chi : \widehat{\mathbb{Z}} \to \mathbb{C}^\times, \quad \chi((x_p)_{p \in \mathcal{P}}) = \prod_{p \in \mathcal{P}} \chi_p(x_p),
				\]
				où chaque $\chi_p$ est un caractère de $\mathbb{Z}_p$. Cela justifie l’interprétation des représentations irréductibles de $\widehat{\mathbb{Z}}$ comme des caractères sur chaque $\mathbb{Z}_p$.
				
			\end{enumerate}
			
			
			
			\newpage
			\section*{Conclusion}
			\addcontentsline{toc}{section}{Conclusion}
			Ce chapitre a exploré les interrelations profondes entre le produit tensoriel d'espaces vectoriels, les représentations linéaires de produits de groupes finis et les propriétés du complété profini $\widehat{\mathbb{Z}}$.
			Nous avons commencé par établir les fondements théoriques du produit tensoriel, en étudiant à la fois le cas de deux espaces vectoriels et son extension à des espaces arbitraires. Cette structure, fondamentale en algèbre linéaire, a fourni les bases nécessaires pour aborder les représentations linéaires des produits de groupes finis.
			La seconde section a mis en lumière les spécificités des représentations linéaires, en distinguant les cas de produits de deux groupes et de produits arbitraires. Les résultats obtenus ont été renforcés par une analyse des caractères et de l'irréductibilité de ces représentations, culminant dans une présentation du théorème de décomposition spectrale.
			Enfin, l'étude du complété profini $\widehat{\mathbb{Z}}$ a permis d'approfondir la théorie en examinant ses propriétés algébriques et topologiques, ainsi que ses représentations linéaires et leurs caractères. En particulier, nous avons identifié et analysé les représentations irréductibles de $\widehat{\mathbb{Z}}$, en mettant en évidence leur rôle central dans la théorie des groupes profinis.
			Ce chapitre illustre ainsi comment des outils fondamentaux tels que le produit tensoriel et les représentations linéaires se généralisent et s'enrichissent dans le cadre des groupes finis et profinis, offrant une compréhension unifiée et étendue des structures algébriques et topologiques sous-jacentes.
			
			
			\chapter*{CONCLUSION}
			% Partie avec la taille personnalisée
			{
				\applyfontsize % Application locale de la taille de police 12pt
				
				Ce mémoire a exploré les représentations linéaires des groupes finis et infinis, avec un accent particulier sur le complété profini des groupes. 
				Nous avons étudié les notions de base des groupes, sous-groupes, homomorphismes, espaces vectoriels et applications linéaires. Cela a permis de poser une base solide pour comprendre les représentations linéaires, leurs caractères et leurs propriétés.
				Les propriétés fondamentales des représentations irréductibles de groupes finis ont été détaillées, en mettant en évidence leur rôle dans la décomposition des représentations générales. Le lien avec les caractères, notamment à travers l’orthogonalité, a été particulièrement mis en avant.
				Une partie significative du mémoire a été dédiée à l'étude du produit tensoriel d'espaces vectoriels et de son rôle dans la construction des représentations de produits de groupes finis. Les critères d’irréductibilité et les relations entre les caractères des produits ont été caractérisés.
				L’étude du complété profini $\widehat{G}$ a permis de lier les propriétés algébriques et topologiques des groupes. Le cas particulier du complété profini de $\mathbb{Z}$ a été traité en détail, montrant la richesse de cette construction dans le contexte des représentations linéaires. Une voie naturelle serait d’étendre les concepts étudiés aux groupes localement compacts, en explorant leurs représentations unitaires et leurs applications en analyse harmonique. Les groupes profinis et leurs représentations trouvent des applications dans des domaines comme la cryptographie et la théorie des nombres. Cela pourrait constituer une extension pratique de ce travail. La simulation numérique des représentations linéaires des groupes infinis pourrait ouvrir de nouvelles perspectives, en permettant de tester expérimentalement certaines hypothèses.Nous avons donc  contribué à enrichir notre compréhension des représentations linéaires, en liant des notions abstraites à des constructions concrètes. Les résultats obtenus jettent les bases pour des recherches futures, aussi bien théoriques qu’appliquées, dans le domaine des mathématiques pures et de leurs applications.
				
			}
			\addcontentsline{toc}{section}{CONCLUSION}
			
			
		}
		
		
		% Partie avec la taille personnalisée
		{
			\applyfontsize % Application locale de la taille de police 12pt
			
			\begin{thebibliography}{99}
				
				\bibitem{Guichardet}
				Alain Guichardet.
				\newblock {\em Tensor products of $C^{*}-$Algebras, Part II. Infinite tensor products}.
				\newblock Lecture Notes Series $N^0$ 13, 1969.
				
				\bibitem{savage2018linear}
				Alistair SAVAGE.
				\newblock {\em Linear Algebra I}.
				\newblock 2018.
				
				\bibitem{deschamps}
				Bruno Deschamps.
				\newblock {\em Groupes profinis et théorie de Galois}.
				\newblock Clarendon Press, Oxford, 1998.
				
				\bibitem{schwarzweller2009chinese}
				Christoph Schwarzweller.
				\newblock {\em The Chinese Remainder Theorem, its Proofs and its Generalizations in Mathematical Repositories}.
				\newblock Studies in Logic, Grammar and Rhetoric, volume 18, number 31, pages 103--119, Bialystok University Press, 2009.
				
				\bibitem{schaub1997}
				Daniel Schaub.
				\newblock {\em Éléments de la Théorie de Groupes}.
				\newblock Cours de licence de Mathématiques, Université d’Angers, 1997/98.
				
				\bibitem{harville1997trace}
				David A. Harville.
				\newblock {\em Matrix Algebra from a Statistician’s Perspective}.
				\newblock Springer, 1997, pages 49--53.
				
				\bibitem{hararirepresentations}
				David Harari.
				\newblock {\em Représentations linéaires des groupes finis}.
				
				\bibitem{renard2009groupes}
				David Renard and Laurent Schwartz.  
				\newblock {\em Groupes et représentations}.  
				\newblock École polytechnique, 2009.
				
				
				\bibitem{Dragomir}
				Dragomir Z. Dokovic.
				\newblock {\em Pairs of Involutions in the General Linear Group}.
				\newblock Journal of Algebra, 100, 214--223, 1986.
				
				\bibitem{farhi2024polycopie}
				Farhi, Bakir.
				\newblock \textit{Polycopié d'Algèbre bilinéaire (Algèbre 4)}.
				\newblock National Higher School of Mathematics-Alger, 2024.
				
				
				\bibitem{minkowski1911gesammelte}
				Hermann Minkowski.
				\newblock {\em Gesammelte Abhandlungen}.
				\newblock BG Teubner, volume 2, 1911.
				
				\bibitem{cartan1999homological}
				Henri Cartan and Samuel Eilenberg.
				\newblock {\em Homological Algebra}.
				\newblock Volume 19, Princeton University Press, 1999.
				
				\bibitem{serre1971representation}
				Jean-Pierre Serre.
				\newblock Repr{\'e}sentation lin{\'e}aire des groupes finis.
				\newblock Hermann, Paris, 1971.
				
				\bibitem{cheung2018algebre}
				Kevin  Cheung et Mathieu Lemire.
				\newblock {\em Algèbre Linéaire et Applications}.
				\newblock 2018.
				
				
				\bibitem{ribes-zalesskii}
				Luis Ribes and Pavel Zalesskii.
				\newblock Profinite Groups.
				\newblock Springer, Berlin, 2010.
				
				
				\bibitem{bridger2001limits}
				Mark  Bridger.
				\newblock Limits: A New Approach to Real Analysis.
				\newblock Springer, New York, 2001.
				
				\bibitem{hall2018theory}
				Marshall Hall.
				\newblock {\em The theory of groups}.
				\newblock Courier Dover Publications, 2018.
				
				
				\bibitem{fried2023fieldarithmetic}
				Michael David Fried and Moshe Jarden.
				\newblock Infinite Galois Theory and Profinite Groups.
				\newblock In {\em Field Arithmetic}, pages 1--19. Springer Nature Switzerland, Cham, 2023.
				
				\bibitem{bourbaki2013general}
				Nicolas BOURBAKI. 
				\newblock {\em General topology: chapters 1--4}.
				\newblock Springer Science \& Business Media, 2013.
				
				\bibitem{pei1996chinese}
				Pei, Dingyi, Salomaa, Arto, et Ding, Cunsheng.
				\newblock {\em Chinese remainder theorem: applications in computing, coding, cryptography}.
				\newblock World Scientific, 1996.
				
				\bibitem{maclane1971categories}
				Saunders Mac Lane.
				\newblock \emph{Categories for the Working Mathematician}.
				\newblock Springer, New York, 1971.
				
				\bibitem{lang2012algebra}
				Serge Lang.
				\newblock \emph{Algebra}.
				\newblock Volume 211, Springer Science \& Business Media, 2012.
				
				
				\bibitem{axler2024linear}
				Sheldon Axler.
				\newblock {\em Linear algebra done right}.
				\newblock Springer Nature, 2024.
				
				\bibitem{ribet2004graduate}
				Sheldon Axler, Frederick Gehring , and Kenneth Alan Ribet.
				\newblock {\em Graduate Texts in Mathematics 111}.
				\newblock Springer, 2004.
				
				
				\bibitem{greub2012linear}
				Werner H. Greub
				\newblock {\em Linear Algebra}.
				\newblock Springer Science \& Business Media, volume 23, 2012.
				
				
				\bibitem{herfort2012profinite}
				Wolfgang Herfort  
				\newblock {\em Introduction to Profinite Groups}.  
				\newblock Mimar Sinan Fine Arts University, 2012.  
				
				
				
				
				
				
				
				
				
				
				
			\end{thebibliography}
			
		}
		
		
	\end{onehalfspace} 
	
	
	
	
	
	
\end{document}

































\documentclass[12pt]{article}
\usepackage{amsmath, amssymb}

\begin{document}
	
	\title{Foncteur Contravariant et Système Projectif}
	\author{}
	\date{}
	\maketitle
	
	\section*{Introduction}
	Soient \( I \) un ensemble ordonné filtrant et \( (X_i, f_{ij}) \) un système inductif dans une catégorie \( \mathcal{C} \), indexé par \( I \). Nous allons appliquer un foncteur contravariant \( F \) à ce système et montrer que cela produit un système projectif dans une autre catégorie \( \mathcal{D} \).
	
	\section*{Définition du système inductif}
	Un système inductif \( (X_i, f_{ij}) \) est constitué des objets \( X_i \) de la catégorie \( \mathcal{C} \) et des morphismes \( f_{ij}: X_i \to X_j \) pour \( i \leq j \), satisfaisant les relations suivantes :
	\begin{itemize}
		\item \( f_{ii} = \text{id}_{X_i} \), l'identité sur \( X_i \),
		\item \( f_{jk} \circ f_{ij} = f_{ik} \) pour \( i \leq j \leq k \), la compatibilité des morphismes.
	\end{itemize}
	
	Le système inductif est associé à la colimite \( \varinjlim X_i \), qui est un objet \( X \) dans \( \mathcal{C} \), muni de morphismes \( u_i: X_i \to X \), satisfaisant les propriétés universelles de la colimite.
	
	\section*{Application du foncteur contravariant}
	Soit \( F: \mathcal{C} \to \mathcal{D} \) un foncteur contravariant. Nous appliquons \( F \) au système inductif \( (X_i, f_{ij}) \). Le foncteur contravariant applique chaque morphisme \( f_{ij}: X_i \to X_j \) à un morphisme inversé dans \( \mathcal{D} \), c'est-à-dire que :
	\[
	F(f_{ij}): F(X_j) \to F(X_i)
	\]
	est un morphisme dans \( \mathcal{D} \).
	
	Ainsi, \( F \) transforme le système inductif en un système projectif dans \( \mathcal{D} \), constitué des objets \( F(X_i) \) et des morphismes \( F(f_{ij}) \).
	
	\section*{Vérification de la compatibilité dans le système projectif}
	Pour montrer que les morphismes \( F(f_{ij}) \) forment un système projectif, il nous faut vérifier la compatibilité des morphismes dans \( \mathcal{D} \), à savoir :
	\[
	F(f_{jk}) \circ F(f_{ij}) = F(f_{ik}) \quad \text{pour} \quad i \leq j \leq k.
	\]
	Cela découle directement de la compatibilité des morphismes dans \( \mathcal{C} \), où \( f_{jk} \circ f_{ij} = f_{ik} \). Par contravariance de \( F \), on a donc :
	\[
	F(f_{jk}) \circ F(f_{ij}) = F(f_{ik}).
	\]
	
	\textbf{\underline{Montrons que \( (F(X_i), F(f_{ij}) ) \) est un système projectif dans \( \mathcal{D} \) }}: \\
	
	Soit \( F: \mathcal{C} \to \mathcal{D} \) un foncteur contravariant. Nous appliquons \( F \) au système inductif \( (X_i, f_{ij}) \). On a :  
	
	\begin{itemize}
		\item Chaque morphisme \( f_{ij}: X_i \to X_j \) dans \( \mathcal{C} \) devient un morphisme
		\( F(f_{ij}): F(X_j) \to F(X_i) \) dans \( \mathcal{D} \).
		\item Identité. \( f_{ii} = \text{id}_{X_i} \) dans \( \mathcal{C} \) devient \( F(f_{ii}) = \text{id}_{F(X_i)} \). En effet, \( F \) est contravariant, donc \( F(\text{id}_{X_i}) = \text{id}_{F(X_i)} \).
		\item Pour \( i \leq j \leq k \), \( f_{jk} \circ f_{ij} = f_{ik} \) dans \( \mathcal{C} \) devient \( F(f_{ij}) \circ F(f_{jk}) = F(f_{ik}) \) dans \( \mathcal{D} \).
	\end{itemize}
	
	
	\textbf{\underline{Montrons que \( (F(X), F(u_{i}) : F(X) \to F(X_i) ) \) est une famille de morphismes }} \\	
	
	\textbf{\underline{compatibles dans \( \mathcal{D} \) }}: \\
	
	Les morphismes \( u_i: X_i \to X \) dans \( \mathcal{C} \) deviennent  \( F(u_i) : F(X) \to F(X_i) \) dans \( \mathcal{D} \). Il vient que pour tout \( i \leq j \), \( u_j \circ f_{ij} = u_i \) dans \( \mathcal{C} \) devient \( F(f_{ij}) \circ F(u_j) = F(u_i) \) dans \( \mathcal{D} \).
	
	La Propriete universelle dans \( \mathcal{C} \) garantit que Si un autre objet \( Y \) et des morphismes \( v_i: X_i \to Y \) satisfont \( v_j \circ f_{ij} = v_i \), alors il existe un unique morphisme \( v: X \to Y \) tel que \( v \circ u_i = v_i \).Ce qui implique en appliquant $F$ que 
	
	
	
	\section*{Conclusion}
	Ainsi, l'application du foncteur contravariant \( F \) à un système inductif \( (X_i, f_{ij}) \) dans \( \mathcal{C} \) donne un système projectif dans \( \mathcal{D} \). En d'autres termes, \( F(\varinjlim X_i) = \varprojlim F(X_i) \), ce qui montre que le foncteur contravariant transforme un système inductif en un système projectif.
	
\end{document}
